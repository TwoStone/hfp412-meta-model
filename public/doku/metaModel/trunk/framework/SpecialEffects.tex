%%%%%%%%%%%%%%%%%%%%%%%%%%%%%%%%%%
%             WARNUNG            %
%%%%%%%%%%%%%%%%%%%%%%%%%%%%%%%%%%

% In dieser Datei wirds richtig freakig.
% Also wenn du hier was ändern willst,
% dann solltest du wirklich Ahnung haben.
% Ansonsten lass lieber die Finger davon ;)


%%%%%%%%%%%%%%%%%%%%%%%%%%%%%%%%%%
%       Quellenverzeichnis       %
%%%%%%%%%%%%%%%%%%%%%%%%%%%%%%%%%%

\makeatletter
\renewenvironment{thebibliography}[1]
     {\subsection*{\refname}%
      \@mkboth{\MakeUppercase\refname}{\MakeUppercase\refname}%
      \list{\@biblabel{\@arabic\c@enumiv}}%
           {\settowidth\labelwidth{\@biblabel{#1}}%
            \leftmargin\labelwidth
            \advance\leftmargin\labelsep
            \@openbib@code
            \usecounter{enumiv}%
            \let\p@enumiv\@empty
            \renewcommand\theenumiv{\@arabic\c@enumiv}}%
      \sloppy
      \clubpenalty4000
      \@clubpenalty \clubpenalty
      \widowpenalty4000%
      \sfcode`\.\@m}
     {\def\@noitemerr
       {\@latex@warning{Empty `thebibliography' environment}}%
      \endlist}
\makeatother


%%%%%%%%%%%%%%%%%%%%%%%%%%%%%%%%%%
%       Anhangsverzeichnis       %
%%%%%%%%%%%%%%%%%%%%%%%%%%%%%%%%%%

\makeatletter
\newcommand*{\maintoc}{ % Hauptinhaltsverzeichnis
  \begingroup
    \@fileswfalse % Kein neues Verzeichnis öffnen
    \renewcommand*{\appendixattoc}{ % Trennanweisung im Inhaltsverzeichnis
      \value{tocdepth}=-10000 % lokal tocdepth auf sehr kleinen Wert setzen
    }%
    \tableofcontents % Verzeichnis ausgeben
  \endgroup
}

\newcommand*{\appendixtoc}{ % Anhangsinhaltsverzeichnis
  \begingroup
    \edef\@alltocdepth{\the\value{tocdepth}} % tocdepth merken
    \setcounter{tocdepth}{-10000} % Keine Verzeichniseinträge
    %\renewcommand*{\contentsname}{Anhangsverzeichnis} % Verzeichnisname ändern
    %\renewcommand*{\appendixattoc}{ % Trennanweisung im Inhaltsverzeichnis
    %  \setcounter{tocdepth}{\@alltocdepth} % tocdepth wiederherstellen
    %}%
    %\tableofcontents % Verzeichnis ausgeben
    \setcounter{tocdepth}{\@alltocdepth} % tocdepth wiederherstellen
  \endgroup
}
\newcommand*{\appendixattoc}{ % Trennanweisung im Inhaltsverzeichnis
}
%\g@addto@macro\appendix{ % \appendix erweitern
  %\newpage % Neue Seite
 %\phantomsection % Sprungmarke korrigieren
  %\addcontentsline{toc}{section}{\appendixname} % Eintrag ins Hauptverzeichnis
  %\addtocontents{toc}{\protect\appendixattoc} % Trennanweisung in die toc-Datei
%}
\makeatother