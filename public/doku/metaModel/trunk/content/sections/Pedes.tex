\section{Pedes}

Fragt sich, wie eigentlich Literaturreferenzen, Abkürzungen und Glossareinträge aussehen. Erstmal
ein bisschen was referenzieren: \refLibLiteratur{adams2001per} \refLibLiteratur{gamma1995design}
\refLibInternet{test95undso}.

%\blindtext \todo{Dies ist ein etwas längerer Text wegen Umbruch und so}

\QUESTION[Was war da denn los]

-- glsresetall --\glsresetall

\acl{ivv}

\ac{ivv}

\acl{ivv}

\ac{ivv}

\acl{ivv}


acp: Das ist doch klar, es bla ist bla dies das die \ac{ivv} oder?

acp: Das ist doch klar,\footnote{Dieses Komma ist auf jeden Fall der Hammer :D} es bla ist bla dies
das die \acsp{ivv} oder?

ac: Das ist doch klar, es bla ist bla dies das die \ac{ivv} oder?

ac: Das ist doch klar, es bla ist bla dies das die \ac{ivv} oder?

So ist das in der \ac{IT}, manchmal geht alles gut, aber in der Regel geht es in der \ac{IT} dann
doch drunter und drüber ;) Anders als bei der \acs{VGH} und der \ac{GmbH}, dort kann man
anderes beobachten.

-- glsresetall --\glsresetall

Und sowas ist eigentlich ein \gla{Webservice}{SOA-Webservice} oder so :D

\gl{Webservice} und der Plural ist \glp{Webservice}

\acp{ibA}

\acsp{ibA}

\acfp{ibA}

%\ac{ivv}

Hier müsste jetzt eigentlich ivvs rauskommen: \ace{ivv}{s}

\ac{ivv}

\ac{ivv}