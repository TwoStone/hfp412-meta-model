\section{Ausblick}
Im Laufe des Projekts sind viele Ideen aufgekommen, die aufgrund von Entwurfsentscheidungen oder Zeitmangel 
jedoch nicht umgesetzt werden konnten. Dieser kurze Ausblick skizziert eine Auswahl von Erweiterungsmöglichkeiten und 
soll nachträgliche Implementierungen anregen.

\begin{description}
  \item[Multiplizitäten] Zurzeit besitzen Assoziationen keine Einschränkungen. 
Es werden sozusagen überall $0..n$-Multiplizitäten verwendet. 
Mit der Einführung von Multiplizitäten können Assoziationen eingeschränkt werden.
Es bietet sich beispielsweise an, eine Klasse \term{Multiplicity} zwischen \term{Association} und \term{Type} zu hängen,
welche zwei boolsche Attribute $\leq$ und $\geq$ enthält. 
Damit lassen sich die vier Multiplizitäten $0..n$, $1..n$, $1$ und $0..1$ darstellen.
Dies zieht u.a. einige Konsistenzbedingungen nach sich, welche beim Anlegen von Links beachtet werden müssen. 
  \item[Mehrstellige Assoziationen] Aktuell gibt es für eine Assoziation nur genau ein Quell- und genau ein Zieltypen. 
  Um Mehrstellige Assoziationen und Links zu implementieren, könnten die Formal- und Aktualparameter verwendet werden. 
  Eine Assoziation bestünde dann aus den zwei explizit genannten Quell- und Zieltypen sowie einer Liste von Formalparametern 
  wobei jeder Formalparameter eine weitere Quelle darstellen würde. Um dies adäquat auf der Exemplarebene abzubilden, 
  würden die Links sich ihrer Aktualparameter bedienen. Das aktuelle \MM sieht diese Implementierung schon vor, da sowohl 
  Formal- als auch Aktualparameter von den jeweiligen Oberklassen \term{AbstractOperation} und \term{Message++Link} ausgehen.
  \item[Gleichheit von Operationen] Zum derzeitigen Entwicklungsstand sind zwei Operationen gleich sobald sie den gleichen Namen haben. Dies ist in vielen Fällen nicht
  sinnvoll, da zwei Operationen keinerlei Bezug zueinander haben wenn sie in unterschiedlichen Typen liegen. Selbst dann nicht, wenn sie gleich heißen. 
  Ein praktikablereres Konzept ist, dass zwei Operationen gleich sind sobald sie die gleiche Quelle, den gleichen Namen und die gleichen Parameterlisten (also Anzahl der Parameter und Typisierung dieser) haben. 
  \item[Messages] Wie schon im Kapitel \ref{Message:Message} erläutert, wurden Messages in der ersten Implementierungsstufe nicht umgesetzt. 
  Zum gegenwärtigen Zeitpunkt existiert für ein Message-Exemplar sowohl die Interpretation des Nachrichtenaufrufs, als auch des Nachrichteninhalts.
Bevor Messages umgesetzt werden können, muss sich geeinigt werden, welche der genannten Interpretationen für ein Message-Exemplars zutrifft. 
  \item[ActualParameter] Da aus genannten Gründen weder Message-Exemplare erzeugt, noch mehrstellige Assoziationen/Links angelegt werden können, 
  haben Aktualparameter derzeit keinerlei Daseinsberechtigung. Aus diesen Gründen wird die Verwaltung von Aktualparametern dem Anwender nicht angeboten.
  \item[Zentrale Ablage der Constraints] Durch die durch die Einführung von \term{ModelItem} geschaffene Möglichkeit einen Abhängigkeitsgraphen 
		auszuwerten, können alle Modellelemente ermittelt werden, deren Konsistenz nach einer Änderung zu prüfen ist. Wenn alle Konsistenzbedingungen 
		in einer zentralen Komponente abgelegt sind, ist es möglich nach einer Änderung einfach die Konsistenz aller (transitiv) abhängigen Modellelemente zu überprüfen. 
		Da alle Änderungen in Transaktionen geschehen, kann bei einem inkonsistenten Folgezustand ein Rollback eingeleitet werden.
		Großer Vorteil ist, dass veränderliche Elemente keinerlei Kenntnis über die Konsistenzbedingungen ihrer abhängigen Modellobjekte mehr benötigen.
\end{description}

Weitere Themen neben den o.g. sind beispielsweise Versionierung, Posting Rules, Path. Diese werden in diesem Dokument jedoch nicht weiter ausgeführt.