\section{Ausblick}

\TODO[Ideen Formulieren]


\begin{description}
  \item[Multiplizitäten] Zurzeit besitzen Assoziationen keine Einschränkungen. 
Es werden sozusagen überall $0..n$-Multiplizitäten verwendet. 
Mit der Einführung von Multiplizitäten können Assoziationen eingeschränkt werden.
 
Es bietet sich beispielsweise an, eine Klasse Multiplicity zwischen \term{O++A} und \term{Type} zu hängen,
welche zwei boolsche Attribute $\leq$ und $\geq$ enthält. 
Damit lassen sich die vier Multiplizitäten $0..n$, $1..n$, $1$ und $0..1$ darstellen.
Dies zieht u.a. einige Konsistenzbedingungen nach sich, welche beim Anlegen von Links beachtet werden müssen. 
  \item[Mehrstellige Assoziationen]
  \item[Messages] Wie bereits im Abschnitt \ref{Messages} beschrieben, sind Messages zwar im Metamodell enthalten, 
  eine Implementierung wurde in dieser Ausbaustufe allerdings nicht umgesetzt. 
  \item[Path]
  \item[Posting Rules]
  \item[Versionierung]
  \item[Zentrale Ablage der Constraints]
\end{description}
  