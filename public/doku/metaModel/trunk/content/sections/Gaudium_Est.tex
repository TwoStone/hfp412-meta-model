\section{Gaudium Est}

Das hier ist ein Blindtext zum Testen von Textausgaben. Wer diesen Text liest, ist selbst Schuld. Wo
kommt eigentlich der Zeilenumbruch? Zicke zacke Hühnerkacke.

\subsection{Unterkapitel}

Ein Mensch, der sagen wir als \emph{Christ}, streng gegen Mord und Totschlag ist, hält einen Krieg,
wenn überhaupt, nur gegen Heiden für erlaubt. Ein anderer Mensch, ein frommer Heide, tut keinem Menschen
was zu leide. Nur gegenüber Christenhunden, wär jedes Mitleid falsch empfunden. Der ewige Kriege
blutiger Spur, kommt nur von diesem kleinen \qq{nur}. \refLibInternet[S.~95~ff.]{test95undso}

Weiteres siehe \refSec{Ukp} ne?!

\subsubsection{Unterunterkapitel}\lblSec{Ukp}

\TODO[Ein dickes Todo]

Genau sehen wir das in \refImg[Abb.~]{KundeDerIvv} auf \refPageImg{KundeDerIvv}.

\img[width=\relWidth{0.75}]{Logo_ivv_FHDW.png}{Dies ist die Bildunterschrift.}{KundeDerIvv}

Wir arbeiten alle gern mit der \acs{EDV} oder nicht? Doch \acl{EDV} ist toll, weil \ac{EDV} toll
ist. \ac{EDV} Dies das bla. \acf{EDV}. Man kann das auch oder? Für Lars gibts jetzt Glossareinträge:
Und zwar mag Lars die \glp{Xena} oder etwa nicht?

\tbl{Anwendungsfälle}{anwendungsfaelle}{|l||p{\relWidth{0.50}}|}{
	\thl{Name}					&	\thl{<Hier könnte Ihr Text stehen>}	\tr
	Auslöser					&	<Hier könnte Ihr Text stehen>	\tr
	Beteiligte Akteure			&	<Hier könnte Ihr Text stehen>	\tr
	Vorbedingungen				&	\blindtext						\tr
	Eingehende Informationen	&	<Hier könnte Ihr Text stehen>	\tr
	Ergebnis					&	<Hier könnte Ihr Text stehen>	\tr
	Nachbedingungen				&	<Hier könnte Ihr Text stehen>	\tr
	Ablaufbeschreibung			&	<Hier könnte Ihr Text stehen>	\tr
	Ausnahmesituation			&	<Hier könnte Ihr Text stehen>	\tr
}{\defaultTblBg}
