\subsection{Measurement}

Martin Fowler unterscheidet in seinem Werk \refLibLiteratur{fowler1997analysis} zwischen qualitativen und quantitativen
Messungen. Wird beispielsweise die Blutgruppe (A, B, AB und 0) eines Menschen typisiert, so handelt es sich um
eine qualitative Messung. Im Kontext dieses Projekts wird diese Art von Messung als \term{Observation} (siehe Abschnitt
\ref{Operation:Associationen}) bezeichnet. Unter quantitativen Messungen, bezeichnet als \term{Measurement}, wird
beispielsweise das Feststellen der Körpergröße eines Menschen verstanden.

\term{Measurements} (beispielsweise Körpergröße von Hugo Egon Balder: 190cm) werden in Form von Quantitäten bzw.
\term{Quantities} auf Konten (\term{Accounts}) erfasst. Des Weiteren gibt es \term{MeasurementTypes} und
\term{AccountTypes}, welche in diesem Abschnitt noch genauer erläutert werden.

Technisch gesehen befinden sich auf der Modellebene im Bereich \term{Measurement} zunächst drei Klassen:
\term{MeasurementType}, \term{AccountType} und deren gemeinsame Oberklasse \term{QuantifiedObjectType}. Auf der
Exemplarebene finden sich deren Pendants als \term{Measurement}, \term{Account} und \term{QuantifiedObject}.

Die Operation \term{aggregate} des \term{QuantifiedObjects} fordert zudem eine sogenannte \term{AggregationStrategy},
die in vier Unterklassen konkret ausgeprägt werden kann.
Zusätzlich existieren drei Manager: der \term{MeasurementTypeManager}, der \term{AccountTypeManager} und der
\term{AccountManager}. Diese Klassen werden im Folgenden nach Themengebieten beschrieben.


\subsubsection{QuantifiedObject}\lblSec{Measurement:QuantifiedObject}

Das \term{QuantifiedObject} repräsentiert zunächst einmal ganz abstrakt ein Objekt, das mit einer Quantität versehen
ist. Zu diesem Zweck kennt es ein \term{Object}, auf das es sich bezieht und direkt oder indirekt eine \term{Quantity}.
Als konkrete Unterklassen hat es \term{Measurement} und \term{Account}. Ein \term{Measurement} ist dabei die
direkte Ausprägung eines quantifizierten Objekts, es kennt seine \term{Quantity}. 

Der \term{Account} kennt seine \term{Quantity} hingegen nur indirekt. Sie ergibt sich aus dem Aggregieren der Einträge
\term{entries} des \term{Accounts}. Auf welche Art und Weise aggregiert wird, bestimmt dabei eine
\term{AggregationStrategy}. Diese kann durch Ausführen der Operation \term{aggregate} auf ein \term{QuantifiedObject}
angewandt werden.

Ein \term{QuantifiedObject} ist zudem in der Modellebene in einem \term{QuantifiedObjectType} typisiert.


\subsubsection{QuantifiedObjectType}\lblSec{Measurement:QuantifiedObjectType}

Ein \term{QuantifiedObjectType} kategorisiert \term{QuantifiedObjects} nach zwei Kriterien: dem \term{Type}, in dem das
zum \term{QuantifiedObject} gehörende \term{Object} typisiert ist, und dem \term{AbstractUnitType}, in dem die \term{AbstractUnit}
der zum \term{QuantifiedObject} gehörenden \term{Quantity} typisiert ist. Er bildet somit die abstrakte Oberklasse des
\term{MeasurementTypes} und des \term{AccountTypes} und hat keine Operationen.


\subsubsection{Account}

Ein Konto (Account) gehört zu einem bestimmten Objekt (Object) und beinhaltet Einträge (entries) in Form von quantitativen Messungen (Measurements). 
Jedes Konto ist in einem Typ (AccountType) klassifiziert und kann außerdem mehrere Unterkonten (Komposition von subAccounts) besitzen.
Auf einem Konto kann auf Basis einer definierten Strategie (AggregationStrategy) aggregiert werden. 
Beispielsweise kann so Summe, Durchschnitt, Maximum oder Minimum der Einträge errechnet werden. 

Im Metamodell besitzt ein Account die folgenden Assoziationen.
\begin{description}
	\item[MAccountType type] Typ des Accounts.
	\item[Account** subAccounts hierarchy AccountHierarchy] Liste von Unterkonten.
	\item[Measurement** entries] Liste von Einträgen.
	\item[MObject object] Von QuantifiedObject geerbt.
\end{description}

Hinzu kommen die folgenden Operationen, welche zum Befüllen der o.g. Listen dienen.
\begin{description}
	\item[addEntry] Fügt einen neues Measurement in die entries-Liste hinzu. Das Element muss vom selben Typ sein wie die anderen Einträge der Liste (Konsistenzbedingungen folgen weiter unten).  
	\item[addSubAccount] Fügt einem Konto ein neues Unterkonto hinzu. Hierbei muss der Typ des neuen Unterkontos $\leq$ dem Typ des Hauptkontos sein (Konsistenzbedingungen folgen weiter unten).
\end{description}


\subsubsection{AccountType}

Der Typ eines Accounts (AccountType) dient zur Klassifizieren eines Kontos. Beispielsweise kann ein Konto als Bankkonto klassifiziert werden. Untertyp (SubAccountType) kann an dieser Stelle z.B. ein Giro- oder Kreditkartenkonto sein.
Sobald ein AccountType Exemplare besitzt, kann dieser nicht mehr gelöscht werden (siehe folgender Abschnitt). Um Hierarchien dieser Art abbilden zu können sind die folgenden Assoziationen und Operationen notwendig. 

Im Metamodell besitzt ein AccountType (neben der vom \term{QuantifiedObjectType} geerbten) die folgende Assoziation.
\begin{description}
	\item[MAccountType** subAccountTypes hierarchy MAccountTypeHierarchy] Diese Liste verbindet AccountTypes in Form einer Hierachie. Besonders bei dem Hinzufügen von SubAccounts ist diese Hierarchie relevant (dazu mehr im folgenden Abschnitt).
\end{description}

Hinzu kommt die folgende Operation, welche zum Befüllen der o.g. Liste dient.
\begin{description}
	\item[addSubAccountType] Fügt einen neuen AccountType der subAccountTypes Liste eines AccountTypes hinzu.
\end{description}


\subsubsection{Account \& AccountType}

Wie bereits beschrieben, ist ein Account in einem AccountType klassifiziert. Dieser Typ ist unter anderem relevant, wenn dem Konto ein neues Unterkonto hinzugefügt werden soll.  
In \refImg{Measurement:AccTypeHier} wird eine AccountType Hierarchie dargestellt. \term{A} ist in diesem Beispiel der oberste AccountType, welcher die darunterliegenden SubAccountTypes referenziert. 
Auf \term{F} wird von keinem AccountType referenziert und refernziert auch selbst keine anderen AccountTypes.  

\img[width=\relWidth{0.4}]{measurement/AccountTypeHierarchy.png}{AccountType Hierarchie}{Measurement:AccTypeHier}

Nun sollen Exemplare (also Accounts) erstellt werden, welche SubAccounts enthalten. In \refImg{Measurement:SubAccounts} können die im Folgenden beschriebenen Fälle nachvollzogen werden.
Der Einfachheit halber wird mit \term{K1:A} ein Account \term{K1} mit AccountType \term{A} bezeichnet. Da $C \leq A$ ist, darf \term{K2} als SubAccount zu \term{K1} hinzugefügt werden.
In den anderen drei Fällen ist der AccountType vom SubAccount entweder nicht $\leq$ dem Typen des Hauptkontos (Beispiel 2 und 4, von links) oder aber der AccountType ist in einer anderen Hierarchie (Beispiel 3, von links).

\img[width=\relWidth{0.5}]{measurement/SubAccounts.png}{SubAccount Exemplare}{Measurement:SubAccounts}

Daraus ergibt sich die folgende Konsistenzbedingung, welche bei Nichteinhaltung zu einer \term{ConsistencyException} führt.
\begin{equation} \forall \, a \in Account: \forall s \in a.subaccounts: s.type \leq a.type
\end{equation}


\subsubsection{AccountManager}

Der AccountManager dient zur Erstellung von Accounts und beinhaltet alle Accounts des Modells.

Im Metamodell besitzt der AccountManager die folgende Assoziation.
\begin{description}
	\item[Account** accounts] Diese Liste beinhaltet alle Accounts des Modells.
\end{description}

Hinzu kommt die folgende Operation, welche zum Befüllen der o.g. Liste dient.
\begin{description}
	\item[createAccount] Dient zum Erstellen von Accounts. Es wird ein Name, ein \term{MAccountType} sowie ein \term{MObject} übergeben, auf welches sich das Konto bezieht.
\end{description}

Neben den Accounts werden auch die \term{AccountTypes} zentral verwaltet. Dazu dient der \term{AccountTypeManager}.


\subsubsection{AccountTypeManager}

Der AccountTypeManager dient zur Erstellung von \term{AccountTypes} und beinhaltet alle \term{AccountTypes} des Modells.

Im Metamodell besitzt der \term{AccountTypeManager} die folgende Assoziation.
\begin{description}
	\item[MAccountType** accountTypes] Beschreiben
\end{description}

Hinzu kommt die folgende Operation, welche zum Befüllen der o.g. Liste dient.
\begin{description}
	\item[createAccountType] Dient zum Erstellen von \term{AccountTypes}. Es wird ein Name, ein \term{MType} und ein \term{UnitType} übergeben.
\end{description}
\TODO[Alex: Brauchen wir hier noch eine Konsistenzbedingung bzgl. Account-AccountType-UnitType = Account.entries-Measurement.Quantity-Type???]

Wie bereits beschrieben werden in Accounts Messungen in Form von Measurements erfasst. 
Im folgenden Abschnitt wird auf die Assoziationen und Operationen von Measurements genauer eingegangen.
\TODO[Überleitung via account.entries - passt so?]


\subsubsection{Measurement}

Ein \term{Measurement} repräsentiert eine quantitative Messung eines \term{Objects}. So kann mit ihr zum Beispiel die Regensäule des gestrigen Tages
oder eine Buchung auf ein Konto festgehalten werden. Damit dies funktioniert, benötigt ein \term{Measurement} zunächst ein \term{Object}, auf das es
sich beziehen kann. Dieses erbt es von \term{QuantifiedObject} (s.~\refSec{Measurement:QuantifiedObject}).

Zum anderen benötigt es die Höhe des Betrages und die Einheit, in der die Messung durchgeführt wurde. Diese Informationen speichert das \term{Measurement}
in einer \term{AbstractQuantity}. Alle \term{Units} dieser \term{AbstractQuantity} sind dabei im \term{UnitType} des \term{MeasurementTypes} typsiert,
auf den das \term{Measurement} zeigt. Das \term{Measurement} selbst hat keine nicht"=geerbten Operationen.


\subsubsection{MeasurementType}

Ein \term{MeasurementType} kategorisiert mehrere \term{Measurements}. Dabei greift er auf dieselben Kriterien wie die Oberklasse \term{QuantifiedObjectType}
zurück (s.~\refSec{Measurement:QuantifiedObjectType}). Er hat keine nicht"=geerbten Assoziationen oder Operationen.


\subsubsection{Measurement \& MeasurementType}

\TODO[Bj: Verschieben wa? :)]
\begin{equation} \forall \, m_1, m_2 \in Account.entries: m_1.type = m_2.type
\end{equation}


\subsubsection{MeasurementTypeManager}

Der \term{MeasurementTypeManager} dient zur Verwaltung und Erstellung aller \term{MeasurementTypes}. Er besitzt dazu eine Liste \term{measurementTypes},
in der er die existierenden \term{MeasurementTypes} vorhält. Zusätzlich hat er folgende Operation:

\begin{description}
	\item[createMeasurementType] Erzeugt einen neuen \term{MeasurementType} anhand der übergebenen Parameter und gibt ihn zurück.
\end{description}


\subsubsection{AggregationStrategy}

Die \term{AggregationStrategy} wird -- wie bereits in \refSec{Measurement:QuantifiedObject} angedeutet -- von der Operation \term{aggregate} benutzt,
um die \term{AbstractQuantity} eines \term{QuantifiedObjects} zu bestimmen. Dazu besitzt sie die abstrakte Operation \term{aggregateMeasurements}, die eine 
berechnete \term{AbstractQuantity} zurückliefert. Hierzu benötigt die Operation als Parameter die Liste der zu aggregierenden \term{Measurements} und
zusätzlich einen \term{AbstractUnitType}, dessen \term{defaultUnit} für das neutrale Element der Aggregation benutzt wird. 

\term{AggregationStrategy} besitzt vier vorgefertigte, konkrete Strategien zur Aggregation: \term{SumAggregationStrategy}, \term{MinAggregationStrategy},
\term{MaxAggregationStrategy} und \term{AvgAggregationStrategy}. \TODO[Konkrete Strategien beschreiben]


\subsubsection{Löschen im Bereich Measurement}

\TODO[Kapitel schreiben]

-- Was ist löschbar?
-- Was nicht? Warum?