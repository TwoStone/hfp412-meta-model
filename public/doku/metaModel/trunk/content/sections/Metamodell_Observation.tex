\subsection{Observation}\label{Observations}
In der ursprünglichen Modellierung handelte es sich bei einer \term{Observation} um eine Assoziation bei der das Ziel auf einen 
Typ zeigte, der nur aus \term{Singletons} bestand. Damit dies aber möglich ist, müssen in der Typ-Hierarchie alle Knoten 
abstrakt und alle Blätter \term{Singletons} sein. Diese Bedingung muss aber für allgemeine Singletons nicht gelten. Um an dieser 
Stelle eine klare Trennung zu erreichen und die Transparenz des Modells zu erhöhen, wurden \term{Observation} als eigener Modellabschnitt ausgelagert.

\img[width=\relWidth{1}]{observation/observation.pdf}{Ausschnitt Observations}{img_observations}

\term{Observations} werden auf Modellebene durch die Klassen \term{Enum} und \term{ObservationType} repräsentiert. 
Zum Interpretieren der \term{Observations} dienen Enumerationen. Das bedeutet, dass jeder konkreten \term{Observation}
ein Wert einer Enumeration zugewiesen wird. Die Klasse \term{Enum} typisiert dabei die Enumeration. 
Im aktuellen Entwicklungsstand ist es nicht vorgesehen, \term{Enumerations} zu spezialisieren.

Ein Beispiel für einen \term{Enum}-Typ ist \qq{Blutgruppe}. Die Klasse \term{ObservationType} stellt auf Modellebene die 
Typisierung einer \term{Observation} dar. Dabei wurde festgelegt, dass jede \term{Observation} genau einer Enumeration 
und genau einem Typen zugeordnet sein muss. Es stellt sich die Frage, warum nicht mehrere Typen zugeordnet werden können. 
Der Grund dafür ist, dass dies bereits über komplexe Typen definiert und somit an dieser Stelle darauf verzichtet werden kann.

Auf der Exemplarebene werden \term{Observations} mit den Klassen \term{EnumValue} und \term{Observation} abgebildet. 
Die Klasse \term{EnumValue} spiegelt dabei einen konkreten Enumerationswert (z.B. \qq{Blutgruppe A}) wieder, 
wobei die Zuordnung zur Enumeration über die \term{theType}-Assoziation erfolgt. Die Klasse \term{Observation} definiert eine konkrete Observation. 
Sie ist typisiert in einem \term{ObservationType} und hat jeweils eine Zu-eins-Assoziation zu einem konkreten Enumerationswert sowie einem konkreten Objekt.

Es ergeben sich die folgenden Konsistenzbedingungen:
\begin{itemize}
  \item \term{Enum}-Objekte dürfen nur gelöscht werden wenn weder \term{EnumValue} noch \term{ObservationType} Objekte diese referenzieren.
  \item \term{ObservationType}-Objekte dürfen nur gelöscht werden, wenn es keine Ausprägungen auf der Exemplarebene gibt (\term{Observations}).
  \item Der Name von \term{EnumValue}-Objekten muss innerhalb des selben Typs (\term{Enum}) eindeutig sein. Somit soll verhindert werden, dass beispielsweise zweimal der Wert \qq{Blutgruppe A} einer Enumeration \qq{Blutgruppe} zugewiesen werden kann.
\end{itemize}

Für die \term{Observation}-Objekte müssen folgende Bedingungen gelten:
\begin{equation}\forall o \in Observation: o.theType.enumType = o.enumValue.theType
\end{equation}
\begin{equation}\forall o \in Observation: o.theObsObject.type \leq o.theType.theType
\end{equation}

Für die Erzeugung, das Löschen und die Verwaltung von \term{Observations} wurden die folgenden Manager definiert:
\begin{description}
\item[EnumerationManager] Erzeugen und Löschen von \term{Enumerations}.
\item[ObsTypeManager] Erzeugen und Löschen von \term{ObservationTypes}.
\item[ObservationManager] Erzeugen und Löschen von \term{Observations}.
\item[EnumValueManager] Erzeugen und Löschen von Enumerationswerten.
\end{description}
