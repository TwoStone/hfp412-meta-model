\subsection{Operations und Associations}


\subsection{Messages und Links}

Messages und Links haben zwar eine gemeinsame Oberklasse (MessageOrLink), dennoch wurden in der
ersten Ausbaustufe lediglich die Links implementiert. Zu den Messages existiert eine rudimentäre
Implementierung die dem Anwender aber über die Oberfläche nicht bereitgestellt wird, da die 
Semantik eines Messageexemplars bislang nicht eindeutig geklärt ist. \vspace{15pt}


Ein Link ist eine konkrete Ausprägungen einer Assoziation. 
Folglich ist die Erstellung eines Links nur möglich, wenn eine Assoziation als Typ gewählt wurde und der zu erstellende Link
sich an die  daraus ergebenden Konsitenzbedingungen hält.
Für Links stehen dem Anwender nur zwei Funktionen zur Verfügung.

\subsubsection{Einen Link erstellen - createLink}
Wie zuvor erwähnt, ist es für einen Link zwingend notwendig sich an die Konsistenzbedingungen der Assoziation zu halten.
Bei der Erstellung eines Links wird eine \emph{ConsistencyException,} wenn die Quell- und/oder Zielobjekte gemäß der Typebene nicht 
gestattet sind.

\begin{enumerate}
\item $\forall \, l \in Link: l.source.type.isLessOrEqual(l.type.source)$
\item $\forall \, l \in Link: l.target.type.isLessOrEqual(l.type.target)$
\end{enumerate}

Des weiteren is es Assoziationen möglich an mehreren Hierarchien teilzunehmen. Zyklische Assoziationen sind auf der Metaebene möglich, 
da sich sonst keinerlei Aggregationsbeziehung abbilden ließe.
Auf der Ebene der Exemplare hingegen, müssen \emph{alle Links einer Assoziation in allen Hierarchien zyklenfrei sein.}
Wenn ein Anwender versucht zyklische Links zu erzeugen, wird eine \emph{CycleException} geworfen und der jeweilige Link wird nicht erstellt.

\subsubsection{Löschen eines Links - removeLink}
Beim Entfernen eines bestehenden Links gibt es keinerlei Restriktionen zu beachten. Der Anwender kann bestehende Links beliebig entfernen.