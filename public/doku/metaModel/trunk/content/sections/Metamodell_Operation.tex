\subsection{Operations und Associations}

Zur Abbildung von Operationen und Assoziationen wurde im Zuge der Projektarbeit die Klasse \term{AbstractOperation} entwickelt von der sowohl Operation als auch Assoziation erben. 
Da es sich bei Assoziationen um spezielle, parameterlose Operationen handelt, wurde eine gemeinsame Oberklasse gewählt.
Sowohl Operationen als auch Assoziationen haben genau einen Quell- und einen Zieltyp, wobei diese in den unterschiedlichen Kontexten (Operation oder Assoziation) semantisch unterschiedlich interpretiert werden. 
Im Folgenden werden sowohl die Modell- als auch Exemplarebene erläutert und alle Operationen beschrieben, die es dem Anwender erlauben Operationen und Assoziationen 
sowie Messages\footnote{Wie im Folgenden erläutert wird, sind Messages nicht Bestandteil der ersten Implementierungsstufe. Sie wurden nur aus Gründen der Vollständigkeit mit aufgeführt.} und Links zu verwalten.  

\img[width=\relWidth{1.0}]{operation/Ausschnitt_Operation.pdf}{Klassendiagramm - Ausschnitt Operation (vereinfacht)}{img_operation}
\subsubsection{Operationen}

Die Klasse Operation repräsentiert die Abbildung von Operationen in dem erarbeiteten Modell.
Wie im Modell (siehe \refImg{img_operation}) ersichtlich, hat die Oberklasse \term{AbstractOperation} zwei Assoziationen zu \term{Type}. 
Im Kontext von Operationen ist die Quelle (source) der Typ, welcher die Operation enthält. 
Das Ziel (target) einer Operation entspricht dem Rückgabetypen.
Neben Quell- und Zieltypen besitzt eine Operation eine Liste von Formalparametern, welche 
wiederum über die Klasse \term{Type} typisiert sind.

Die Klasse Operation enthält nur genau zwei Operationen.
Die Operation \textbf{isStatic()} gibt genau dann true zurück, wenn der Quelltyp dieser Operation der leeren Summe entspricht.
Zusätzlich zu der Bedingung aus isStatic gilt für die zweite Operation \textbf{isConstant()}, dass die Operation keine Parameter enthalten darf. 


\subsubsubsection{OperationManager}

Der OperationManager beinhaltet zwei Listen zur Verwaltung von Operationen und Parametern.

\begin{description}
\item[Operation** operations] Diese Liste enthält alle Operationen, die der Manager verwaltet.
\item[FormalParameter** formalParameters] Hier werden alle Formalparameter abgelegt, die der Manager verwaltet.
\end{description}

Um diese Listen verwalten zu können, stellt der Operationmanager 10 transaktionale Operationen bereit.
Im Folgenden werden diese Operationen kurz beschrieben, wobei sowohl \textbf{Parameter} als auch \term{Exceptions} durch entsprechende Formatierungen hervorgehoben werden.

\begin{description}
\item[createOperation]
Diese Operationen dient zum Erstellen einer neuen Operation. Um die Operation modellkonform erstellen zu können, sind vier Parameter notwendig:
Die ersten zwei Parameter legen fest, zu welchem Typen diese Operation erstellt werden soll \textbf{(source)} bzw. welcher Typ als Rückgabetyp dient \textbf{(target).}
Die Bezeichnung der Operation wird über den dritten Parameter \textbf{(name)} festgelegt. 
Zur Definition der Parameterliste der zu erstellenden Operation dient der vierte Parameter \textbf{(fp).}
Wie bei den meisten Klassen dieses Projekts, die eine Bezeichnung haben, dürfen auch Operationsnamen nicht mehrfach verwendet werden. Beim Versuch zwei Operationen mit identischem Namen anzulegen, wird eine \emph{DoubleDefinitionException} erzeugt und die Erstellung wird abgebrochen.
\item[createStaticOperation]
Diese Operation bietet dem Anwender eine komfortable Möglichkeit eine statische Operation zu erstellen. Die Quell-Assoziation wird dazu auf EmptyTypeDisjunction gesetzt und es ist nur noch \textbf{der Rückgabetyp, der Name und die Parameterliste} auswählbar. Durch die Delegation gelten exakt die gleichen Konsistenzbedingungen wie beim Erstellen eine Operation (createOperation).
\item[createConstant]
Mithilfe dieser Operation ist es dem Anwender möglich auf einfache Art eine Konstante zu erstellen. Da eine Konstante eine Operation ist, die eine leere Parameterliste hat und als Quelle auf EmptyTypeDisjunction verweist, sind das genau die Parameter die beim Aufruf nicht mehr angegeben werden müssen bzw. können. Auch hierbei handelt es sich um Delegation an die createOperation-Operation und es gelten genau die gleichen Konsistenzbedingungen.
\item[createVoidOperation]
Die createVoid-Operation ist die vorerst letzte Operation zum benutzerfreundlicheren Erstellen einer Operation. Anders als bei den zwei Operationen zuvor, sind hier Quelle und Parameterliste wählbar. Da die zu erstellende Operation aber keinen Rückgabetypen haben soll, ist das Ziel vordefiniert und kann vom Anwender beim Aufruf nicht angegeben werden. Genau wie zuvor gelten auch hier die Konsistenzbedingungen der createOperation-Operation.
\item[createFormalParamater]
Um Formalparameter für Operationen verwenden zu können, müssen diese zuvor erstellt werden. Genau hierfür wird dem Anwender die createFormalParameter-Operation angeboten.
Zum erfolgreichem Erstellen eines Formalparameters ist es notwendig, diesem einen \textbf{Namen} sowie einen \textbf{Typen} zu geben. Nach erfolgreicher Erstellung ist der erstellte Formalparameter in der zuvor beschriebenen Formalparameter-Liste zu finden. Die Erstellung kann entweder an der mehrfachen Verwendung des Formalparameter-Namens oder der Typisierung des Parameters in EmptyTypeDisjunktion scheitern. Im ersten Fall wird eine \emph{DoubleDefinitionException} erstellt, im Zweiten eine \emph{ConsistencyException} wobei beide Exceptions zum Abbruch führen.
\item[addFormalParameter]
Um Formalparameter nachträglich einer Operation zuzuweisen, benötigt diese Operation sowohl die \textbf{Operation} an sich als auch den hinzuzufügenden \textbf{Formalparameter.} Sollte dieser Formalparameter nicht schon in der Formalparameterliste der gewählten Operation sein, wird dieser hinzugefügt. Im anderen Fall wird eine \emph{ConsistencyExceptionException} erstellt und der Vorgang wird abgebrochen.
\item[addMultipleFormalParameter]
Diese Operation dient dem Zweck eine Liste von Formalparametern der gewählten Operation zuzuweisen.
Zu diesem Zweck ist es notwenig die \textbf{Liste von Formalparametern} sowie 
die gewählte \textbf{Operation} an die addMultipleFormalparameter-Operation zu übergeben. 
Aufgrund von Delegation gelten auch hier die Konsistenzbedingung der addFormalParamter-Operation. 
\item[removeFormalParameterFromOperation]
Wenn ein Formalparameter aus der Parameterliste einer Operation entfernt werden soll, stellt diese Operation die notwendige Funktionalität bereit. Zum Entfernen eines \textbf{Formalparameters} aus einer \textbf{Operation} sind diese beiden Angaben essentiell. Sofern der zu entfernende Formalparameter in der Formalparameterliste der gewählten Operation ist und es keine Exemplare zu der Operation gibt, wird der Parameter aus der Parameterliste entfernt.
Sollte der Parameter nicht in der Parameterliste der Operation sein oder es Exemplare zu der Operation geben, wird eine \term{ConsistencyException} erzeugt.

Anmerkung: Zum aktuellen Zeitpunkt kann der zweite Fall nicht eintreten, da sich Message-Objekte nicht an der Oberfläche erstellen lassen.
\item[removeFormalParameter]
Um einen Formalparameter vollständig aus dem System zu entfernen, benötigt diese Operation den \textbf{Formalparameter}. Sofern es keine Exemplare zu diesem Parameter gibt und keine Operation diesen Formalparameter in seiner Parameterliste hat, wird der gewählte Formalparameter aus dem System entfernt. Im anderen Fall wird eine \emph{ConsistencyException} erstellt und der Löschvorgang abgebrochen.
\item[removeOperation]
Mithilfe dieser Operation lassen sich Operationen aus dem System entfernen. Zur erfolgreichen Durchführung wird lediglich die zu entfernende \textbf{Operation} benötigt. Das einzige Abbruchkriterium dieser Operation ist die Existenz eines Exemplars zu der gewählten Operation. Da sich zum aktuellen Entwicklungszeitpunkt Messages nicht erzeugen lassen, kann dies nicht auftreten.

\end{description}

\newpage
\subsubsection{Assoziationen}\label{Operation:Associationen}

Die Klasse Association repräsentiert die Möglichkeit Assoziationen zwischen zwei beliebigen Typen abzubilden.
Neben dem Quell- und Zieltypen, kann eine Assoziation an beliebig vielen Hierarchien teilnehmen.
Wie bei vielen anderen Modellbestandteilen, sind auch die Namen der Assoziation indiziert. Daraus folgt auch, dass es keine gleichnamigen 
Assoziationen geben darf.

Wie im digitalen Anhang dieser Dokumentation zu sehen ist, können sowohl Operationen als auch Assoziationen Formalparameter beinhalten. Die Möglichkeit der Zuweisung von
Formalparametern an Assoziationen, wird dem Anwender über die Oberfläche nicht bereitgestellt. 
Formalparameter für Assoziationen können als n-stellige Assoziationen interpretiert werden.
Die Möglichkeit der Zuweisung von Formalparametern an Assoziationen wird dem Anwender über die Oberfläche nicht zur Verfügung gestellt, da sie nicht Teil der ersten Implementierungsstufe sind.

\subsubsubsection{AssociationManager}

Der AssoziationManager bietet sechs transaktionale Operationen an und beinhaltet zwei Listen, welche er verwaltet.
\begin{description}
\item[Association ** associations] In dieser Liste werden alle erstellten Assoziationen aufbewahrt und durch entsprechende Operationen verwaltet.
\item[Hierarchy ** hierarchies] Erstellte Hierarchien werden in dieser Liste abgelegt.
\end{description}
Diese beiden Listen werden dem Anwender an der 
Oberfläche präsentiert. Über entsprechende Operationen kann der Anwender die Elemente dieser Listen verwalten.

Wie schon bei dem OperationManager, werden \textbf{Parameter} und \term{Exceptions} durch entsprechende Formatierungen kenntlich gemacht.

\begin{description}
\item[createAssociation] Bei Aufruf dieser Operation müssen \textbf{Quell- und Zieltyp} sowie der \textbf{Name} der zu erstellenden Assoziation 
angegeben werden.
Sollte der Name schon von einer anderen Assoziation verwendet werden, erzeugt die Operation eine \term{DoubleDefinitionException} und die Assoziation wird nicht
erstellt. Außerdem darf eine Assoziation nie in der leeren Disjunktion als Quelle oder Ziel typisiert sein. In einem solchen Fall
wird eine \term{ConsistencyException} erstellt.
Wenn keine der genannten Exceptions aufgetreten ist, wird die Assoziation erstellt und zur Liste der bekannten Assoziationen hingefügt. 
\item[removeAssociation] Diese Operation erwartet beim Aufruf nur die zu entfernende \textbf{Assoziation}. 
Bei erfolgreicher Durchführung dieser Operation, wird die
angegebene Assoziation aus dem System entfernt.
Sollte es ein oder mehrere Exemplare (Links) zu dieser Assoziation geben oder sich die zu löschende Assoziation in mindestens einer Hierarchie befinden, wird eine \term{ConsistencyException} erzeugt und die Assoziation wird nicht entfernt. 
\item[createHierarchy] Da es in dem erarbeiteten Modell keine leeren Hierarchien geben darf, wird neben dem \textbf{Namen} der zu erstellenden Hierarchie auch eine 
\textbf{Assoziation} erwartet. Nach erfolgreicher Durchführung, wird die Hierarchie erstellt und die angegebene Assoziation der neuen Hierarchie zugeordnet.
Insbesondere folgende Gründe führen zum Misserfolg:
\begin{itemize}
\item Es exisitiert bereits eine Hierarchie mit diesem Namen. Das hat zur Folge, dass eine \term{DoubleDefinitionException} erzeugt wird.
\item Es existiert auf der Exemplarebene ein Zyklus. Wie in der Beschreibung zu addAssociation erläutert, ist dieser Umstand für Hierarchien untersagt.
 Es resultiert eine \term{CycleException.}
 \item Durch den internen Aufruf von addAssociation, kann es bei dieser Operation außerdem zu einer \term{ConsistencyException} kommen.
\end{itemize}
\item[removeHierarchy] Nach erfolgreicher Ausführung dieser Operation, wird die übergebene \textbf{Hierarchie} aus dem System entfernt. 
Es wird lediglich die Hierarchie entfernt. Die Assoziationen, welche an der gelöschten Hierarchie teilnehmen, bleiben weiterhin bestehen. Durch den Template-Mechanismus von ModelItem kann diese Operation außerdem eine \term{ConsistencyException} erzeugen. Aufgrund der Tatsache, dass Hierarchien in der aktuellen Implementierungsstufe keine abhängigen Objekte haben, kann es nie zu dieser Exception kommen.
\item[addAssociation] Diese Operation erwartet zum einen die \textbf{Hierarchie} zu der die Assoziation hinzugefügt werden soll und zum anderen die \textbf{Assoziation} selbst.
Wenn es auf der Exemplarebene keine zyklischen Links zu dieser Assoziation gibt, wird die Assoziation der Hierarchie zugeordnet. Sollte es zyklische Links
zu dieser Assoziation geben oder die Assoziation ist bereits in dieser Hierarchie, werden entsprechende Exceptions (\term{CycleException,} bzw. \term{ConsistencyException}) 
erstellt und die Assoziation wird der Hierarchie nicht zugeordnet.
\item[removeAssoFrmHier] Zum Entfernen von Assoziationen aus Hierarchien, ist es notwendig dieser Operation sowohl die \textbf{Hierarchie}, als auch die zu entfernende 
\textbf{Assoziation} anzugeben. Dabei wird lediglich die Assoziation aus der Hierarchie entfernt, die Assoziation an sich bleibt bestehen. 
Sofern die angegebene Assoziation an dieser Hierarchie teilnimmt, wird diese Verbindung durch die Ausführung dieser Operation entfernt. 
Sollte dies nicht der Fall sein, wird eine \term{NotAvailableException} geworfen. Einen Sonderfall bildet die letzte Assoziation einer
Hierarchie. Diese darf gemäß des Modells nicht entfernt werden, da leere Hierarchien nicht zulässig sind. Beim Versuch die letzte
Assoziation einer Hierarchie aus der Hierarchie zu entfernen, wird eine \term{ConsistencyException} erzeugt und der Versuch wird abgebrochen. 
\end{description}
