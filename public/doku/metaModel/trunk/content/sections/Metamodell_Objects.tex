\subsection{Objektwelt}
Im Folgenden werden die Anpassungen der Klasse \term{Object} und deren Methoden im Vergleich zum Modell aus der Vorlesung betrachtet.
\subsubsection{�nderungen und neue Methoden}
Die Klasse \term{Object} wurde wie viele Klassen im Modell um den Pr�fix \qq{M} erweitert um Verwechselungen mit der Klasse \term{java.lang.Object} zu vermeiden.
Desweiteren wurde der Klasse die Methode \term{getProductType()} hinzugef�gt, die ein MTypeConjunction liefert.
Diese TypeConjunction repr�sentiert die Produkttypen des Objekts, der implizit durch die Assoziation \term{types} dargestellt wird.

\subsubsection{L�schen}
Das L�schen von \term{MObject}-Exemplaren funktioniert wie auch sonst im System �ber die Klasse \term{MModelItem}.
Die Abh�ngigkeiten eines \term{MObject}s sind hierbei 
\begin{itemize}
\item \term{Links} die entweder mit \term{source} oder \term{target} auf das Objekt zeigen,
\item \term{Measurement}s und \term{Account}s
\item \term{Observation}s
\item und \term{NameInstance}s
\end{itemize}
Das L�schen eines Objekts ist zur Zeit nur m�glich, wenn keine Exemplare dieser Klassen im System vorhanden sind, di� von dem Objekt abh�ngen.

\subsubsection{Singletons und \qq{normale} Objekte}
F�r die Unterscheidung von Exemplaren von Singleton-Typen und \qq{normalen} Objekte, wurden eine Vererbungshierarchie eingef�hrt, welche Objekte in \term{MObject} (\qq{normale} Objekte)
und \term{MSingletonObject} (Exemplare von Singleton Typen) partitioniert. Die dazu eingef�hrte Oberklasse \term{AbstractObject} abstrahiert die Gemeinsamkeiten der beiden Klassen. 
Durch die Partitionierung ist es m�glich die unterschiedlichen geltenden Konsistenzbedingungen in den jeweiligen Klassen zu implementieren.
Dazu wurde die Types Assoziation in den konkreten Klassen hinterlegt, wobei SingletonExemplare nur einen Typen haben k�nnen.
Entsprechend sind auch die Operationen zum hinzuf�gen, �ndern und entfernen von Typen nur in der Klasse \term{MObject} zu finden.

Kritik:
Diese Designentscheidung ist kritisch zu betrachten. Dadurch, dass Objekte in der sp�teren Anwendung immer Exeplare entweder von MObject oder MSingeltonObject sind,
ist ein �ndern des Attributs \qq{singleton} an Exemplaren des Typs MAtomicTypes kritisch. Dies f�hrt dazu, dass die Exemplare der Objekte weggeschmissen werden m�ssen.
Dies ist ins besondere bei Refactorings, die Singletons entfernen (wie z.B. beim Implemetieren von Mandantenf�higkeit) problematisch.
Es ist ratsam in eine sp�teren Version dieses Problem etwa mit einem \qq{State}-Pattern zu vermeiden.