\subsection{Objektwelt}
Im Folgenden werden die Anpassungen der Klasse \term{Object} und deren Methoden im Vergleich zum Modell aus der Vorlesung betrachtet.

\img[width=\relWidth{0.5}]{type/Object.pdf}{Ausschnitt Objektwelt}{img_obj}

\subsubsection{Änderungen und neue Methoden}
Die Klasse \term{Object} wurde wie viele Klassen im Goja-Modell um den Präfix \qq{M} erweitert um Verwechselungen mit der Klasse \term{java.lang.Object} zu vermeiden.
Desweiteren wurde der Klasse die Methode \term{getProductType()} hinzugefügt, die ein TypeConjunction liefert.
Diese TypeConjunction repräsentiert die Produkttypen des Objekts, der implizit durch die Assoziation \term{types} dargestellt wird.

\subsubsection{Löschen}
Das Löschen von \term{Object}-Exemplaren funktioniert wie auch sonst im System über die Klasse \term{MModelItem}.
Die Abhängigkeiten eines \term{Object}s sind hierbei 
\begin{itemize}
\item \term{Links} die entweder mit \term{source} oder \term{target} auf das Objekt zeigen,
\item \term{Measurement}s und \term{Account}s
\item \term{Observation}s
\item und \term{NameInstance}s
\end{itemize}
Das Löschen eines Objekts ist zur Zeit nur möglich, wenn keine Exemplare dieser Klassen im System vorhanden sind, die von dem Objekt abhängen.

\subsubsection{Singletons und \qq{normale} Objekte}
Für die Unterscheidung von Exemplaren von Singleton-Typen und \qq{normalen} Objekte, wurden eine Vererbungshierarchie eingeführt, welche Objekte in \term{Object} (\qq{normale} Objekte)
und \term{SingletonObject} (Exemplare von Singleton Typen) partitioniert. Die dazu eingeführte Oberklasse \term{AbstractObject} abstrahiert die Gemeinsamkeiten der beiden Klassen. 
Durch die Partitionierung ist es möglich die unterschiedlichen geltenden Konsistenzbedingungen in den jeweiligen Klassen zu implementieren.
Dazu wurde die Types Assoziation in den konkreten Klassen hinterlegt, wobei SingletonExemplare nur einen Typen haben können.
Entsprechend sind auch die Operationen zum Hinzufügen, Ändern und Entfernen von Typen nur in der Klasse \term{Object} zu finden.

Kritik:
Diese Designentscheidung ist kritisch zu betrachten. Dadurch, dass Objekte in der späteren Anwendung immer Exeplare entweder von MObject oder MSingeltonObject sind,
ist ein Ändern des Attributs \qq{singleton} an Exemplaren des Typs MAtomicTypes kritisch. Dies führt dazu, dass die Exemplare der Objekte weggeschmissen werden müssen.
Dies ist ins besondere bei Refactorings, die Singletons entfernen (wie z.B. beim Implemetieren von Mandantenfähigkeit) problematisch.
Es ist ratsam in einer späteren Version dieses Problem etwa mit einem \qq{State}-Pattern zu vermeiden.