\subsection{Type}
Das Typsystem ist die zentrale Komponente des Metamodells. Im folgenden Abschnitt sind die 
Änderungen bezüglich des in der Vorlesung erarbeiteten Modells kurz zusammengefasst.

\subsubsection{Aspekte und atomare Typen}

Aspekte stellen gemäß der ursprünglichen Modellierung eine Dimension dar, in deren Elementen sich Objekte dynamisch klassifizieren lassen. 
Atomare Typen der Klasse \emph{AtomicType} werden genau einem Aspekt zugeordnet. Durch die \emph{lessThan}-Assoziation lässt sich aspektintern eine 
partielle Ordnung \emph{isLessOrEqualThan} ableiten.  

\img[width=\relWidth{0.75}]{type/Aspekte.pdf}{Umarrangierter Ausschnitt Aspekte}{img_aspekts}

Das Interface \emph{AnythingORAtomicType} ist als Zieltyp der \emph{lessThan}-Assoziation ergänzt worden, 
um die Möglichkeit zu schaffen einen atomaren Typen mit Obertyp so zu ändern, dass er anschließend keinen 
Obertypen mehr besitzt. Folgende Möglichkeiten wurden in der Entwurfsphase betrachtet: 
\begin{enumerate}
  		\item superType Assoziation mit Multiplizität $0..1$.
		\item State-Pattern für atomare Typen \(state \in \{rootType, subType\}\).
        \item Automatisch mitgenerierte Obertypen für Aspekte.
        \item Interface über Anything und AtomicType.
\end{enumerate}

Die erste Variante lässt sich in GOJA aufgrund der restriktiven Set-Methoden nicht realisieren, da diese keine NULL-Werte akzeptieren.
Ein State-Pattern ist die sauberste Implementierung, ist aber wegen fehlenden Mehrwertes ausgeschieden. 
Variante drei hat den Nachteil, dass die mitgenerierten Typen die Semantik \qq{unkategorisiert} im zugehörigen Aspekt besitzen.
Damit wird pro Aspekt ein Platzhalter für NULL generiert und sorgt für zusätzliche Komplexität ohne Mehrwert. 
Da ohne State-Pattern aber eine Repräsentation für keinen atomaren Obertypen benötigt wird, wird an dieser Stelle \emph{Anything} verwendet 
da es ohnehin Obertyp eines jeden Typen ist. Das ist auch der wichtigste Unterschied zwischen \emph{Anything} und einem atomaren Obertyp; Anything ist Teil jedes Aspekts, 
während atomare Typen genau einem Aspekt zugeordnet sind. 


\subsubsubsection{Aspekte}

\textbf{Zusätzliche Konsistenzbedinungen} \newline
Die einzige Konsistenzbedingung von Aspekten ist, dass sie paarweise verschiedene Namen tragen müssen. 

\textbf{Manager} \newline
Aspekte werden vom \emph{AspectManager} verwaltet. Alle Operation werden transaktional aufgerufen und prüfen die Konsistenzbedingung. 

\begin{description}
\item[createAspect] legt einen neuen Aspekt an.
\item[renameAspect] benennt einen Aspekt um.
\item[deleteAspect] löscht einen Aspekt, wenn keine abhängigen Modellelemente existieren.
\end{description}

\textbf{Abhängige Elemente}
\begin{itemize}
	\item Atomare Typen, die in diesem Aspekt angelegt sind.
\end{itemize}

\newpage
\subsubsubsection{Atomare Typen}

\textbf{Zusätzliche Konsistenzbedinungen}
\begin{enumerate}
  		\item Paarweise verschiedene Namen in einem Aspekt
  		\item \(\forall a \in AT : \lnot(a.singleton \land a.abstract) \)
\end{enumerate}

\textbf{Manager} \newline
Atomare Typen werden vom \emph{TypeManager} verwaltet. 

\begin{description}
\item[createAtomicRootType] legt neuen atomaren Typen mit Obertyp \emph{Anything} an. 
\item[createAtomicSubType] legt neuen atomaren Typen unter einem andern an. 
\item[renameAtomicType] benennt einen atomaren Typen um. 
\item[changeAbstract] ist eine modifizierte Set-Methode. Kann einen konkreten Typen nur abstrakt umdeklarieren, wenn er weder Singletons ist, noch 
Objekte in ihm klassifiziert sind.
\item[changeSingleton] ist eine modifizierte Set-Methode. Kann die Singleton-Eigenschaft nur wiederrufen, wenn das \emph{SingletonObject} gelöscht werden kann. 
 Kann die Singleton-Eigenschaft nur gewähren, wenn keine Objekte existieren und der Typ nicht abstrakt ist\footnote{Im Folgenden wird auf die Semantik von Singletons näher eingegangen.}. 
 \item[delteAtomicType] löscht einen atomaren Typen, falls keine Abhängigkeiten bestehen.
\end{description}

\textbf{Singletons} \newline
Die Singleton-Eigenschaft eines atomaren Typen A bedeutet ausschließlich, dass genau ein Objekt existiert, dessen speziellster Typ A ist.
Von Singletons darf beliebig abgeleitet werden. 
In der \emph{initializeOnCreation}-Operation des atomaren Typen wird, falls er über die Singletoneigenschaft verfügt, 
sein \emph{SingletonObject} erstellt. In der \emph{prepareForDeletion}-Operation wird - falls vorhanden - das \emph{SingeletonObject}
mit gelöscht. Die \emph{changeSingleton}-Operation verwaltet ebenfalls \emph{SingletonObject}s. In den entsprechenden Operationen im 
\term{TypeManager} werden \term{SingeltonObject}s beim \term{ObjectManager} registriert bzw. entfernt.
Der Zeitpunkt des Erzeugens von term{SingeltonObject}s ist so früh gewählt um eine 1:1 Beziehnung zwischen Typ und \emph{SingeletonObject} zu garantieren.
Damit wird zum Ausdruck gebracht, dass Typ und \emph{SingeletonObject} ein Objekt sind. Diese Interpretation führt zu einigen Einschränkungen beim Ändern des Singleton-Attributs, 
sodass es für eine spätere Ausbaustufe geplant ist, die Semantik der Singleton-Eigenschaft zu \emph{höchstens einem Objekt zum Typen} zu ändern. Dies ermöglicht dann 
auch einen nicht-Singleton Typen mit einem Objekt nachträglich als Singleton zu klassifizieren ohne das Objekt vorher löschen zu müssen.  
 
\newpage
\textbf{Abhängige Elemente}
\begin{enumerate}
  		\item Direkte atomare Untertypen.
  		\item Bei Singletons: Abhängige Objekte des \emph{SingletonObject}s.
  		\item Siehe Typabstraktion.
\end{enumerate}

\subsubsection{Komplexe Typen}

Komplexe Typen bieten die Möglichkeit Und- und Oder-Typen zu bilden. Zyklenfreiheit wird hierbei dadurch gewährleistet, 
dass \emph{containedTypes} und alle Spezialisierungen Teil einer Hierarchie sind. Komplexe Typen sind immutable,
lassen sich aber (bis auf \emph{Anything} und \emph{Nothing}) löschen, falls keine Abhängigkeiten existieren.

\img[width=\relWidth{1}]{type/ComplexTypes.pdf}{Ergänzter Ausschnitt: Komplexe Typen}{img_complexTypes}

Um für Unterklassen von \emph{ComplexType} die Assoziation \emph{containedTypes} weiter einschränken zu 
können, ist sie spezialisiert worden und ist nun als \emph{fetch}-Operation im Modell vorhanden. 
Ziel ist, z.B. in den leeren Typen keine Änderungen an der Assoziation zu erlauben und diese für die Bereitstellung 
der disjunktiven Normalform mit weiteren Constraints zu versehen (siehe z.B. \emph{NonEmptyAtomicTypeConjunction}) 

Bei Komplexen Typen wird zwischen struktureller und semantischer Äquivalenz unterschieden. Zwei Typen sind strukturell 
äquivalent, wenn sie identisch aufgebaut sind. Zwei Typen A,B sind semantisch äquivalent, wenn nach den in der Vorlesung 
aufgestellten Regeln für \term{isLessOrEqual} gilt: 
\begin{equation}A.isLessOrEqual(B) \land B.isLessOrEqual(A)
\end{equation}

Die strukturelle Äquivalenzrelation ist Teilmenge der semantische Äquivalenzrelation.
So sind die beiden Typkonjunktionen A**B und B**A zwar semantischt äquivalent, aber nicht strukturell.

\textbf{Zusätzliche Konsistenzbedinungen} \newline
\begin{equation}\forall t \in NEATConjunction : 
	\forall ct \in t.containedTypes : ct.instanceOf AT
\end{equation}
\begin{equation}\forall t \in NEATConjunction : 
	t.containedTypes.length > 0
\end{equation}
\begin{equation}
\begin{split}
	\forall t \in AbsTConjunction :\\ 
	\forall a,b \in t.containedTypes : a.Aspects \cap b.Aspects = \emptyset
\end{split}
\end{equation}
\begin{equation}
\begin{split}
	\forall t \in NEDisjunctiveNF :\\ 
	\forall ct \in t.containedTypes : ct.instanceOf ATConjunction
\end{split}
\end{equation}
\begin{equation}\forall t \in NEDisjunctiveNF : 
	t.containedTypes.length > 0
\end{equation}


\textbf{Manager} \newline
Komplexe Typen werden vom \term{TypeManager} verwaltet. Es wird sichergestellt, dass 
die angelegten komplexen Typen paarweise nicht strukturell äquivalent sind. 

\begin{description}
\item[createTypeConjunction] legt normalisierten\footnote{Im Folgenden Abschnitt wird die Typnormalisierung ausführlich beschrieben.} \qq{UND-Typen} an, falls noch kein strukturell äquivalenter Typ verwaltet wird. Andernfalls wird dieser zurückgegeben.
\item[createTypeDisjunction] legt normalisierten \qq{ODER-Typen} an, falls noch kein strukturell äquivalenter Typ verwaltet wird. Andernfalls wird dieser zurückgegeben.
\item[deleteComplexType] löscht einen komplexen Typen, falls keine Abhängigkeiten bestehen.
\end{description}


\textbf{Typnormalisierung} \newline
In den \term{create}-Operationen werden die Typen normalisiert. 
\begin{enumerate}
\item \textbf{Überflüssige Verschachtelungen auflösen}  \newline 
	Wird eine Konjunktion erstellt, so werden alle als Argument übergebenen Konjunktionen (rekursiv) entpackt und deren Bestandteile direkt verwendet.
	Bei Disjunktionen wird analog verfahren. 
\item \textbf{Überflüssige Elemente entfernen}  \newline 
	Die in Schritt 1 erstellte Menge an Argumenten wird nun paarweise verglichen. Stehen zwei Argumente bzgl. \term{lessOrEqual} in Relation, so wird 
	bei Konjunktionen ausschließlich der speziellere und bei Disjunktionen ausschließlich der allgemeinere Typ übernommen. Neutrale Elemente werden 
	entfernt. 
\end{enumerate} 

\textbf{Persistenz} \newline
Komplexe Typen werden grundsätzlich \term{delayedPersistent} erstellt und werden erst, wenn sie vom \term{TypeManager} verwaltet werden, persistiert. 
Typen, die als Rückgabewerte für Operationen auf \term{ModelItem}s (z.B. \term{fetchProductType} auf \term{AbstractObject}) erzeugt werden, 
werden in der Regel nicht persistiert. 

Auf \textbf{abhängige Elemente} wird im folgenden Abschnitt eingegangen.

\subsubsection{Typabstraktion}
Die Klasse \term{Type} abstrahiert von atomaren und komplexen Typen. Um die inverse Richtung der spezialisierten Assoziation 
\term{containedTypes} nutzen zu können, wurde eine Operation \term{fetchTypesContainingMe} implementiert. 

\textbf{Abhängige Elemente}
\begin{enumerate}
  		\item Komplexe Typen, die diesen Typen direkt beinhalten.
  		\item ObservationTypes, die Observationen für Objekte dieses Typs typisieren.
  		\item Formalparamter, die in diesem Typ typisiert sind.
  		\item Operationen, die diesen Typ als Quelle oder als Ziel besitzen.
  		\item QuantifiedObjectTypes, die auf diesen Typen verweisen. 
\end{enumerate}
