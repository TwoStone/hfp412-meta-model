\subsection{Typesystem}

\subsubsection{Aspekte und atomare Typen}

Aspekte stellen gemäß ursprünglichen Modellierung eine Dimension dar, in deren Elementen sich Objekte dynamisch klassifizieren lassen. 
Atomare Typen der Klasse \emph{AtomicType} werden genau einem Aspekt zugeordnet. Durch die \emph{lessThan}-Assoziation lässt sich aspektintern eine 
partielle Ordnung \emph{isLessOrEqualThan} ableiten.  

\img[width=\relWidth{0.5}]{type/Aspekte.png}{Umarrangierter Ausschnitt Aspekte}{img_aspekts}

Das Interface \emph{AnythingORMATomicType} ist als Zieltyp der \emph{lessThan}-Assoziation ergänzt worden, 
um die Möglichkeit zu schaffen einen atomaren Typen mit Obertyp so zu ändern, dass er anschließend keinen 
Obertypen mehr besitzt. Folgende Möglichkeiten wurden in der Entwurfsphase betrachtet: 
\begin{enumerate}
  		\item superType Assoziation mit Multiplizität 0..1
		\item State-Pattern für atomare Typen \(state \in \{rootType, subType\}\)
        \item automatisch mitgenerierte Obertypen für Aspekte
        \item Interface über Anything und AtomicType
\end{enumerate}

Die erste Variante lässt sich in GOJA aufgrund der restriktiven Set-Methoden nicht realisieren, da diese keine NULL-Werte akzeptieren.
Ein State-Pattern wäre die sauberste Implementierung gewesen, ist aber wegen fehlenden Mehrwertes ausgeschieden. 
Variante drei hat den Nachteil, dass die mitgenerierten Typen die Semantik \qq{unkategorisiert} im zugehörigen Aspekt besitzen.
Damit wird pro Aspekt ein Platzhalter für Null generiert und sorgt für zusätzliche Komplexität ohne Mehrwert. 
Da ohne State-Pattern aber eine Repräsentation für keinen atomaren Obertypen benötigt wird, wird an dieser Stelle \emph{Anything} verwendet 
da es ohnehin Obertyp eines jeden Typen ist. Dies ist insofern unsauber, als das \emph{Anything} nicht Teil eines einzigen Aspektes ist.
Weitere Nachteile sind bisher nicht ersichtlich.



\TODO[Kapitel schreiben]