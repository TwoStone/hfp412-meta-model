% Kurzanleitung (weiteres siehe Commands-Datei)

% Ein Glossareintrag wird folgendermaßen hinzugefügt:
%	\addGl{Label}{Name}{Name im Plural}{Beschreibung}
% Das Label sollte dabei nur [A-Za-z0-9] enthalten
% und wird nur zum Referenzieren benutzt (nie im Dokument).
% Es muss natürlich global eindeutig sein.

% Ein Glossareintrag wird im Text folgendermaßen referenziert:
%	\gl{Label}	Gibt den Namen des Glossareintrags aus
%				und erzeugt einen Link zum entsprechenden Eintrag.
%	\glp{Label}	Gibt den Namen des Glossareintrags in der Pluralform aus
%				und erzeugt einen Link zum entsprechenden Eintrag.

% Möchte man nicht den normalen Namen ausgeben,
% gibt es zudem die folgende Möglichkeit:
%	\gla{Label}{Alternative Beschriftung}

\addGl{Acronym}{Acronym}{Acronyme}{%
	Acronyms behave a bit differently than normal glossary terms. On first use the ac command will
	display \qq{<full> (<abbrv>)}. On subsequent uses only the abbreviation will be displayed.
}
\addGl{Webservice}{Webservice}{Webservices}{%
	Das World Wide Web Consortium definiert die Bereitstellung eines Webservices als Unterstützung zur
	Zusammenarbeit zwischen verschiedenen Anwendungsprogrammen, die auf unterschiedlichen Plattformen
	und/oder Frameworks betrieben werden.
	Ein Webservice oder Webdienst ist eine Software-Anwendung, die mit einem Uniform Resource Identifier
	(URI) eindeutig identifizierbar ist und deren Schnittstelle als XML-Artefakt definiert, beschrieben
	und gefunden werden kann. Ein Webservice unterstützt die direkte Interaktion mit anderen
	Software-Agenten unter Verwendung XML-basierter Nachrichten durch den Austausch über
	internetbasierte Protokolle.
}
\addGl{Xena}{Xena}{Xenaesi}{%
	Dies ist ein Artikel über Xena, die fast so aussieht wie Brigitte. Brigitte hat lange blonde Haare,
	trägt ne Brille und nur Schuhe.
}











