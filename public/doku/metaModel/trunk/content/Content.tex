\begin{document} % Ab hier beginnt der wirkliche Text, alles nachfolgende erscheint im Dokument
\resetPageTitle % Seitentitel initial setzen

% Titelseite
\pagenumbering{Roman} % Nummerierung der Seiten in römischen Zahlen
\thispagestyle{empty} % Kopf- und Fußzeile ausblenden
\begin{center}

~\vfill

\titlepageBox{
	\imgUnlabeled[width=\relWidth{0.35}]{logo-fhdw.pdf}
}

\titlepageBoxHeadline{Fachhochschule für die Wirtschaft Hannover}{-- FHDW --}

\titlepageBoxHeadline{Mitschrift der Vorlesung}{\textbf{Quantitative Forschungsmethoden}}
\titlepageBoxHeadline{Thema}{\textbf{Protokoll vom 18.05.2013}}


\titlepageBoxHeadline{Verfasser:}{
	Niels Alexander Bellhäuser
}

\titlepageBoxHeadline{2. Theoriequartal}{%
	Studiengang Master of Science (M.Sc.)\\%
	Business Process Engineering%
}

\titlepageBoxHeadline{Eingereicht am:}{\today}

\end{center} % Inhalt einbinden

% Abstract
%\newpage
%\setPageTitle{Abstract} % Seitentitel ändern, da diese Section ausgeblendet ist
%\include{\content/pages/Abstract} % Inhalt einbinden

% Inhaltsverzeichnis
\newpage
\resetPageTitle % Wichtig: Seitentitel wieder zurücksetzen, da er auf der letzten Seite geändert wurde
\maintoc % Inhaltsverzeichnis einfügen

% Beginn des Inhalts
\newpage
\pagenumbering{arabic} % Nummerierung der Seiten in arabischen Zahlen
\setcounter{page}{1} % Seitenzahlen zurück auf Anfang setzen
\glsresetall % Alle verwendeten Abkürzungen zurücksetzen

% Kapitel
\section{Einleitung}

Im Rahmen des Master Studiums (M.Sc. Studiengang: Business Process Engineering) an der FHDW Hannover
war es für die Studiengruppe \qq{HFP412} die Aufgabe, Inhalte und Hintergründe der Veranstaltung \qq{Quantitative Forschungsmethoden}\footnote{Dozentin: Frau Dr. Sylvie Gasnier} zu dokumentieren.

\img[width=\relWidth{0.9}]{skalenniveaus.pdf}{Verschiedene Zahlen- oder Skalenniveaus}{img_skalenniveaus}

Die vorliegende Ausarbeitung konzentriert sich auf den Zweig der qualitativen Skalen. Genauer das Thema der \qq{Nominal- und Ordinalskalen} (siehe \refImg{img_skalenniveaus}). 
Die theoretischen Hintergründen werden anhand von praktischen Beispielen mit Hilfe von Microsoft Excel untermauert. 
Als Datenquelle dient hierfür der Münchner Mietspiegel aus dem Jahre 2003\footnote{\url{http://data.ub.uni-muenchen.de/2/1/miete03.asc}}.

Um die statistische Auswertung zu erleichtern, müssen einige Vorarbeiten geleistet werden. Zunächst müssen pro Merkmalsart die verschiedenen Ausprägungen 
analysiert werden. Hierbei entstehen ein Codeplan und eine Datenmatrix, welche als Grundlage für die folgenden Auswertungen dienen. 
Auf die konkreten Ausprägungsarten wird zu einem späteren Zeitpunkt eingegangen. Um an die
Ausprägungsarten herzuführen, werden in Abschnitt \ref{sec:defNot} zunächst Codeplan und Datenmatrix
skizziert.
 % Inhalt einbinden

% Kapitel
%\section{Grundlagen}

\blindtext

\subsection{Idee des Metamodells}

\blindtext

\subsection{Theoretischer Ansatz von Martin Fowler}

\TODO[Bezug zum Buch herstellen und als Auslöser darstellen] \refLibLiteratur{fowler1997analysis}

\TODO[Beschreiben, welche Teile umgesetzt wurden, und dann auf die Folgekapitel verweisen]
 % Inhalt einbinden

% Kapitel
\section{Metamodell} 

% Kapitel
\subsection{Types}

\TODO[Kapitel schreiben] % Inhalt einbinden

% Kapitel
\subsection{Operations und Associations}

\subsubsection{Operationen}

Die Klasse Operation repräsentiert die Abbildung von Operationen in dem erarbeiteten Modell.
Wie in dem Modell aus Abbildung XXX ersichtlich, hat die Oberklasse (AbstractOperation) zwei direkte Assoziationen zu Type. 
Im Kontext von Operationenen ist die Quelle (source) der Typ, welcher die Operation enthält. 
Das Ziel (target) einer Operation entspricht dem Rückgabetypen.
Neben Quell und Zieltypen enthält eine Operation eine Menge von Formalparametern welche 
wiederum über die Klasse Type typisiert sind.

Die Klasse Operation enthält nur genau zwei Operation.
Die Operation \textbf{isStatic()} gibt genau dann true zurück, wenn der Quelltyp dieser Operation der leeren Summe entspricht.
Zusätzlich zu der Bedingung aus isStatic gilt für die zweite Operation \textbf{isConstant()}, dass die Operation keine Parameter enthalten darf. 

\subsubsubsection{OperationManager} \newline
Der OperatioManager bietet 10 transaktionale Operationen und beinhaltet zur Verwaltung vier Listen wobei zwei davon nur abgeleitete Informationen
enthalten.

\begin{description}
\item[Operation** operations] Diese Liste enthält alle Operationen die der Manager verwaltet.
\item[derived Operation** staticOperations] Diese Liste enthält ausschließlich statische Operationen: $\forall \, o  \in staticOperations : o.isStatic()==true$ 
\item[derived Operation** constants] Diese Liste enthält ausschließlich Konstanten. Demnach gilt: $\forall \, o  \in constants : o.isConstant()==true$
\item[FormalParameter** formalParameters] Hier werden alle Formalparameter abgelegt die der Manager verwaltet.
\end{description}


TODO Operationen beschreiben!

\subsubsection{Assoziationen}

Die Klasse Association repräsentiert die Möglichkeit Assoziationen zwischen zwei beliebigen Typen abzubilden.
Neben dem Quell- und Zieltypen, kann eine Assoziation an beliebig vielen Hierarchien teilnehmen.
Eine Assoziation entspricht der Interpretation einer Observation, wenn alle Zieltypen entweder singleton sind oder abstrakt und als Unterklassen nur singletons enthalten.
Wie bei vielen anderen Modellbestandteilen, sind auch die Namen der Assoziation indiziert. Daraus folgt auch, dass es keine gleichnamigen 
Assoziationen geben darf.

Wie in der Abbildung XXX zu sehen ist, können sowohl Operationen als auch Assoziationen Formalparameter beinhalten. Die Möglichkeit der Zuweisung von
Formalparametern an Assoziationen, wird dem Anwender über die Oberfläche nicht bereitgestellt. 
Eine Mögliche zur Interpretation von Formalparametern für Assoziationen sind die n-stelligen Assoziationen
welche in der ersten Implementierungsstufe nicht umgesetzt wurden.

\subsubsubsection{AssociationManager} \newline
Der AssoziationManager bietet sechs transaktionale Operationen an und beinhaltet zwei Listen, welche er verwaltet.
Die eine Liste beinhaltet alle erstellten Assoziationen und die andere alle Hierarchien. Diese beiden Listen werden dem Anwender an der 
Oberfläche präsentiert und dienen der Verwaltung der darin enthaltenen Objekte.

Im Folgenden werden die Operationen des Managers kurz beschrieben, wobei sowohl \textbf{Parameter} als auch \emph{Exceptions} durch entsprechende Formatierungen hervorgehoben wurden.

\begin{description}
\item[createAssociation] Bei Aufruf dieser Operation müssen \textbf{Quell- und Zieltyp} sowie der \textbf{Name} der zu erstellenden Assoziation 
angegeben werden.
Sollte der Name schon von einer anderen Assoziation verwendet werden, wirft die Operation eine \emph{DoubleDefinitionException} und die Assoziation wird nicht
erstellt. Ein weiterer Grund für das nicht Erstellen einer Assoziation ist, wenn Quell- oder Zieltyp der leeren Summe entsprechen. In einem solchen Fall
wird eine \emph{ConsistencyException} erstellt.
Wenn keine der genannten Exceptions aufgetreten ist, wird die Assoziation erstellt und zur Liste der bekannten Assoziationen hingefügt. 
\item[removeAssociation] Diese Operation erwartet beim Aufruf nur die zu entfernende \textbf{Assoziation}. 
Bei erfolgreicher Durchführung dieser Operation, wird die
angegebene Assoziation aus dem System entfernt.
Sollte es ein oder mehrere Exemplare (Links) zu dieser Assoziation geben oder sich die zu löschende Assoziation in mindestens einer Hierarchie befinden, wird eine \emph{ConsistencyException} erzeugt und die Assoziation wird nicht entfernt. 
\item[createHierarchy] Da es in dem erarbeiteten Modell keine leeren Hierarchien geben darf, wird neben dem Namen der zu erstellenden \textbf{Hierarchie} auch eine 
\textbf{Assoziation} erwartet. Nach erfolgreicher Durchführung, wird die Hierarchie erstellt und die angegebene Assoziation der neuen Hierarchie zugeordnet.
Insbesondere folgende Gründe führen zum Misserfolg:
\begin{itemize}
\item Es exisitiert bereits eine Hierarchie mit diesem Namen. Das hat zur Folge, dass eine \emph{DoubleDefinitionException} erzeugt wird.
\item Es existiert auf der Exemplarebene ein Zyklus. Wie in der Beschreibung zu addAssociation erläutert, ist dieser Umstand ist für Hierarchien untersagt.
 Es resultiert eine \emph{CycleException.}
\end{itemize}
\item[removeHierarchy] Nach erfolgreicher Ausführung dieser Operation, wird die übergebene \textbf{Hierarchie} aus dem System entfernt. 
Es wird lediglich die Hierarchie entfernt, die beinhaltenden Assoziationen an sich bleiben weiterhin bestehen.
\item[addAssociation] Diese Operation erwartet zum einen die \textbf{Hierarchie} zu der die Assoziation hinzugefügt werden soll und zum Anderen die \textbf{Assoziation} selbst.
Wenn es auf der Exemplarbene keine zyklischen Links zu dieser Assoziation gibt, wird die Assoziation der Hierarchie zugeordnet. Sollte es zyklische Links
zu dieser Assoziation geben oder die Assoziation ist bereits in dieser Hierarchie, werden entsprechende Exceptions (\emph{CycleException,} bzw. \emph{DoubleDefinitionException}) 
erstellt und die Assoziation wird der Hierarchie nicht zugeordnet. UMSCHREIBEN(wirklich)
\item[removeAssoFrmHier] Zum Entfernen von Assoziationen aus Hierarchien, ist es notwendig dieser Operation sowohl die \textbf{Hierarchie}, als auch die zu entfernende 
\textbf{Assoziation} anzugeben. Dabei wird lediglich die Assoziation aus der Hierarchie entfernt, die Assoziation an sich bleibt bestehen. 
Sofern die angegebene Assoziation an dieser Hierarchie teilnimmt, wird diese Verbindung durch die Ausführung dieser Operation entfernt. 
Sollte dies nicht der Fall sein, wird eine \emph{NotAvailableException} geworfen. Ein Sonderfall bildet sie letzte Assoziation einer
Hierarchie. Diese darf gemäß des Modells nicht entfernt werden, da leere Hierarchien nicht zuläßig sind. Beim Versucht die letzte
Assoziation einer Hierarchie aus der Hierarchie zu entfernen, wird eine \emph{ConsistencyException} erzeugt und der Versucht wird abgebrochen. 
\end{description}



\subsection{Messages und Links}

Messages und Links haben eine gemeinsame Oberklasse MessageOrLink. In der
ersten Ausbaustufe werden lediglich die Links implementiert. Zu den Messages existiert eine rudimentäre
Implementierung die dem Anwender aber über die Oberfläche nicht bereitgestellt wird, da die 
Semantik eines Messageexemplars bislang nicht eindeutig geklärt ist. \vspace{15pt}

Ein Link ist eine konkrete Ausprägungen einer Assoziation. 
Folglich ist die Erstellung eines Links nur möglich, wenn eine Assoziation als Typ gewählt wurde und der zu erstellende Link
sich an die  daraus ergebenden Konsistenzbedingungen hält.
Für Links stehen dem Anwender nur zwei Operationen zur Verfügung.

\subsubsection{Einen Link erstellen - createLink}
Wie zuvor erwähnt, ist es für einen Link zwingend notwendig sich an die Konsistenzbedingungen der Assoziation zu halten.
Bei der Erstellung eines Links wird eine \emph{ConsistencyException} erzeugt, wenn die Quell- und/oder Zielobjekte gemäß der Typebene nicht 
gestattet sind.

\begin{enumerate}
\item $\forall \, l \in Link: l.source.type.isLessOrEqual(l.type.source)$
\item $\forall \, l \in Link: l.target.type.isLessOrEqual(l.type.target)$
\end{enumerate}

Desweiteren ist es Assoziationen möglich an mehreren Hierarchien teilzunehmen. Zyklische Assoziationen sind auf der Metaebene möglich, 
da sich sonst keinerlei Aggregationsbeziehung abbilden ließe.
Auf der Ebene der Exemplare hingegen, müssen \emph{alle Links, die in Assoziation typisiert sind, welche an einer oder mehreren Hierarchien teilnehmen, zyklenfrei sein.}
Wenn ein Anwender versucht zyklische Links zu erzeugen, wird eine \emph{CycleException} erzeugt und der jeweilige Link wird nicht erstellt.

\subsubsection{Löschen eines Links - removeLink}
Beim Entfernen eines bestehenden Links gibt es keinerlei Restriktionen zu beachten. Der Anwender kann bestehende Links beliebig löschen. % Inhalt einbinden

% Kapitel
\subsection{Quantity}
TODO Dient zum Abbilden von Werten mit Einheit

\subsubsection{Die Manager} \newline
\subsubsubsection{UnitTypeManager} \newline
Der UnitTypeManager beinhaltet zwei auf der Oberfläche sichtbare und zwei nicht sichtbare Listen zur Verwaltung von Einheiten (Units) und Einheitentypen (UnitTypes).

\begin{description}
\item[AbsUnitType** unitTypes] Diese Liste enthält alle Einheitentypen, die der Manager verwaltet.
\item[AbsUnit ** units] Hier werden alle Einheiten abgelegt, die der Manager verwaltet.
\item[ReferenceType** refTypes] Diese Liste enthält alle ReferenzTypen, die der Manager verwaltet. Sie ist nicht sichtbar und dient nur zur internen Verarbeitung von zusammengesetzten Einheitentypen (CompoundUnitTypes)
\item[Reference** refs] Hier werden alle Referenzen abgelegt, die der Manager verwaltet. Sie ist nicht sichtbar und dient nur zur internen Verarbeitung von zusammengesetzten Einheiten (CompoundUnits)
\end{description}

\subsubsubsection{QuantityManager} \newline

\subsubsection{Einheitentypen} \newline

\subsubsection{Einheiten} \newline
Fachlich gesehen können Einheiten (Units) von Quantitäten angenommen werden und sind in Einheitentypen typisiert.
Dabei wird zwischen atomaren Einheiten (z.B. Meter [m]) und zusammengesetzten Einheiten (z.B. Kilometer pro Stunde [km/h]) unterschieden.
\subsubsubsection{Verwalten von Units} \newline
Für die Verwaltung von Units und CompoundUnits stellt der UnitTypeManager acht transaktionale Operationen bereit.
Folgende dieser Operationen sind direkt über die Oberfläche erreichbar:

\begin{description}
\item[createUnit]
Diese Operationen dient zum Erstellen einer neuen Unit. Da jede Unit fachlich in einem UnitType typisiert werden und einen Namen haben muss, können dieser Methode diese Werte entsprechend übergeben werden. Eine DoubleDefinitionException wird geworfen, wenn eine Unit mit dem gewählen Nmane bereits existiert.
\item[changeUName]
Diese Operationen dient zum umbenennen einer Unit. Auch hier wird die DoubleDefinitionException im doppelten Namensfall geworfen.
\item[fetchScalar]
Liefert die eine CompountUnit, die keine Referenzen zu anderen Units aufweist.
\item[addReference]
Diese Operation kann sowohl auf Zusammengesetzten Einheiten, als auch auf atomaren Einheiten angewendet werden. Sie dient zum erstellen von CompoundUnits. Entsprechend der createUnit-Methode wird hier ein Name benötigt. Die entsprechende neue Referenz zur ausgewählten Unit wird durch einen Exponenten definiert. Die DoubleDefinitionException wird auch hier geworfen, dalls es zu Namenskonflikten kommt.
\end{description}

Folgende Operationen sind für die interne Verwarbeitung relevant und können nicht über die Oberfläche aufgerufen werden:
\begin{description}

\item[getExistingCU]
Hier wird anhand einer Liste von vorhandenen Referenzen eine CompoundUnit ermittelt, welche durch genau diese Referenzen definiert ist. Sollte diese Unit noch nicht existieren, wird null zurückgeliefert. Diese Operation dient zum vermeiden von Doppelt anlegten CompoundUnits.
\item[fetchCU]
Diese Operation ist ähnlich der getExistingCU()-operation. jedoch wird hierbei die CompoundUnit angelegt, falls sie noch nicht existiert. Die Angabe eines Namens ist erforderlich, falls eine neue CompoundUnit zustande kommen sollte. Auch hier wird entsprechend eine DoubleDefinitionException geworfen, falls eine Unit mit dem gewählen Nmane bereits existiert.
\item[fetchReference]
Mihilfe dieser Operation kann eine Referenz-Instanz mit einem gewissen Exponenten auf eine bestimmte Unit ermittelt werden. Falls diese Instanz noch nicht existierte, wird sie erzeugt. Das dient zum vermeiden von Doppelt anlegten Referenzen.
\end{description}

\subsubsubsection{Conversions} \newline
Eine Conversion, also die Umrechnung von einer Unit zur entspreched zum UnitTyp gehörigen DefaultUnit, kann über zwei Wege zustande kommen, bzw. verändet werden: Zum einen über die setConversion-Operation und zum anderen über die setDefaultUnit-Operation.
Conversions sind immer lineare Funktionen, damit eine Umkehrbarkeit gewährleistet ist. Jede Conversion enthält also einen Faktor und eine Konstante um der linearen Funktion mx+b gerecht zu weden.
Mittels setConversion() wird eine bereits gesetzte Conversion für eine Unit überschrieben.
Bei setDefaultUnit() wird eine Umrechnung von bereits vorhanden Conversions notwendig, da diese ja in Abhängigkeit zu einer nun veralteten DefaultUnit angegeben wurden. Das betrifft alle Umrechnungen für Units zum selben UnitType wie die DefaultUnit. Die folgende Grafik beschreibt die Umrechnungen, falls sich eine DefaultUnit ändert.
\begin{figure}[setDefaultUnit]
	\includegraphics{images/setDefaultUnit}
	\caption{Umrechnung beim Ändern einer DefaultUnit}
\end{figure}

\subsubsection{Quantitäten} \newline % Inhalt einbinden

% Kapitel
\subsection{Measurement}
Martin Fowler unterscheidet in seinem Werk \refLibLiteratur{fowler1997analysis} zwischen qualitativen und quantitativen Messungen. 
Wird beispielsweise die Blutgruppen (A, B, AB und 0) eines Menschen typisiert, so handelt es sich um eine qualitative Messung. 
Im Kontext dieses Projekts wird diese Art von Messung als \term{Observation} (siehe Abschnitt \ref{Operation:Associationen}) bezeichnet. 
Unter quantitativen Messungen, bezeichnet als \term{Measurement}, wird beispielsweise das Feststellen der Körpergröße eines Menschen verstanden.
Measurements (beispielsweise Körpergröße von Hugo Egon Balder: 190cm) werden in Form von Quantitäten bzw. \term{Quantities} auf Konten (Accounts) erfasst.
Des Weiteren gibt es \term{MeasurementTypes} und \term{AccountTypes}, welche in diesem Abschnitt noch genauer erläutert werden.

\TODO[Alex: noch Ergänzungen? Bzw. 'ne bessere Überleitung zum folgenden Absatz?]

Auf der Modellebene befinden sich im Bereich \term{Measurement} zunächst drei Klassen: \term{MeasurementType}, \term{AccountType}
und deren gemeinsame Oberklasse \term{QuantifiedObjectType}. Auf der Exemplarebene finden sich deren Pendants als \term{Measurement},
\term{Account} und \term{QuantifiedObject}.

Die Operation \term{aggregate} des \term{QuantifiedObjects} fordert zudem eine sogenannte \term{AggregationStrategy},
die in vier Unterklassen konkret ausgeprägt werden kann.
Zusätzlich existieren drei Manager: der \term{MeasurementTypeManager}, der \term{AccountTypeManager} und der \term{AccountManager}.
Diese Klassen werden im Folgenden nach Themengebieten beschrieben.


\subsubsection{QuantifiedObject}

\TODO[Kapitel schreiben]

-- Attribute: MObject object

-- Nur Operation aggregate


\subsubsection{QuantifiedObjectType}

\TODO[Kapitel schreiben]

-- Attribute: MType type, AbsUnitType unitType

-- Keine Operationen


\subsubsection{Account}
Ein Konto (Account) gehört zu einem bestimmten Objekt (Object) und beinhaltet Einträge (entries) in Form von quantitativen Messungen (Measurements). 
Jedes Konto ist in einem Typ (AccountType) klassifiziert und kann außerdem mehrere Unterkonten (Komposition von subAccounts) besitzen.
Auf einem Konto kann auf Basis einer definierten Strategie (AggregationStrategy) aggregiert werden. 
Beispielsweise kann so Summe, Durchschnitt, Maximum oder Minimum der Einträge errechnet werden. 

Im Metamodell besitzt ein Account die folgenden Assoziationen.
\begin{description}
	\item[MAccountType type] Typ des Accounts.
	\item[Account** subAccounts hierarchy AccountHierarchy] Liste von Unterkonten.
	\item[Measurement** entries] Liste von Einträgen.
	\item[MObject object] Von QuantifiedObject geerbt.
\end{description}

Hinzu kommen die folgenden Operationen, welche zum Befüllen der o.g. Listen dienen.
\begin{description}
	\item[addEntry] Fügt einen neues Measurement in die entries-Liste hinzu. Das Element muss vom selben Typ sein wie die anderen Einträge der Liste (Konsistenzbedingungen folgen weiter unten).  
	\item[addSubAccount] Fügt einem Konto ein neues Unterkonto hinzu. Hierbei muss der Typ des neuen Unterkontos $\leq$ dem Typ des Hauptkontos sein (Konsistenzbedingungen folgen weiter unten).
\end{description}


\subsubsection{AccountType}
Der Typ eines Accounts (AccountType) dient zur Klassifizieren eines Kontos. Beispielsweise kann ein Konto als Bankkonto klassifiziert werden. Untertyp (SubAccountType) kann an dieser Stelle z.B. ein Giro- oder Kreditkartenkonto sein.
Sobald ein AccountType Exemplare besitzt, kann dieser nicht mehr gelöscht werden (siehe folgender Abschnitt). Um Hierarchien dieser Art abbilden zu können sind die folgenden Assoziationen und Operationen notwendig. 

Im Metamodell besitzt ein AccountType (neben der vom \term{QuantifiedObjectType} geerbten) die folgende Assoziation.
\begin{description}
	\item[MAccountType** subAccountTypes hierarchy MAccountTypeHierarchy] Diese Liste verbindet AccountTypes in Form einer Hierachie. Besonders bei dem Hinzufügen von SubAccounts ist diese Hierarchie relevant (dazu mehr im folgenden Abschnitt).
\end{description}

Hinzu kommt die folgende Operation, welche zum Befüllen der o.g. Liste dient.
\begin{description}
	\item[addSubAccountType] Fügt einen neuen AccountType der subAccountTypes Liste eines AccountTypes hinzu.
\end{description}


\subsubsection{Account \& AccountType}
Wie bereits beschrieben, ist ein Account in einem AccountType klassifiziert. Dieser Typ ist unter anderem relevant, wenn dem Konto ein neues Unterkonto hinzugefügt werden soll.  
In Abbildung \refImg{Measurement:AccTypeHier} wird eine AccountType Hierarchie dargestellt. \term{A} ist in diesem Beispiel der oberste AccountType, welcher die darunterliegenden SubAccountTypes referenziert. 
Auf \term{F} wird von keinem AccountType referenziert und refernziert auch selbst keine anderen AccountTypes.  

\img[width=\relWidth{0.4}]{measurement/AccountTypeHierarchy.png}{AccountType Hierarchie}{Measurement:AccTypeHier}

Nun sollen Exemplare (also Accounts) erstellt werden, welche SubAccounts enthalten. In Abbildung \refImg{Measurement:SubAccounts} können die im Folgenden beschriebenen Fälle nachvollzogen werden.
Der Einfachheit halber wird mit \term{K1:A} ein Account \term{K1} mit AccountType \term{A} bezeichnet. Da $C \leq A$ ist, darf \term{K2} als SubAccount zu \term{K1} hinzugefügt werden.
In den anderen drei Fällen ist der AccountType vom SubAccount entweder nicht $\leq$ dem Typen des Hauptkontos (Beispiel 2 und 4, von links) oder aber der AccountType ist in einer anderen Hierarchie (Beispiel 3, von links).

\img[width=\relWidth{0.5}]{measurement/SubAccounts.png}{SubAccount Exemplare}{Measurement:SubAccounts}

Daraus ergibt sich die folgende Konsistenzbedingung, welche bei Nichteinhaltung zu einer \term{ConsistencyException} führt.
\begin{equation} \forall \, a \in Account: \forall s \in a.subaccounts: s.type \leq a.type
\end{equation}




\subsubsection{AccountManager}
Der AccountManager dient zur Erstellung von Accounts und beinhaltet alle Accounts des Modells.

Im Metamodell besitzt der AccountManager die folgende Assoziation.
\begin{description}
	\item[Account** accounts] Diese Liste beinhaltet alle Accounts des Modells.
\end{description}

Hinzu kommt die folgende Operation, welche zum Befüllen der o.g. Liste dient.
\begin{description}
	\item[createAccount] Dient zum Erstellen von Accounts. Es wird ein Name, ein \term{MAccountType} sowie ein \term{MObject} übergeben, auf welches sich das Konto bezieht.
\end{description}

Neben den Accounts werden auch die \term{AccountTypes} zentral verwaltet. Dazu dient der \term{AccountTypeManager}.


\subsubsection{AccountTypeManager}
Der AccountTypeManager dient zur Erstellung von \term{AccountTypes} und beinhaltet alle \term{AccountTypes} des Modells.

Im Metamodell besitzt der \term{AccountTypeManager} die folgende Assoziation.
\begin{description}
	\item[MAccountType** accountTypes] Beschreiben
\end{description}

Hinzu kommt die folgende Operation, welche zum Befüllen der o.g. Liste dient.
\begin{description}
	\item[createAccountType] Dient zum Erstellen von \term{AccountTypes}. Es wird ein Name, ein \term{MType} und ein \term{UnitType} übergeben.
\end{description}
\TODO[Alex: Brauchen wir hier noch eine Konsistenzbedingung bzgl. Account-AccountType-UnitType = Account.entries-Measurement.Quantity-Type???]

Wie bereits beschrieben werden in Accounts Messungen in Form von Measurements erfasst. 
Im folgenden Abschnitt wird auf die Assoziationen und Operationen von Measurements genauer eingegangen.
\TODO[Überleitung via account.entries - passt so?]


\subsubsection{Measurement}

\TODO[Kapitel schreiben]

-- Attribute: MMeasurementType type, AbsQuantity quantity

-- Keine Operationen


\subsubsection{MeasurementType}

\TODO[Kapitel schreiben]

-- Keine Attribute

-- Keine Operationen


\subsubsection{Measurement \& MeasurementType}

\TODO[Alex: passt das? Oder denk ich da falsch herum?]
\begin{equation} \forall \, m_1, m_2 \in Account.entries: m_1.type = m_2.type
\end{equation}


\subsubsection{MeasurementTypeManager}

\TODO[Kapitel schreiben]

-- Attribute: MMeasurementType** measurementTypes

-- Operationen:

\begin{description}
	\item[createMeasurementType] Beschreiben
\end{description}


\subsubsection{AggregationStrategy}

\TODO[Kapitel schreiben]

-- Keine Attribute

-- Nur Operation aggregateMeasurements


\subsubsection{Löschen im Bereich Measurement}

\TODO[Kapitel schreiben]

-- Was ist löschbar?
-- Was nicht? Warum? % Inhalt einbinden % Inhalt einbinden

% Kapitel
\section{Ausblick}
Im Laufe des Projekts sind viele Ideen aufgekommen, die aufgrund von Entwurfsentscheidungen oder der kurzen Laufzeit des Projektes 
nicht umgesetzt werden konnten. Dieser kurze Ausblick skizziert eine Auswahl von Erweiterungsmöglichkeiten und 
soll nachträgliche Implementierungen anregen.

\begin{description}
  \item[Multiplizitäten] Zurzeit besitzen Assoziationen keine Einschränkungen. 
	Es werden entsprechend überall $0..n$-Multiplizitäten verwendet. 
	Mit der Einführung von Multiplizitäten können Assoziationen eingeschränkt werden.
	Es bietet sich beispielsweise an, eine Klasse \term{Multiplicity} zwischen \term{Association} und \term{Type} zu hängen,
	welche zwei boolsche Attribute $\leq$ und $\geq$ enthält. Damit lassen sich die vier Multiplizitäten $0..n$, $1..n$, $1$ und $0..1$ darstellen.
	Dies zieht u.~a. einige Konsistenzbedingungen nach sich, welche beim Anlegen von Links beachtet werden müssen. 
  \item[Mehrstellige Assoziationen] Aktuell gibt es für eine Assoziation nur genau einen Quell- und genau einen Zieltypen. 
  	Um mehrstellige Assoziationen und Links zu implementieren, bietet sich die Verwendung von Formal- und Aktualparameter an. 
  	Eine Assoziation besteht dann aus den zwei explizit genannten Quell- und Zieltypen sowie einer Liste von Formalparametern, 
  	wobei jeder Formalparameter eine weitere Quelle darstellt. Um dies adäquat auf der Exemplarebene abzubilden, 
  	bedienen sich die Links ihrer Aktualparameter. Das aktuelle \MM sieht diese Implementierung schon vor, da sowohl 
  	Formal- als auch Aktualparameter von den jeweiligen Oberklassen \term{AbstractOperation} und \term{Message++Link} ausgehen.
  \item[Gleichheit von Operationen] Zum derzeitigen Entwicklungsstand sind zwei Operationen gleich, sobald sie den gleichen Namen haben. 
  	Dies ist in vielen Fällen nicht	sinnvoll, da zwei Operationen keinerlei Bezug zueinander haben, wenn sie in unterschiedlichen Typen liegen. 
  	Selbst dann nicht, wenn sie gleich heißen. Ein praktikableres Konzept ist, dass zwei Operationen gleich sind, sobald sie dieselbe Quelle, 
  	den gleichen Namen und die gleichen Parameterlisten (also Anzahl der Parameter und Typisierung dieser) haben. 
  \item[Messages] Wie schon im Kapitel \ref{Message:Message} erläutert, wurden Messages in der ersten Implementierungsstufe nicht umgesetzt. 
  	Zum gegenwärtigen Zeitpunkt existiert für ein Message-Exemplar sowohl die Interpretation des Nachrichtenaufrufs, als auch des Nachrichteninhalts.
	Bevor Messages umgesetzt werden können, muss entschieden werden, welche der genannten Interpretationen für ein Message-Exemplar zutrifft. 
  \item[ActualParameter] Da aus den genannten Gründen weder Message-Exemplare erzeugt, noch mehrstellige Assoziationen/Links angelegt werden können, 
  	haben Aktualparameter derzeit keinerlei Daseinsberechtigung. Von daher wird dem Anwender die Verwaltung von Aktualparametern nicht angeboten.
  \item[Zentrale Ablage der Constraints] Durch die Einführung von \term{ModelItem} wird die Möglichkeit geschaffen, einen Abhängigkeitsgraphen 
	auszuwerten. Dadurch können alle Modellelemente ermittelt werden, deren Konsistenz nach einer Änderung zu prüfen ist. Wenn alle Konsistenzbedingungen 
	in einer zentralen Komponente abgelegt sind, ist es möglich, nach einer Änderung einfach die Konsistenz aller (transitiv) abhängigen Modellelemente zu überprüfen. 
	Da alle Änderungen in Transaktionen geschehen, kann bei einem inkonsistenten Folgezustand ein Rollback eingeleitet werden.
	Großer Vorteil ist, dass veränderliche Elemente keinerlei Kenntnis über die Konsistenzbedingungen ihrer abhängigen Modellobjekte mehr benötigen.
\end{description}

Weitere Themen neben den oben genannten sind beispielsweise Versionierung, Posting Rules, Path. Diese werden in diesem Dokument jedoch nicht weiter ausgeführt. % Inhalt einbinden

% Abbildungsverzeichnis
\newpage
\resetPageTitle % Wichtig: Seitentitel wieder zurücksetzen, da er auf der letzten Seite geändert wurde
\listoffigures % Abbildungsverzeichnis einfügen

% Tabellenverzeichnis
%\newpage
%\listoftables % Tabellenverzeichnis einfügen

% Abkürzungsverzeichnis
\newpage
\renewcommand*{\glspostdescription}{} % Text am Ende eines Verzeichniseintrags (Standard: ".")
\printglossary[type=\acronymtype,style=super,nonumberlist] % Abkürzungen einbinden

% Glossar
\newpage
\renewcommand*{\glspostdescription}{~--} % Text am Ende eines Glossareintrags (Standard: ".")
\printglossary[style=altlist,nonumberlist=false] % Glossar ausgeben

% Quellenverzeichnis
\newpage
\setPageTitle{Quellenverzeichnis} % Seitentitel ändern, da diese Section ausgeblendet ist
\section*{Quellenverzeichnis}
\addcontentsline{toc}{section}{Quellenverzeichnis} % Fügt diesen Punkt ins Inhaltsverzeichnis ein

% Formatieren des Quellenverzeichnisses (alphadin = Formatierung nach DIN 1505)
\bibliographystyleLiteratur{alphadin} % Literatur formatieren
\bibliographystyleRecht{alphadin} % Rechtsvorschriften formatieren
\bibliographystyleInternet{alphadin} % Internetquellen formatieren
\bibliographystyleUnternehmen{alphadin} % Unternehmensquellen formatieren

\bibliographyLiteratur{\content/lib/Literatur} % Einfügen der Literatur
%\bibliographyRecht{\content/lib/Recht} % Einfügen der Rechtsvorschriften
%\bibliographyInternet{\content/lib/Internet} % Einfügen der Internetquellen
%\bibliographyUnternehmen{\content/lib/Unternehmen} % Einfügen der Unternehmensquellen % Inhalt einbinden

% Beginn des Anhangs
\newpage
\resetPageTitle % Wichtig: Seitentitel wieder zurücksetzen, da er auf der letzten Seite geändert wurde
\appendix % Beginn des Anhangs (diese Zeile auskommentieren, falls kein Anhang vorhanden ist)
\appendixtoc % Schwarze Magie - niemals auskommentieren
% Kapitel einbinden
%\include{\content/sections/appendix/Beispielanhang}

% Kapitel einbinden
%\include{\content/sections/appendix/Weiterer_Anhang} % Inhalt einbinden

\end{document}