%%%%%%%%%%%%%%%%%%%%%%%%%%%%%%%%%%%%%%%%%%%%%%%%
%              ALLGEMEINE BEFEHLE              %
%%%%%%%%%%%%%%%%%%%%%%%%%%%%%%%%%%%%%%%%%%%%%%%%

% Neuer Befehl: Unterkapitel unter \subsubsection
\newcommand{\subsubsubsection}[1]{\paragraph{#1}\hfill}

% Neuer Befehl: Unterkapitel unter \subsubsubsection
\newcommand{\subsubsubsubsection}[1]{\subparagraph{#1}\hfill}

% Neuer Befehl: Wechsel auf Sonderschriftart
\newcommand{\fontHeadline}[1]{{\sffamily#1}}

% Neuer Befehl: Setzt den Seitentitel auf den übergebenen Wert
%				OBACHT! ändert den Titel ebenfalls für alle folgenden Seiten -> \resetPageTitle benutzen
\newcommand{\setPageTitle}[1]{\fancyfoot[L]{\fontHeadline{\footnotesize\nouppercase{#1}}}}
% Neuer Befehl: Setzt den Seitentitel auf den Standardwert
\newcommand{\resetPageTitle}{\setPageTitle{\leftmark}}

% Neuer Befehl: Gibt die übergebene Breite relativ zur Textbreite zurück
\newcommand{\relWidth}[1]{#1\textwidth}

% Neuer Befehl: Fachwörter im Text hervorheben
\newcommand{\term}[1]{\emph{#1}}

% Neuer Befehl: Gibt den übergebenen Text in einfachen deutschen Anführungszeichen zurück
\newcommand{\q}[1]{\glq#1\grq\xspace}

% Neuer Befehl: Gibt den übergebenen Text in doppelten deutschen Anführungszeichen zurück
\newcommand{\qq}[1]{\glqq#1\grqq\xspace}

% Neuer Befehl: Ein sichtbares TODO in den Text einfügen
\newcommand{\todo}[1]{%
	\textcolor{redD}{\fontHeadline{{\bfseries ~TODO~}#1~}}%
	\marginnote{\textcolor{redD}{\fontHeadline{{\bfseries !!!}}}}%
}

% Neuer Befehl: Ein sichtbares TODO in den Text einfügen
\newcommand{\TODO}[1][Hier gibt es etwas zu tun]{\todo{#1}}

% Neuer Befehl: Eine sichtbare Frage in den Text einfügen
\newcommand{\QUESTION}[1][]{%
	\textcolor{green8}{\fontHeadline{{\bfseries ~FRAGE~}#1~}}%
	\marginnote{\textcolor{green8}{\fontHeadline{{\bfseries ???}}}}%
}


%%%%%%%%%%%%%%%%%%%%%%%%%%%%%%%%%%%%%%%%%%%%%%%%
%                REFERENZIEREN                 %
%%%%%%%%%%%%%%%%%%%%%%%%%%%%%%%%%%%%%%%%%%%%%%%%

% Neuer Befehl: Erzeugt ein neues Label für ein Kapitel, damit es referenziert werden kann
% @in {}: Labelname des Kapitels
\newcommand{\lblSec}[1]{\label{sec:#1}}

% Neuer Befehl: Gibt eine Referenz zu dem übergebenen Kapitel zurück
% @in {}: Labelname des Kapitels
% @in []: Beschriftung vor der Referenz
\newcommand{\refSec}[2][Abschnitt~]{#1\ref{sec:#2}}

% Neuer Befehl: Gibt die Seite zurück, auf der sich das übergebene Kapitel befindet
% @in {}: Labelname des Kapitels
% @in []: Beschriftung vor der Referenz
\newcommand{\refPageSec}[2][Seite~]{#1\pageref{sec:#2}}

% Neuer Befehl: Gibt eine Referenz zu dem übergebenen Bild zurück
% @in {}: Labelname des Bildes
% @in []: Beschriftung vor der Referenz
\newcommand{\refImg}[2][Abbildung~]{#1\ref{img:#2}}

% Neuer Befehl: Gibt die Seite zurück, auf der sich das übergebene Bild befindet
% @in {}: Labelname des Bildes
% @in []: Beschriftung vor der Referenz
\newcommand{\refPageImg}[2][Seite~]{#1\pageref{img:#2}}

% Neuer Befehl: Gibt eine Referenz zu der übergebenen Tabelle zurück
% @in {}: Labelname der Tabelle
% @in []: Beschriftung vor der Referenz
\newcommand{\refTbl}[2][Tabelle~]{#1\ref{tbl:#2}}

% Neuer Befehl: Gibt die Seite zurück, auf der sich die übergebene Tabelle befindet
% @in {}: Labelname der Tabelle
% @in []: Beschriftung vor der Referenz
\newcommand{\refPageTbl}[2][Seite~]{#1\pageref{tbl:#2}}

% Neuer Befehl: Gibt eine Referenz zu dem übergebenen Bibliothekseintrag zurück
% @in {}: Labelname des Bibliothekseintrags
% @in []: Seiten- bzw. Stellenangabe innerhalb des Eintrags
\newcommand{\refLibLiteratur}{\citeLiteratur}
\newcommand{\refLibInternet}{\citeInternet}
\newcommand{\refLibRecht}{\citeRecht}
\newcommand{\refLibUnternehmen}{\citeUnternehmen}


%%%%%%%%%%%%%%%%%%%%%%%%%%%%%%%%%%%%%%%%%%%%%%%%
%            ABKÜRZUNGEN & GLOSSAR             %
%%%%%%%%%%%%%%%%%%%%%%%%%%%%%%%%%%%%%%%%%%%%%%%%

% Neuer Befehl: Definiert eine neue Abkürzung.
% @in {}: Label
% @in {}: Abkürzung
% @in {}: Beschreibung
\newcommand{\addAc}[3]{%
	\newacronym{#1}{#2}{#3}%
}

% Neuer Befehl: Definiert eine neue Abkürzung mit Pluralformen.
% @in {}: Label
% @in {}: Abkürzung
% @in {}: Abkürzung im Plural
% @in {}: Beschreibung
% @in {}: Beschreibung im Plural
\newcommand{\addAcp}[5]{%
	\newacronym[plural={#3},descriptionplural={#5}]{#1}{#2}{#4}%
}

% Neuer Befehl: Gibt beim ersten Vorkommen Beschreibung + Abkürzung aus,
% danach nur noch die Abkürzung
% Hinweis: Beeinflusst den \ac-Befehl
% @in {}: Label der Abkürzung
\newcommand{\ac}{\gls}

% Neuer Befehl: Gibt die Abkürzung mit angehängtem Suffix aus
% (z. B. für den Genitiv).
% Hinweis: Beeinflusst den \ac-Befehl
% @in {}: Label der Abkürzung
% @in {}: Suffix
\newcommand{\ace}[2]{\glsdisp{#1}{\glsname{#1}#2}}

% Neuer Befehl: Gibt nur die Abkürzung aus.
% Hinweis: Beeinflusst den \ac-Befehl _nicht_
% @in {}: Label der Abkürzung
\newcommand{\acs}{\glsname}

% Neuer Befehl: Gibt nur die Beschreibung aus.
% Obacht: Funktioniert nicht mit Sonderzeichen oder Umlauten!
% Hinweis: Beeinflusst den \ac-Befehl _nicht_
% @in {}: Label der Abkürzung
\newcommand{\acl}{\glsdesc}

% Neuer Befehl: Gibt immer Beschreibung + Abkürzung aus.
% Hinweis: Beeinflusst den \ac-Befehl _nicht_
% @in {}: Label der Abkürzung
\newcommand{\acf}{\glsfirst}

% Neuer Befehl: Gibt beim ersten Vorkommen Beschreibung + Abkürzung im Plural aus,
% danach nur noch die Abkürzung
% Hinweis: Beeinflusst den \ac-Befehl
% @in {}: Label der Abkürzung
\newcommand{\acp}{\glspl}

% Neuer Befehl: Gibt die Abkürzung im Plural aus.
% Hinweis: Beeinflusst den \ac-Befehl _nicht_
% @in {}: Label der Abkürzung
\newcommand{\acsp}{\glsplural}

% Neuer Befehl: Gibt Beschreibung + Abkürzung im Plural aus.
% Hinweis: Beeinflusst den \ac-Befehl _nicht_
% @in {}: Label der Abkürzung
\newcommand{\acfp}{\glsfirstplural}

% Neuer Befehl: Gibt einen Link zur angegebenen Abkürzung
% mit dem angegebenen Text (statt der vordefinierten Beschreibung) aus.
% Hinweis: Beeinflusst den \ac-Befehl
% @in {}: Label der Abkürzung
% @in {}: Anzeigetext
\newcommand{\aca}{\glsdisp}

% Neuer Befehl: Glossareintrag hinzufügen.
% @in {}: Label des Eintrags
% @in {}: Name des Eintrags
% @in {}: Pluralform des Namens
% @in {}: Beschreibung
\newcommand{\addGl}[4]{%
	\newglossaryentry{#1}{name=#2,description={#4},plural=#3,sort=#1}%
}

% Neuer Befehl: Gibt den Namen des Glossareintrags aus
% und erzeugt einen Link zum entsprechenden Eintrag.
% @in {}: Label des Eintrags
\newcommand{\gl}{\gls}

% Neuer Befehl: Gibt den Namen des Glossareintrags in der Pluralform aus
% und erzeugt einen Link zum entsprechenden Eintrag.
% @in {}: Label des Eintrags
\newcommand{\glp}{\glspl}

% Neuer Befehl: Gibt einen Link zum angegebenen Glossareintrag
% mit dem angegebenen Text (statt der vordefinierten Beschreibung) aus.
% @in {}: Label der Eintrags
% @in {}: Anzeigetext
\newcommand{\gla}{\glsdisp}


%%%%%%%%%%%%%%%%%%%%%%%%%%%%%%%%%%%%%%%%%%%%%%%%
%                    BILDER                    %
%%%%%%%%%%%%%%%%%%%%%%%%%%%%%%%%%%%%%%%%%%%%%%%%

% Neuer Befehl: Bild ohne Beschriftung einfügen
% @in {} 1:	Pfad des Bildes
% @in []:	Optionen für das Bild wie Breite, Drehung, etc. - mehrere Optionen kommasepariert, Reihenfolge einhalten!
%			Key			Value				Beschreibung
%			---------------------------------------------------------------------------
%			angle		<zahl>				Dreht das Bild um <zahl> Grad
% 			width		<zahl>				Breitenangabe in Pixeln
%						<zahl>cm/mm/in		Breitenangabe in Zentimeter/Millimeter/Inch
%						\relWidth{<zahl>}	Breitenangabe relativ zur Textbreite
%			height		s. width
%			scale		<zahl>				Skaliert das Bild auf die angegebene Größe
%			---------------------------------------------------------------------------
%			Beispiel: \imgUnlabeled{bild.jpg} oder \imgUnlabeled[angle=-90,width=42cm]{bild.jpg}
\newcommand{\imgUnlabeled}[2][scale=1.0]{\includegraphics[#1]{\imgPath#2}}

% Neuer Befehl: Bild mit Beschriftung einfügen
% @in {}1:	Pfad des Bildes
% @in {}2:	Sichtbare Bildbeschriftung
% @in {}3:	Internes Label (eineindeutig im Gesamtdokument!)
% @in []:	Optionen für das Bild, s. \imgUnlabeled
% Beispiel: \img{bild.jpg}{Ein schönes Bild}{bild1} oder \img[angle=-90,width=42cm]{bild.jpg}{Ein schönes Bild}{bild1}
\newcommand{\img}[4][scale=1.0]{
	\begin{figure}[h!tbp]
	\centering
	\imgUnlabeled[#1]{#2}
	\caption{#3}
	\label{img:#4}
	\end{figure}
}


%%%%%%%%%%%%%%%%%%%%%%%%%%%%%%%%%%%%%%%%%%%%%%%%
%                   TABELLEN                   %
%%%%%%%%%%%%%%%%%%%%%%%%%%%%%%%%%%%%%%%%%%%%%%%%

% Neuer Befehl: Fügt eine Tabelle ohne Linien ein
% @in {}: Sichtbare Beschriftung
% @in {}: Labelname zum Referenzieren der Tabelle
% @in {}: Tabellenstil, [ll] erzeugt beispielsweise zwei linksbündige Spalten
% @in {}: Tabelleninhalt (Tipp: Neue Zeilen mit \tr[] erzeugen)
% @in {}: Zusatzoptionen wie Farbdefinitionen für Zeilen
%		  \rowcolors{1}{red}{green} erzeugt bspw. ungerade Zeilen mit rotem
%		  und gerade Zeilen mit grünen Hintergrund ab der ersten Zeile
% @see: http://en.wikibooks.org/wiki/LaTeX/Tables
\newcommand{\tblNoBorder}[5]{
	\begin{table}[h!tbp]
	#5
	\caption{#1}
	\centering
	\begin{tabular}[t]{#3}
	#4
	\end{tabular}
	\label{tbl:#2}
	\end{table}
}

% Neuer Befehl: Fügt eine Tabelle mit einer horizontalen Trennlinie oben ein
% @in {}: Sichtbare Beschriftung
% @in {}: Labelname zum Referenzieren der Tabelle
% @in {}: Tabellenstil, [|l|l|] erzeugt beispielsweise zwei linksbündige Spalten mit vertikalen Trennlinien
%		  Mögliche Optionen: l = linksbündig, r = rechtsbündig, c = zentriert
%							 p{\relWidth{0.5}} = linksbündig mit fester Breite, bricht Text automatisch um
% @in {}: Tabelleninhalt (Tipp: Neue Zeilen mit \tr erzeugen)
% @in {}: Zusatzoptionen wie Farbdefinitionen für Zeilen
%		  \rowcolors{1}{red}{green} erzeugt bspw. ungerade Zeilen mit rotem
%		  und gerade Zeilen mit grünen Hintergrund ab der ersten Zeile
% @see: http://en.wikibooks.org/wiki/LaTeX/Tables
\newcommand{\tbl}[5]{\tblNoBorder{#1}{#2}{#3}{\hline#4}{#5}}

% Neuer Befehl: Erzeugt eine neue Zeile in einer Tabelle
% @in []: Freilassen (nicht weglassen!), um keine Linie zu erzeugen
\newcommand{\tr}[1][\hline]{\\#1}

% Neuer Befehl: Formatiert Text in einer Zelle als Überschriftszelle
% @in {}: Der Text in der Zelle (ohne "&")
% @in []: Hintergrundfarbe der Zelle
\newcommand{\thl}[2][\defaultTableHeadlineBackground]{\textbf{#2}\cellcolor{#1}}

% Neuer Befehl: Zusatzoption für Farbdefinitionen in Tabellen: Färbt die Zeilen abwechseln mit Standardfarben
% @in []: Erste Zeile ab der gefärbt wird (standardmäßig ab der ersten, sprich alle Zeilen werden gefärbt)
\newcommand{\defaultTblBg}[1][1]{\rowcolors{#1}{\defaultTableRowBackgroundOdd}{\defaultTableRowBackgroundEven}}


%%%%%%%%%%%%%%%%%%%%%%%%%%%%%%%%%%%%%%%%%%%%%%%%
%       SEITENSPEZIFISCHE FORMATIERUNGEN       %
%%%%%%%%%%%%%%%%%%%%%%%%%%%%%%%%%%%%%%%%%%%%%%%%

% Neuer Befehl: Formatbox Titelseite
\newcommand{\titlepageBox}[1]{\fontHeadline{\normalsize#1}\\\vfill{}}

% Neuer Befehl: Formatbox Titelseite mit Überschrift
\newcommand{\titlepageBoxHeadline}[2]{\fontHeadline{\normalsize#1}\\[0.2cm]\titlepageBox{#2}}