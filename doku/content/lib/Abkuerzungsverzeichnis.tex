% Kurzanleitung (weiteres siehe Commands-Datei)

% Eine Abkürzung wird folgendermaßen hinzugefügt:
%	\addAc{Label}{Kurzform}{Langform}
% Das Label sollte dabei nur [A-Za-z0-9] enthalten
% und wird nur zum Referenzieren benutzt (nie im Dokument).
% Es muss natürlich global eindeutig sein.

% Alternativ lässt sich eine Abkürzung mit Pluralform hinzufügen:
%	\addAcp{Label}{Kurzform}{Kurzform im Plural}{Langform}{Langform im Plural}

% Eine Abkürzung wird im Text folgendermaßen referenziert:
%	\ac{Label}	Gibt beim ersten Vorkommen Beschreibung + Abkürzung aus, danach nur noch die Abkürzung
%	\acf{Label}	Beschreibung + Abkürzung wird ausgegeben
%	\acs{Label}	Nur die Abkürzung wird ausgegeben
%	\acl{Label}	Nur die Beschreibung wird ausgegeben

% Für die Pluralformen gibt es analog:
%	\acp{Label}
%	\acfp{Label}
%	\acsp{Label}
% Leider gibt es \aclp{Label} aufgrund eines internen Fehlers nicht.

% Möchte man nicht die normale Kurz- bzw. Langform ausgeben,
% gibt es zudem die folgende Möglichkeit:
%	\aca{Label}{Alternative Beschriftung}

% Und falls man z. B. den Genitiv bilden möchte:
%	\ace{Label}{Suffix}	Gibt die Kurzform mit angehängtem Suffix aus

\addAc	{EDV}{EDV}{Elektronische Datenverarbeitung}
\addAc	{GmbH}{GmbH}{Gesellschaft mit beschränkter Haftung}
\addAcp	{ibA}{ibA}{ibAs}{iVAS-basierende Anwendung}{iVAS-basierende Anwendungen}
\addAc	{IT}{IT}{Informationstechnologie}
\addAc	{iVAS}{iVAS}{ivv Versicherungsanwendungssystem}
\addAc	{ivv}{ivv}{Informationsverarbeitung für Versicherungen GmbH}
\addAc	{Java EE}{Java EE}{Java Platform, Enterprise Edition}
\addAc	{OEVO}{ÖVO}{Öffentliche Versicherung Oldenburg}
\addAc	{VGH}{VGH}{Versicherungsgruppe Hannover}