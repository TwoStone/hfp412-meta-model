\begin{document} % Ab hier beginnt der wirkliche Text, alles nachfolgende erscheint im Dokument
\resetPageTitle % Seitentitel initial setzen

% Titelseite
\pagenumbering{Roman} % Nummerierung der Seiten in römischen Zahlen
\thispagestyle{empty} % Kopf- und Fußzeile ausblenden
\begin{center}

~\vfill

\titlepageBox{
	\imgUnlabeled[width=\relWidth{0.35}]{logo-fhdw.pdf}
}

\titlepageBoxHeadline{Fachhochschule für die Wirtschaft Hannover}{-- FHDW --}

\titlepageBoxHeadline{Dokumentation des Projekts}{\textbf{Metamodell}}

\titlepageBoxHeadline{Verfasser:}{
	Alexander Bellhäuser,
	Björn Bodensieck,\\
	Malte Kastner,
	Thimo Koenig,\\
	Stefan Pietsch,
	Stefanie Rademacker,\\
	Marius Tempelmeier,
	Niklas Walter,\\
	Christin Weckbrod,
	Patrick Wolf
}

\titlepageBoxHeadline{1. Theoriequartal}{%
	Studiengang Master of Science (M.Sc.)\\%
	Business Process Engineering%
}

\titlepageBoxHeadline{Eingereicht am:}{\submissionDate}

\end{center} % Inhalt einbinden

% Abstract
%\newpage
%\setPageTitle{Abstract} % Seitentitel ändern, da diese Section ausgeblendet ist
%\section*{Abstract}

 % Inhalt einbinden

% Inhaltsverzeichnis
\newpage
\resetPageTitle % Wichtig: Seitentitel wieder zurücksetzen, da er auf der letzten Seite geändert wurde
\maintoc % Inhaltsverzeichnis einfügen

% Beginn des Inhalts
\newpage
\pagenumbering{arabic} % Nummerierung der Seiten in arabischen Zahlen
\setcounter{page}{1} % Seitenzahlen zurück auf Anfang setzen
\glsresetall % Alle verwendeten Abkürzungen zurücksetzen

% Kapitel
\section{Einleitung}

Im Rahmen des Master Studiums (M.Sc. Studiengang: Business Process Engineering) an der FHDW Hannover
war es für die Studiengruppe \qq{HFP412} die Aufgabe, Inhalte und Hintergründe der Veranstaltung \qq{Quantitative Forschungsmethoden}\footnote{Dozentin: Frau Dr. Sylvie Gasnier} zu dokumentieren.

\img[width=\relWidth{0.9}]{skalenniveaus.pdf}{Verschiedene Zahlen- oder Skalenniveaus}{img_skalenniveaus}

Die vorliegende Ausarbeitung konzentriert sich auf den Zweig der qualitativen Skalen. Genauer das Thema der \qq{Nominal- und Ordinalskalen} (siehe \refImg{img_skalenniveaus}). 
Die theoretischen Hintergründen werden anhand von praktischen Beispielen mit Hilfe von Microsoft Excel untermauert. 
Als Datenquelle dient hierfür der Münchner Mietspiegel aus dem Jahre 2003\footnote{\url{http://data.ub.uni-muenchen.de/2/1/miete03.asc}}.

Um die statistische Auswertung zu erleichtern, müssen einige Vorarbeiten geleistet werden. Zunächst müssen pro Merkmalsart die verschiedenen Ausprägungen 
analysiert werden. Hierbei entstehen ein Codeplan und eine Datenmatrix, welche als Grundlage für die folgenden Auswertungen dient. 
Auf die konkreten Ausprägungsarten wird zu einem späteren Zeitpunkt eingegangen. Um an die
Ausprägungsarten herzuführen, werden in Abschnitt \ref{sec:defNot} zunächst Codeplan und Datenmatrix
skizziert.
 % Inhalt einbinden

% Kapitel
%\section{Grundlagen}

\blindtext

\subsection{Idee des Metamodells}

\blindtext

\subsection{Theoretischer Ansatz von Martin Fowler}

\TODO[Bezug zum Buch herstellen und als Auslöser darstellen] \refLibLiteratur{fowler1997analysis}

\TODO[Beschreiben, welche Teile umgesetzt wurden, und dann auf die Folgekapitel verweisen]
 % Inhalt einbinden

% Kapitel
\section{Metamodell} 

% Kapitel
\subsection{ModelItem}

\term{ModelItem} ist als Abstraktion für alle Modellelemente konzeptioniert. 
Zentrale Aufgabe dieser Klasse ist es, einen Abhängigkeitsgraphen über Modellelemente erstellen zu können und 
das Löschen dieser zu vereinheitlichen. Der Abhängigkeitsgraph ist, durch die Modellierung bedingt, azyklisch.
Die in der Implementierung angewandte Pragmatik verbietet das Löschen von Modellelementen, von denen weitere Elemente 
abhängig sind. Der Entwurf ermöglicht es auch nachträglich dieses Vorgehen mit geringem Aufwand durch ein kaskadierendes Löschen zu 
ersetzen. 

\img[width=\relWidth{0.5}]{modelItem/modelItem.png}{Klasse ModelItem}{img_modelItem}


\subsubsection{Operationen}
\begin{description}
\item[fetchDependentItems()] 
liefert alle Modellelemente zurück, die vom Element auf dem sie aufgerufen wird, direkt abhängig sind.
Ein Element \emph{A} ist von einem Element \emph{B} abhängig, wenn \emph{A} eine Referenz auf \emph{B} hält, also z.b.   
\emph{A} in \emph{B} klassifiziert ist oder u.a. aus einem \emph{B}-Objekt besteht.
 
\item[delete()] vereinheitlich das Löschen von ModelItems. ModelItem stellt eine \term{Template-Method} für das 
Standardverhalten bereit. Es wird zuerst \term{fetchDependentItems()} aufgerufen und geprüft ob abhängige Modellelemente 
existieren. Werden Abhänigkeiten gefunden, so wird eine \term{ConsistencyException} ausgelöst. Andernfalls wird anschließend 
die abstrakte Operation \emph{prepareForDeletion()} aufgerufen. Lässt sich diese ohne Ausnahme ausführen wird das Modellelement 
abschließend durch den Aufruf von \emph{getThis().delete\$ME} gelöscht.

\item[prepareForDeletion()] ist als Analogon zu \term{initializeOnCreation()} entworfen worden und ermöglicht ein 
\qq{Aufräumen} vor dem tatsächlichen Löschen. Ein Singleton-Type erstellt z.B. im \term{initializeOnCreation()} sein zugehöriges 
Singleton-Objekt und löscht dies in \emph{prepareForDeletion()}. 

\end{description}
\subsubsection{Weitere Ideen}

\TODO[Kapitel schreiben]
 % Inhalt einbinden

% Kapitel
\newpage
\subsection{Typesystem}
Das Typsystem ist die zentrale Komponente des Metamodells. Im folgenden Abschnitt sind die 
Änderungen bezüglich des in der Vorlesung erarbeiteten Modells kurz zusammengefasst.

\subsubsection{Aspekte und atomare Typen}

Aspekte stellen gemäß ursprünglichen Modellierung eine Dimension dar, in deren Elementen sich Objekte dynamisch klassifizieren lassen. 
Atomare Typen der Klasse \emph{AtomicType} werden genau einem Aspekt zugeordnet. Durch die \emph{lessThan}-Assoziation lässt sich aspektintern eine 
partielle Ordnung \emph{isLessOrEqualThan} ableiten.  

\img[width=\relWidth{0.75}]{type/Aspekte.png}{Umarrangierter Ausschnitt Aspekte}{img_aspekts}

Das Interface \emph{AnythingORMATomicType} ist als Zieltyp der \emph{lessThan}-Assoziation ergänzt worden, 
um die Möglichkeit zu schaffen einen atomaren Typen mit Obertyp so zu ändern, dass er anschließend keinen 
Obertypen mehr besitzt. Folgende Möglichkeiten wurden in der Entwurfsphase betrachtet: 
\begin{enumerate}
  		\item superType Assoziation mit Multiplizität 0..1
		\item State-Pattern für atomare Typen \(state \in \{rootType, subType\}\)
        \item automatisch mitgenerierte Obertypen für Aspekte
        \item Interface über Anything und AtomicType
\end{enumerate}

Die erste Variante lässt sich in GOJA aufgrund der restriktiven Set-Methoden nicht realisieren, da diese keine NULL-Werte akzeptieren.
Ein State-Pattern wäre die sauberste Implementierung gewesen, ist aber wegen fehlenden Mehrwertes ausgeschieden. 
Variante drei hat den Nachteil, dass die mitgenerierten Typen die Semantik \qq{unkategorisiert} im zugehörigen Aspekt besitzen.
Damit wird pro Aspekt ein Platzhalter für Null generiert und sorgt für zusätzliche Komplexität ohne Mehrwert. 
Da ohne State-Pattern aber eine Repräsentation für keinen atomaren Obertypen benötigt wird, wird an dieser Stelle \emph{Anything} verwendet 
da es ohnehin Obertyp eines jeden Typen ist. Dies ist insofern unsauber, als das \emph{Anything} nicht Teil eines einzigen Aspektes ist.
Weitere Nachteile sind bisher nicht ersichtlich.


\subsubsubsection{Aspekte}

\textbf{Zusätzliche Konsistenzbedinungen} \newline
Die einzige Konsistenzbedingung von Aspekten ist, dass sie paarweise verschiedene Namen tragen müssen. 

\textbf{Manager} \newline
Aspekte werden vom \emph{AspectManager} verwaltet. Alle Operation werden transaktional aufgerufen und prüfen die Konsistenzbedingung. 

\begin{description}
\item[createAspect] legt einen neuen Aspekt an.
\item[renameAspect] benennt einen Aspekt um.
\item[deleteAspect] loescht einen Aspekt, wenn keine abhängingen Modellelemente existieren.
\end{description}

\textbf{Abhängige Elemente}
\begin{enumerate}
  		\item Atomare Typen, die in diesem Aspekt angelegt sind.
\end{enumerate}


\subsubsubsection{Atomare Typen}

\textbf{Zusätzliche Konsistenzbedinungen}
\begin{enumerate}
  		\item Paarweise verschiedene Namen in einem Aspekt
  		\item \(\forall a \in AT : !(a.singleton \land a.abstract) \)
\end{enumerate}

\textbf{Manager} \newline
Atomare Typen werden vom \emph{TypeManager} verwaltet. 

\begin{description}
\item[createAtomicRootType] legt einen neuen atomaren Typen mit Obertyp \emph{Anything} an. 
\item[createAtomicSubType] legt einen neuen atomaren Typen unter einem andern an. 
\item[renameAtomicType] benennt einen atomaren Typen um. 
\item[changeAbstract] ist eine modifizierte Set-Methode. Kann einen konkreten Typen nur abstrakt umdeklarieren, wenn er weder Singletons ist, noch 
Objekte in ihm klassifiziert sind.
\item[changeSingleton] ist eine modifizierte Set-Methode. Kann die Singletoneigenschaft nur wiederrufen, wenn das \emph{SingletonObject} gelöscht werden kann. 
 Kann die Singletoneigenschaft nur gewähren, wenn keine Objekte existieren und der Typ nicht abstrakt ist. 
\end{description}

\textbf{Singletons} \newline
Die Singletoneigenschaft eines atomaren Typen A bedeutet ausschließlich, dass genau ein Objekt existiert, dessen konkretester Typ A ist.
Von Singletons darf beliebig abgeleitet werden. 
In der \emph{initializeOnCreation}-Operation des atomaren Typen wird, falls er über die Singletoneigenschaft verfügt, 
sein \emph{SingletonObject} erstellt. In der \emph{prepareForDeletion}-Operation wird - falls vorhanden - das \emph{SingeletonObject}
mitgelöscht. Die \emph{changeSingleton}-Operation verwaltet ebenfalls \emph{SingletonObject}s. In den den entsprechenden Operationen im 
\term{TypeManager} werden \term{SingeltonObject}s beim \term{ObjectManager} registriert bzw. entfernt.
 
\textbf{Abhängige Elemente}
\begin{enumerate}
  		\item Atomare Typen: direkte Untertypen
  		\item bei Singletons: Abhängige Objekte des \emph{SingletonObject}s
  		\item \ldots siehe Typabstraktion
\end{enumerate}

\subsubsection{Komplexe Typen}

Komplexe Typen bieten die Möglichkeit Und- und Oder-Typen zu bilden. Zyklenfreiheit wird hierbei dadurch gewährleistet, 
dass \emph{containedTypes} und alle Spezialisierungen Teil einer Hierarchie sind. Komplexe Typen sind immutable,
lassen sich aber (bis auf \emph{Anything} und \emph{Nothing}) löschen, falls keine Abhängigkeiten existieren.

\img[width=\relWidth{1}]{type/ComplexTypes.png}{Ergänzter Ausschnitt: Komplexe Typen}{img_complexTypes}

Um für Unterklassen von \emph{ComplexType} die Assoziation \emph{containedTypes} weiter einschränken zu 
können, ist sie spezialisiert worden. Ursprünglich ist \emph{containedTypes} abgeleitetes Attribut gewesen,
ist aber wegen Problemen in GOJA zu einer \emph{fetch}-Operation refaktoriert worden. 
Ziel ist, z.B. in den leeren Typen keine Änderungen an der Assoziation zu erlauben und diese für die Bereitstellung 
der disjunktiven Normalform mit weiteren Contraints zu versehen (siehe z.B. \emph{NonEmptyAtomicTypeConjunction}) 

Bei Komplexen Typen wird zwischen struktureller und semantischer Äquivalenz unterschieden. Zwei Typen sind strukturell 
äquivalent, wenn sie identisch aufgebaut sind. Zwei Typen A,B sind semantisch äquivalent, wenn nach den in der Vorlesung 
aufgestellten Regeln für \term{isLessOrEqual} gilt: 

\[A.isLessOrEqual(B) \land B.isLessOrEqual(A) \]


\textbf{Zusätzliche Konsistenzbedinungen} \newline
\begin{equation}\forall t \in NEATConjunction : 
	\forall ct \in t.containedTypes : ct.instanceOf AT
\end{equation}
\begin{equation}\forall t \in NEATConjunction : 
	t.containedTypes.legth > 0
\end{equation}
\begin{equation}\forall t \in NEDisjunctiveNF : 
	\forall ct \in t.containedTypes : ct.instanceOf ATConjunction
\end{equation}
\begin{equation}\forall t \in NEDisjunctiveNF : 
	t.containedTypes.legth > 0
\end{equation}

\textbf{Manager} \newline
Komplexe Typen werden vom \emph{TypeManager} verwaltet. 

\begin{description}
\item[createAtomicRootType] legt einen neuen atomaren Typen mit Obertyp \emph{Anything} an. 
\item[createAtomicSubType] legt einen neuen atomaren Typen unter einem andern an. 
\item[renameAtomicType] benennt einen atomaren Typen um. 
\item[changeAbstract] ist eine modifizierte Set-Methode. Kann einen konkreten Typen nur abstrakt umdeklarieren, wenn er weder Singletons ist, noch 
Objekte in ihm klassifiziert sind.
\item[changeSingleton] ist eine modifizierte Set-Methode. Kann die Singletoneigenschaft nur wiederrufen, wenn das \emph{SingletonObject} gelöscht werden kann. 
 Kann die Singletoneigenschaft nur gewähren, wenn keine Objekte existieren und der Typ nicht abstrakt ist. 
\end{description}




\TODO[Kapitel schreiben] % Inhalt einbinden

% Kapitel
\newpage
\subsection{Objektwelt}
Im Folgenden werden die Anpassungen der Klasse \term{Object} und deren Methoden im Vergleich zum Modell aus der Vorlesung betrachtet.
\subsubsection{Änderungen und neue Methoden}
Die Klasse \term{Object} wurde wie viele Klassen im Modell um den Präfix \qq{M} erweitert um Verwechselungen mit der Klasse \term{java.lang.Object} zu vermeiden.
Desweiteren wurde der Klasse die Methode \term{getProductType()} hinzugefügt, die ein MTypeConjunction liefert.
Diese TypeConjunction repräsentiert die Produkttypen des Objekts, der implizit durch die Assoziation \term{types} dargestellt wird.

\subsubsection{Löschen}
Das Löschen von \term{MObject}-Exemplaren funktioniert wie auch sonst im System über die Klasse \term{MModelItem}.
Die Abhängigkeiten eines \term{MObject}s sind hierbei 
\begin{itemize}
\item \term{Links} die entweder mit \term{source} oder \term{target} auf das Objekt zeigen,
\item \term{Measurement}s und \term{Account}s
\item \term{Observation}s
\item und \term{NameInstance}s
\end{itemize}
Das Löschen eines Objekts ist zur Zeit nur möglich, wenn keine Exemplare dieser Klassen im System vorhanden sind, die von dem Objekt abhängen.

\subsubsection{Singletons und \qq{normale} Objekte}
Für die Unterscheidung von Exemplaren von Singleton-Typen und \qq{normalen} Objekte, wurden eine Vererbungshierarchie eingeführt, welche Objekte in \term{MObject} (\qq{normale} Objekte)
und \term{MSingletonObject} (Exemplare von Singleton Typen) partitioniert. Die dazu eingeführte Oberklasse \term{AbstractObject} abstrahiert die Gemeinsamkeiten der beiden Klassen. 
Durch die Partitionierung ist es möglich die unterschiedlichen geltenden Konsistenzbedingungen in den jeweiligen Klassen zu implementieren.
Dazu wurde die Types Assoziation in den konkreten Klassen hinterlegt, wobei SingletonExemplare nur einen Typen haben können.
Entsprechend sind auch die Operationen zum Hinzufügen, Ändern und Entfernen von Typen nur in der Klasse \term{MObject} zu finden.

Kritik:
Diese Designentscheidung ist kritisch zu betrachten. Dadurch, dass Objekte in der späteren Anwendung immer Exeplare entweder von MObject oder MSingeltonObject sind,
ist ein Ändern des Attributs \qq{singleton} an Exemplaren des Typs MAtomicTypes kritisch. Dies führt dazu, dass die Exemplare der Objekte weggeschmissen werden müssen.
Dies ist ins besondere bei Refactorings, die Singletons entfernen (wie z.B. beim Implemetieren von Mandantenfähigkeit) problematisch.
Es ist ratsam in einer späteren Version dieses Problem etwa mit einem \qq{State}-Pattern zu vermeiden. % Inhalt einbinden

% Kapitel
\newpage
\subsection{Observation}\label{Observations}
In der ursprünglichen Modellierung handelte es sich bei einer \term{Observation} um eine Assoziation bei der das Ziel auf einen 
Typ zeigte, der nur aus \term{Singletons} bestand. Damit dies aber möglich ist, müssen in der Typ-Hierarchie alle Knoten 
abstrakt und alle Blätter \term{Singletons} sein. Diese Bedingung muss aber für allgemeine Singletons nicht gelten. Um an dieser 
Stelle eine klare Trennung zu erreichen und die Transparenz des Modells zu erhöhen, wurden \term{Observation} als eigener Modellabschnitt ausgelagert.

\img[width=\relWidth{1}]{observation/observation.pdf}{Ausschnitt Observations}{img_observations}

\term{Observations} werden auf Modellebene durch die Klassen \term{Enum} und \term{ObservationType} repräsentiert. 
Zum Interpretieren der \term{Observations} dienen Enumerationen. Das bedeutet, dass jeder konkreten \term{Observation}
ein Wert einer Enumeration zugewiesen wird. Die Klasse \term{Enum} typisiert dabei die Enumeration. 
Im aktuellen Entwicklungsstand ist es nicht vorgesehen, \term{Enumerations} zu spezialisieren.

Ein Beispiel für einen \term{Enum}-Typ ist \qq{Blutgruppe}. Die Klasse \term{ObservationType} stellt auf Modellebene die 
Typisierung einer \term{Observation} dar. Dabei wurde festgelegt, dass jede \term{Observation} genau einer Enumeration 
und genau einem Typen zugeordnet sein muss. Es stellt sich die Frage, warum nicht mehrere Typen zugeordnet werden können. 
Der Grund dafür ist, dass dies bereits über komplexe Typen definiert und somit an dieser Stelle darauf verzichtet werden kann.

Auf der Exemplarebene werden \term{Observations} mit den Klassen \term{EnumValue} und \term{Observation} abgebildet. 
Die Klasse \term{EnumValue} spiegelt dabei einen konkreten Enumerationswert (z.B. \qq{Blutgruppe A}) wieder, 
wobei die Zuordnung zur Enumeration über die \term{theType}-Assoziation erfolgt. Die Klasse \term{Observation} definiert eine konkrete Observation. 
Sie ist typisiert in einem \term{ObservationType} und hat jeweils eine Zu-eins-Assoziation zu einem konkreten Enumerationswert sowie einem konkreten Objekt.

Es ergeben sich die folgenden Konsistenzbedingungen:
\begin{itemize}
  \item \term{Enum}-Objekte dürfen nur gelöscht werden wenn weder \term{EnumValue} noch \term{ObservationType} Objekte diese referenzieren.
  \item \term{ObservationType}-Objekte dürfen nur gelöscht werden, wenn es keine Ausprägungen auf der Exemplarebene gibt (\term{Observations}).
  \item Der Name von \term{EnumValue}-Objekten muss innerhalb des selben Typs (\term{Enum}) eindeutig sein. Somit soll verhindert werden, dass beispielsweise zweimal der Wert \qq{Blutgruppe A} einer Enumeration \qq{Blutgruppe} zugewiesen werden kann.
\end{itemize}

Für die \term{Observation}-Objekte müssen folgende Bedingungen gelten:
\begin{equation}\forall o \in Observation: o.theType.enumType = o.enumValue.theType
\end{equation}
\begin{equation}\forall o \in Observation: o.theObsObject.type \leq o.theType.theType
\end{equation}

Für die Erzeugung, das Löschen und die Verwaltung von \term{Observations} wurden die folgenden Manager definiert:
\begin{description}
\item[EnumerationManager] Erzeugen und Löschen von \term{Enumerations}.
\item[ObsTypeManager] Erzeugen und Löschen von \term{ObservationTypes}.
\item[ObservationManager] Erzeugen und Löschen von \term{Observations}.
\item[EnumValueManager] Erzeugen und Löschen von Enumerationswerten.
\end{description}
 % Inhalt einbinden

% Kapitel
\newpage
\subsection{Operations und Associations}

\subsubsection{Operationen}

Die Klasse Operation repräsentiert die Abbildung von Operationen in dem erarbeiteten Modell.
Wie in dem Modell aus dem digitalen Anhang ersichtlich, hat die Oberklasse AbstractOperation zwei  Assoziationen zu MType. 
Im Kontext von Operationen ist die Quelle (source) der Typ, welcher die Operation enthält. 
Das Ziel (target) einer Operation entspricht dem Rückgabetypen.
Neben Quell- und Zieltypen besitzt eine Operation eine Liste von Formalparametern, welche 
wiederum über die Klasse MType typisiert sind.

Die Klasse Operation enthält nur genau zwei Operationen.
Die Operation \textbf{isStatic()} gibt genau dann true zurück, wenn der Quelltyp dieser Operation der leeren Summe entspricht.
Zusätzlich zu der Bedingung aus isStatic gilt für die zweite Operation \textbf{isConstant()}, dass die Operation keine Parameter enthalten darf. 


\subsubsubsection{OperationManager}

Der OperationManager beinhaltet zwei Listen zur Verwaltung von Operationen und Parametern.

\begin{description}
\item[Operation** operations] Diese Liste enthält alle Operationen, die der Manager verwaltet.
\item[FormalParameter** formalParameters] Hier werden alle Formalparameter abgelegt, die der Manager verwaltet.
\end{description}

Um diese Listen verwalten zu können, stellt der Operationmanager 10 transaktionale Operationen bereit.
Im Folgenden werden diese Operationen kurz beschrieben, wobei sowohl \textbf{Parameter} als auch \emph{Exceptions} durch entsprechende Formatierungen hervorgehoben wurden.

\begin{description}
\item[createOperation]
Diese Operationen dient zum Erstellen einer neuen Operation. Um die Operation modellkonform erstellen zu können, sind vier Parameter notwendig:
Die ersten zwei Parameter legen fest, zu welchem Typen diese Operation erstellt werden soll \textbf{(source)} bzw. welcher Typ als Rückgabetyp dient \textbf{(target).}
Die Bezeichnung der Operation wird über den dritten Parameter \textbf{(name)} festgelegt. 
Zur Definition der Parameterliste der zu erstellenden Operation dient der vierte Parameter \textbf{(fp).}
Wie bei den meisten Klassen dieses Projekts, die eine Bezeichnung haben, dürfen auch Operationsnamen nicht mehrfach verwendet werden. Beim Versuch zwei Operationen mit identischem Namen anzulegen, wird eine \emph{DoubleDefinitionException} erzeugt und die Erstellung wird abgebrochen. Es ist außerdem nicht möglich, sowohl Quell- als auch Zieltyp der Operation mit der leeren Summe (EmptyTypeDisjunction) zu versehen. Wird ein solcher Versuch unternommen, mündet dies in der Erstellung einer \emph{ConsistencyException.} Wenn nur Quell- oder Zieltyp der leeren Summe entsprechen, existiert dazu eine semantisch sinnvolle Interpretation und stellt keine Ausnahme dar (siehe unten).
\item[createStaticOperation]
Diese Operation bietet dem Anwender eine komfortable Möglichkeit eine statische Operation zu erstellen. Dazu wird die Quell-Assoziation auf EmptyTypeDisjunction gesetzt und es ist nur noch \textbf{der Rückgabetyp, der Name und die Parameterliste} auswählbar. Durch die Delegation gelten exakt die gleichen Konsistenzbedingungen wie beim Erstellen eine Operation (createOperation).
\item[createConstant]
Mithilfe dieser Operation ist es dem Anwender möglich auf einfache Art eine Konstante zu erstellen. Da eine Konstante eine Operation ist, die eine leere Parameterliste hat und als Quelle auf EmptyTypeDisjunction verweist, sind das genau die Parameter die beim Aufruf nicht mehr angegeben werden müssen bzw. können. Auch hierbei handelt es sich um Delegation an die createOperation-Operation und es gelten genau die gleichen Konsistenzbedingungen.
\item[createVoidOperation]
Die createVoid-Operation ist die vorerst letzte Operation zum benutzerfreundlicheren Erstellen einer Operation. Anders als bei den zwei Operationen zuvor, sind hier Quelle und Parameterliste wählbar. Da die zu erstellende Operation aber keinen Rückgabetypen haben soll, ist das Ziel vordefiniert und kann vom Anwender beim Aufruf nicht angegeben werden. Genau wie zuvor gelten auch hier die Konsistenzbedingungen der createOperation-Operation.
\item[createFormalParamater]
Um Formalparameter für Operationen verwenden zu können, müssen diese zuvor erstellt werden. Genau hierfür wird dem Anwender die createFormalParameter-Operation angeboten.
Zum erfolgreichem Erstellen eines Formalparameters ist es notwendig, diesem einen \textbf{Namen} sowie einen \textbf{Typen} zu geben. Nach erfolgreicher Erstellung ist der erstellte Formalparameter in der zuvor beschriebenen Formalparameter-Liste zu finden. Die Erstellung kann entweder an der mehrfachen Verwendung des Formalparameter-Namens oder der Typisierung des Parameters in EmptyTypeDisjunktion scheitern. Im ersten Fall wird eine \emph{DoubleDefinitionException} erstellt, im Zweiten eine \emph{ConsistencyException} wobei beide Exceptions zum Abbruch führen.
\item[addFormalParameter]
Um Formalparameter nachträglich einer Operation zuzuweisen, benötigt diese Operation sowohl die \textbf{Operation} an sich als auch den hinzuzufügenden \textbf{Formalparameter.} Sollte dieser Formalparameter nicht schon in der Formalparameterliste der gewählten Operation sein, wird dieser hinzugefügt. Im anderen Fall wird eine \emph{DoubleDefinitionException} erstellt und der Vorgang wird abgebrochen.
\item[addMultipleFormalParameter]
Diese Operation dient dem Zweck eine Liste von Formalparametern der gewählten Operation zuzuweisen.
Zu diesem Zweck ist es notwenig die \textbf{Liste von Formalparametern} sowie 
die gewählte \textbf{Operation} an die addMultipleFormalparameter-Operation zu übergeben. 
Aufgrund von Delegation gelten auch hier die Konsistenzbedingung der addFormalParamter-Operation. 
\item[removeFormalParameterFromOperation]
Wenn ein Formalparameter aus der Parameterliste einer Operation entfernt werden soll, stellt diese Operation die notwendige Funktionalität bereit. Zum Entfernen eines \textbf{Formalparameters} aus einer \textbf{Operation} sind diese beiden Angaben essentiell. Sofern der zu entfernende Formalparameter in der Formalparameterliste der gewählten Operation ist und es keine Exemplare zu der Operation gibt, wird der Parameter aus der Parameterliste entfernt.
Sollte der Parameter nicht in der Parameterliste der Operation sein oder es Exemplare zu der Operation geben, wird eine \textbf{ConsistencyException} erzeugt. \newline
Anmerkung: Zum aktuellen Zeitpunkt kann der zweite Fall nicht eintreten, da sich Message-Objekte nicht an der Oberfläche erstellen lassen.
\item[removeFormalParameter]
Um einen Formalparameter vollständig aus dem System zu entfernen, benötigt diese Operation den \textbf{Formalparameter}. Sofern es keine Exemplare zu diesem Parameter gibt und keine Operation diesen Formalparameter in seiner Parameterliste hat, wird der gewählte Formalparameter aus dem System entfernt. Im anderen Fall wird eine \emph{ConsistencyException} erstellt und der Löschvorgang abgebrochen.
\item[removeOperation]
Mithilfe dieser Operation lassen sich Operationen aus dem System entfernen. Zur erfolgreichen Durchführung wird lediglich die zu entfernende \textbf{Operation} benötigt. Das einzige Abbruchkriterium dieser Operation ist die Existenz eines Exemplars zu der gewählten Operation. Da sich zum aktuellen Entwicklungszeitpunkt Messages nicht erzeugen lassen, kann dies nicht auftreten.

\end{description}

\subsubsection{Assoziationen}\label{Operation:Associationen}

Die Klasse Association repräsentiert die Möglichkeit Assoziationen zwischen zwei beliebigen Typen abzubilden.
Neben dem Quell- und Zieltypen, kann eine Assoziation an beliebig vielen Hierarchien teilnehmen.
Wie bei vielen anderen Modellbestandteilen, sind auch die Namen der Assoziation indiziert. Daraus folgt auch, dass es keine gleichnamigen 
Assoziationen geben darf.

Wie im digitalen Anhang dieser Dokumentation zu sehen ist, können sowohl Operationen als auch Assoziationen Formalparameter beinhalten. Die Möglichkeit der Zuweisung von
Formalparametern an Assoziationen, wird dem Anwender über die Oberfläche nicht bereitgestellt. 
Formalparameter für Assoziationen können als n-stellige Assoziationen interpretiert werden.
Die Möglichkeit der Zuweisung von Formalparametern an Assoziationen wird dem Anwender über die Oberfläche nicht zur Verfügung gestellt, da sie nicht Teil der ersten Implementierungsstufe sind.

\subsubsubsection{AssociationManager}

Der AssoziationManager bietet sechs transaktionale Operationen an und beinhaltet zwei Listen, welche er verwaltet.
\begin{description}
\item[Association ** associations] In dieser Liste werden alle erstellten Assoziationen aufbewahrt und durch entsprechende Operationen verwaltet.
\item[Hierarchy ** hierarchies] Alle erstellten Hierarchien werden in dieser Liste abgelegt.
\end{description}
Diese beiden Listen werden dem Anwender an der 
Oberfläche präsentiert. Über entsprechende Operationen kann der Anwender die Elemente dieser Listen verwalten.

Wie schon bei dem OperationManager, werden \textbf{Parameter} und \emph{Exceptions} durch entsprechende Formatierungen kenntlich gemacht.

\begin{description}
\item[createAssociation] Bei Aufruf dieser Operation müssen \textbf{Quell- und Zieltyp} sowie der \textbf{Name} der zu erstellenden Assoziation 
angegeben werden.
Sollte der Name schon von einer anderen Assoziation verwendet werden, erzeugt die Operation eine \emph{DoubleDefinitionException} und die Assoziation wird nicht
erstellt. Außerdem darf eine Assoziation nie in der leeren Disjunktion als Quelle oder Ziel typisiert sein. In einem solchen Fall
wird eine \emph{ConsistencyException} erstellt.
Wenn keine der genannten Exceptions aufgetreten ist, wird die Assoziation erstellt und zur Liste der bekannten Assoziationen hingefügt. 
\item[removeAssociation] Diese Operation erwartet beim Aufruf nur die zu entfernende \textbf{Assoziation}. 
Bei erfolgreicher Durchführung dieser Operation, wird die
angegebene Assoziation aus dem System entfernt.
Sollte es ein oder mehrere Exemplare (Links) zu dieser Assoziation geben oder sich die zu löschende Assoziation in mindestens einer Hierarchie befinden, wird eine \emph{ConsistencyException} erzeugt und die Assoziation wird nicht entfernt. 
\item[createHierarchy] Da es in dem erarbeiteten Modell keine leeren Hierarchien geben darf, wird neben dem Namen der zu erstellenden \textbf{Hierarchie} auch eine 
\textbf{Assoziation} erwartet. Nach erfolgreicher Durchführung, wird die Hierarchie erstellt und die angegebene Assoziation der neuen Hierarchie zugeordnet.
Insbesondere folgende Gründe führen zum Misserfolg:
\begin{itemize}
\item Es exisitiert bereits eine Hierarchie mit diesem Namen. Das hat zur Folge, dass eine \emph{DoubleDefinitionException} erzeugt wird.
\item Es existiert auf der Exemplarebene ein Zyklus. Wie in der Beschreibung zu addAssociation erläutert, ist dieser Umstand für Hierarchien untersagt.
 Es resultiert eine \emph{CycleException.}
\end{itemize}
\item[removeHierarchy] Nach erfolgreicher Ausführung dieser Operation, wird die übergebene \textbf{Hierarchie} aus dem System entfernt. 
Es wird lediglich die Hierarchie entfernt. Die Assoziationen, welche an der gelöschten Hierarchie teilnehmen, bleiben weiterhin bestehen.
\item[addAssociation] Diese Operation erwartet zum einen die \textbf{Hierarchie} zu der die Assoziation hinzugefügt werden soll und zum anderen die \textbf{Assoziation} selbst.
Wenn es auf der Exemplarebene keine zyklischen Links zu dieser Assoziation gibt, wird die Assoziation der Hierarchie zugeordnet. Sollte es zyklische Links
zu dieser Assoziation geben oder die Assoziation ist bereits in dieser Hierarchie, werden entsprechende Exceptions (\emph{CycleException,} bzw. \emph{DoubleDefinitionException}) 
erstellt und die Assoziation wird der Hierarchie nicht zugeordnet.
\item[removeAssoFrmHier] Zum Entfernen von Assoziationen aus Hierarchien, ist es notwendig dieser Operation sowohl die \textbf{Hierarchie}, als auch die zu entfernende 
\textbf{Assoziation} anzugeben. Dabei wird lediglich die Assoziation aus der Hierarchie entfernt, die Assoziation an sich bleibt bestehen. 
Sofern die angegebene Assoziation an dieser Hierarchie teilnimmt, wird diese Verbindung durch die Ausführung dieser Operation entfernt. 
Sollte dies nicht der Fall sein, wird eine \emph{NotAvailableException} geworfen. Einen Sonderfall bildet die letzte Assoziation einer
Hierarchie. Diese darf gemäß des Modells nicht entfernt werden, da leere Hierarchien nicht zulässig sind. Beim Versuch die letzte
Assoziation einer Hierarchie aus der Hierarchie zu entfernen, wird eine \emph{ConsistencyException} erzeugt und der Versuch wird abgebrochen. 
\end{description} % Inhalt einbinden

% Kapitel
\newpage
\subsection{Quantity}
Die Klasse Quantity dient zur Angabe von einheitenbehafteten Werten. Jede Einheit ist dabei in einem Einheitentyp typisiert. Bei den Quantitäten gibt es atomare, die nur einen Wert mit einer Einheit besitzen, und zusammengesetzte, die einen Vektor atomarer Quantitäten mit unterschiedlichen Einheiten darstellen, wobei die Einheiten alle im selben Einheitentyp typisiert sein müssen.

\subsubsection{Die Manager}

\subsubsubsection{UnitTypeManager}

Der UnitTypeManager beinhaltet zwei auf der Oberfläche sichtbare und zwei nicht sichtbare Listen zur Verwaltung von Einheiten (Units) und Einheitentypen (UnitTypes).

\begin{description}
\item[AbsUnitType** unitTypes] Diese Liste enthält alle Einheitentypen, die der Manager verwaltet.
\item[AbsUnit ** units] Hier werden alle Einheiten abgelegt, die der Manager verwaltet.
\item[ReferenceType** refTypes] Diese Liste enthält alle Referenz-Typen, die der Manager verwaltet. Sie ist nicht sichtbar und dient nur zur internen Verarbeitung von zusammengesetzten Einheitentypen (CompoundUnitTypes)
\item[Reference** refs] Hier werden alle Referenzen abgelegt, die der Manager verwaltet. Sie ist nicht sichtbar und dient nur zur internen Verarbeitung von zusammengesetzten Einheiten (CompoundUnits)
\end{description}

\subsubsubsection{FractionManager}\label{FractionManager}

Der FractionManager verwaltet alle Brüche so, dass jede Rationale Zahl als Bruch nur einmal durchgekürzt (als \emph{Repräsentant}) vorkommt.
Über die Operation \textit{getFraction(key:String):Fraction} wird auf einen vorhandenen Bruch zugegriffen, wobei der Schlüssel die \textit{toString-Ausgabe} des Bruchs ist.
Die Operation \textit{addFraction} wird ein neuer Bruch publiziert, wenn er erstmalig benötigt wird. Die Operation \textit{invertSign(f:Fraction):Fraction} sucht das Negativkomplement zu der Zahl und erstellt es, falls nötig.

GOJA unterstütz keine 3-stelligen Assoziationen bei denen der Wert eines Wert-Schlüssel-Paares ein in GOJA primitiver Typ ist. Deswegen wurde die Klasse \textit{FractionWrapper} erstellt, die den Wert des Bruchs umhüllt. Somit wird das Problem umgangen: \texttt{FractionWrapper **String** managedFractions;}


\subsubsubsection{QuantityManager}\label{QuantityManager}

Der QuantityManager verwaltet die Zugriffe auf Quantitäten und ist die Schnittstelle zum Erstellen neuer Exemplare.
Zudem wird der FractionManager als zentrale Fassade für das arithmetische und logische Operieren auf Quantitäten verwendet (siehe dazu Abschnitt \ref{Berechnungen}).

Beim Erstellen einer atomaren Quantität referenziert die \textit{amount}-Assoziation ein Fraction-Objekt aus dem FractionManager. Wird beim Anlegen der Quantität festgestellt, dass es diesen Bruch noch nicht gibt,
wird sie neu erstellt und im FractionManager publiziert. 

\subsubsection{Einheitentypen}
Die Einheitentypen (UnitTypes) geben an, welche verschiedenen Einheiten fachlich zusammen gehören. Beispielsweise sind Dollar und Euro beides Währungen und können damit zusammengerechnet werden. In diesem Fall ist der Einheitentyp Währung und die beiden zugehörigen Einheiten sind Dollar und Euro. Bei Euro und Kilometer hingegen macht es wenig Sinn, diese zu Addieren oder zu Subtrahieren, denn das eine ist eine Währung und das andere eine Strecken-Angabe. Die Multiplikation und Division hingegen können auch auf unterschiedlichen Einheitentypen geschehen, wodurch dann zusammengesetzte Einheitentypen (CompoundUnitTypes) wie Geschwindigkeit (Strecke/Zeit) entstehen.

Zur Verwaltung von UnitTypes und CompoundUnitTypes stellt der UnitTypeManager 8 transaktionale Operationen bereit. Folgende dieser Operationen sind direkt über die Oberfläche erreichbar:

\begin{description}
\item[createUnitType]
Diese Operation dient zum Erstellen eines neuen atomaren Einheitentypen. Da jeder Einheitentyp einen eigenen, eindeutigen Namen besitzen muss, wird dieser als Parameter der Operation übergeben. Sollte es schon einen Einheitentyp, egal ob atomar oder zusammengesetzt, mit diesem Namen geben, gibt es eine DoubleDefinitionException.
\item[fetchScalarType]
Liefert den skalaren Einheitentyp, welcher aus einem zusammengesetzten Einheitentypen ohne Referenzen auf atomare Typen besteht. 
\item[changeUTName]
Mit dieser Operation kann der Name eines Einheitentypen geändert werden. Ebenso wie beim neu erstellen, gibt es auch hier eine DoubleDefinitionException, wenn schon ein Einheitentyp mit diesem Namen existiert.
\item[addReferenceType]
Mit dieser Operation kann einem Einheitentypen, egal ob zusammengesetzt oder atomar, eine Referenz auf einen atomaren Einheitentypen hinzugefügt werden. Hierdurch entsteht dann ein zusammengesetzter Einheitentyp. Zusätzlich zu dem zu referenzierenden Einheitentypen muss noch der Exponent der Referenz und der Name des neu entstehenden zusammengesetzten Einheitentypen übergeben werden. Wie bei den anderen beiden Methoden gilt auch hier: ist der Name schon vorhanden, gibt es eine DoubleDefinitionException. Existiert schon ein Einheitentyp mit genau den selben Referenzen, wird kein neuer Einheitentyp erstellt, sondern der vorhandene zurückgegeben.
\item[setDefaultUnit]
Diese Methode weist einem atomaren Einheitentypen eine Standard-Einheit zu, die für Umrechnungen in andere Einheiten benötigt wird. Die Umrechnungen werden im Kapitel \ref{ConversionsKapitel} genauer erläutert.
\end{description}

Folgende Operationen sind für die interne Verarbeitung relevant und können nicht über die Oberfläche aufgerufen werden:

\begin{description}
\item[getExistingCUT]
Diese Operation ermittelt an Hand einer Liste von Referenzen, ob schon ein zusammengesetzter Einheitentyp mit genau diesen Referenzen existiert, und, wenn vorhanden, wird er zurückgegeben. 
\item[fetchCUT]
Diese Methode prüft zunächst an Hand einer Liste von Referenzen, ob schon ein Einheitentyp mit diesen existiert (dazu wird getExistingCUT verwendet) und legt, wenn nicht vorhanden, diesen neu an. Anschließend wird der Einheitentyp, der durch die übergebenen Referenzen definiert ist, zurückgegeben. Dafür muss auch ein Name übergeben werden, der jedoch nur beim Neu erstellen Verwendung findet. Auch hier gilt: Ist der Name schon vorhanden, gibt es eine DoubleDefinitionException. Diese Methode dient der Vermeidung von doppelt angelegten Einheitentypen.
\item[fetchReferenceType]
Diese Operation ermittelt, ob es schon eine Referenz auf einen Einheitentyp mit einem bestimmten Exponenten gibt. Wenn ja, wird sie zurückgegeben, wenn nicht, wird eine neue erstellt. Diese Methode dient der Vermeidung von doppelt angelegten Referenzen auf Einheitentypen.
\end{description}


\subsubsection{Einheiten}
Fachlich gesehen können Einheiten (Units) von Quantitäten angenommen werden und sind in Einheitentypen typisiert.
Dabei wird zwischen atomaren Einheiten (z.B. Meter [m]) und zusammengesetzten Einheiten (z.B. Kilometer pro Stunde [km/h]) unterschieden.

\subsubsubsection{Objektmodell}

Die Zusammenhänge zwischen Units und UnitTypes sollen an dieser Stelle anhand eines beispielhaften Objektmodells in \refImg{unit_unittypes_objektmodell} verdeutlicht werden.

\img[width=\relWidth{1}]{quantity/unit_unittypes_objektmodell.png}{Objektmodell: Geschwindigkeit und Beschleunigung}{unit_unittypes_objektmodell}

Auf der Typebene gibt es die beiden atomaren UnitTypes \textit{Strecke} und \textit{Zeit}, außerdem zwei CompoundUnitTypes \textit{Geschwindigkeit} und \textit{Beschleunigung}. \textit{Geschwindigkeit} hat zwei Referenzen (RefTypes), eine mit dem Exponent 1 auf \textit{Strecke} und eine mit dem Exponenten -1 auf \textit{Zeit}. \textit{Beschleunigung} hat ebenfalls eine Referenz mit dem Exponenten 1 auf \textit{Strecke}. Hierbei wird dasselbe RefType-Objekt verwendet, welches auch \textit{Geschwindigkeit} nutzt. Außerdem besitzt \textit{Beschleunigung} noch eine Referenz auf \textit{Zeit}. Da diese jedoch den Exponenten -2 hat, wird hierbei ein anderes RefType-Objekt verwendet als bei \textit{Geschwindigkeit}.

Auf der Exemplarebene gibt es die atomaren Units \textit{km}, typisiert in \textit{Strecke}, und \textit{h} und \textit{s}, beide typisiert in \textit{Zeit}. Außerdem gibt es die CompoundUnits \textit{km/h}, welche als Einheitentyp \textit{Geschwindigkeit} hat, und \textit{km/(s*h)}, welche zum Einheitentyp \textit{Beschleunigung} gehört. Auf \textit{km} gibt es eine Referenz mit Exponent 1, welche sowohl von \textit{km/h} als auch von \textit{km/(s*h)} verwendet wird. Die Referenz mit Exponent -1 auf \textit{h} wird ebenfalls von \textit{km/h} und \textit{km/(s*h)} benutzt. Außerdem geht von \textit{km/(s*h)} aus noch eine Referenz mit Exponenten -1 auf \textit{s}. Somit gibt es auf der Exemplarebene von der \textit{Beschleunigungs}-Einheit zwei Referenzen auf \textit{Zeit}-Einheiten. Summiert man jedoch ihre Exponenten, erhält man den auf der Typebene geforderten Exponenten -2.

\subsubsubsection{Verwalten von Units}

Für die Verwaltung von Units und CompoundUnits stellt der UnitTypeManager acht transaktionale Operationen bereit.
Folgende dieser Operationen sind direkt über die Oberfläche erreichbar:

\begin{description}
\item[createUnit]
Diese Operatione dient zum Erstellen einer neuen Unit. Da jede Unit fachlich in einem UnitType typisiert werden und einen Namen haben muss, können dieser Methode diese Werte entsprechend übergeben werden. Eine DoubleDefinitionException wird geworfen, wenn eine Unit mit dem gewählten Namen bereits existiert.
\item[changeUName]
Diese Operation dient zum Umbenennen einer Unit. Auch hier wird die DoubleDefinitionException im doppelten Namensfall geworfen.
\item[fetchScalar]
Liefert die eine CompountUnit, die keine Referenzen zu anderen Units aufweist.
\item[addReference]
Diese Operation kann sowohl auf zusammengesetzten Einheiten, als auch auf atomaren Einheiten angewendet werden. Sie dient zum erstellen von CompoundUnits. Analog zur createUnit-Methode wird hier ein Name benötigt. Die entsprechende neue Referenz zur ausgewählten Unit wird durch einen Exponenten definiert. Die DoubleDefinitionException wird auch hier geworfen, falls es zu Namenskonflikten kommt.
\item[setConversion]
Mittels dieser Operation lässt sich für eine Unit eine Conversion zur aktuellen DefaultUnit des entsprechenden UnitTypes angeben. Die Umrechnungsrate besteht immer aus einem Faktor und einer Konstante. Die Conversions werden im Kapitel \ref{ConversionsKapitel} näher erläutert.
\end{description}

Folgende Operationen sind für die interne Verarbeitung relevant und können nicht über die Oberfläche aufgerufen werden:
\begin{description}

\item[getExistingCU]
Hier wird anhand einer Liste von vorhandenen Referenzen eine CompoundUnit ermittelt, welche durch genau diese Referenzen definiert ist. Sollte diese Unit noch nicht existieren, wird null zurückgeliefert. Diese Operation dient zum vermeiden von doppelt angelegten CompoundUnits.
\item[fetchCU]
Diese Operation ist ähnlich der getExistingCU()-operation. jedoch wird hierbei die CompoundUnit angelegt, falls sie noch nicht existiert. Die Angabe eines Namens ist erforderlich, falls eine neue CompoundUnit zustande kommen sollte. Auch hier wird entsprechend eine DoubleDefinitionException geworfen, falls eine Unit mit dem gewählten Namen bereits existiert.
\item[fetchReference]
Mithilfe dieser Operation kann eine Referenz-Instanz mit einem gewissen Exponenten auf eine bestimmte Unit ermittelt werden. Falls diese Instanz noch nicht existierte, wird sie erzeugt. Das dient zum vermeiden von Doppelt anlegten Referenzen.
\end{description}

\subsubsubsection{Conversions}\label{ConversionsKapitel} 

Eine Conversion, also die Umrechnung von einer Unit zur entsprechend zum UnitType gehörigen DefaultUnit, kann über zwei Wege zustande kommen, bzw. verändert werden: Zum einen über die setConversion-Operation und zum anderen über die setDefaultUnit-Operation.
Conversions sind immer lineare Funktionen, damit eine Umkehrbarkeit gewährleistet ist. Jede Conversion enthält also einen Faktor und eine Konstante (als Fraction), um der linearen Funktion m*x+b gerecht zu weden. Die Angabe von m und b hat folgende Semantik: y DefaultUnit = m * unitWert + b.
Am Beispiel von Fahrenheit und Celsius würde für Celsius die Conversion folgendermaßen lauten, wenn Fahrenheit die DefaultUnit darstellt:

\begin{equation} Wert_{^\circ F} = Wert_{^\circ C} * \frac{9}{5} + 32
\end{equation}

Mittels setConversion() wird eine bereits gesetzte Conversion für eine Unit überschrieben.
Bei setDefaultUnit() wird eine Umrechnung von bereits vorhanden Conversions notwendig, da diese ja in Abhängigkeit zu einer nun veralteten DefaultUnit angegeben wurden. Das betrifft alle Umrechnungen für Units zum selben UnitType wie die DefaultUnit.
\refImg{setDefaultUnitGrafik} beschreibt die Umrechnungen, falls sich eine DefaultUnit ändert.
\img[width=\relWidth{0.8}]{quantity/setDefaultUnit.png}{Umrechnung beim Ändern einer DefaultUnit}{setDefaultUnitGrafik}
Im gezeigten Beispiel gibt es drei Einheiten a, b und c, wobei zunächst a die DefaultUnit zum gemeinsamen UnitType darstellt (oberer Bereich). Die Defaultunit kann nur den Umrechnungsfaktor 1 haben, da die Funktion immer umkehrbar ist (1a=1a). Für den unteren Bereich der Abbildung wird dann die DefaultUnit auf b gesetzt. Die jeweils neuen Conversions ergeben sich, indem die jeweilige Funktion der Unit mit der noch aktuellen Funktion der neuen DefaultUnit (also b) gleich gesetzt wird und mittels Äquivalenzumformung zur neuen DefaultUnit hin umgestellt wird (rechter Bereich). Für die neue DefaultUnit ist danach der Umrechnungsfaktor dann entsprechend wieder 1.

\subsubsection{Quantitäten}\label{Quantitaeten}

\subsubsubsection{Einfache und zusammengesetzte Quantitäten}

Quantitäten können sowohl atomar als auch zusammengesetzt sein.
Dies wird definiert die parts-Assoziation zwischen \term{CompoundQuantity} und \term{Quantity}. 
Dabei kann man zusammengesetzte Quantitäten als Vektoren auffassen, deren Elemente einfache Quantitäten sind. 
\emph{In einem Vektor dürfen nur solche Quantiäten enthalten sein, die mit dem gleichen UnitType assoziiert sind}. 

\subsubsubsection{Berechnungen und Vergleiche}\label{Berechnungen}

Die Addition, Subtraktion, Division und Multiplikation sind als Grundrechenarten (BasicCalculation) im Klassenmodell dargestellt. Man unterscheidet zwischen Grundrechenarten, die die Referenzen der Einheit verändern, und solchen, die dies nicht tun. Z.B. ergibt die Addition zweier Längenangaben (z.B. Meter) wieder eine Längenangabe. Also nennen wir eine solche Rechnung \emph{einheitstypbewahrend}. Im Klassenmodell ist diese Generalisierung als die Klasse \textit{UnitImutabCalc} zu erkennen. Im Gegensatz dazu ergibt z.B. die Multiplikation zweier Längenangaben eine Flächenangabe (z.B. Quadratmeter). Wir nennen diese Rechnungen \emph{einheitstypverändernd} - im Klassenmodell als \textit{UnitMutabCalc} zu sehen. Diese Implementierung besitzt Ansätze eines \emph{Kommando-Entwurfsmusters}.

\subsubsubsection{Berechnungslogik}

Werden zweistellige Operationen zwischen AbsQuantity-Exemplaren ausgeführt, treten abhängig vom Typ der Quantität 3 Fälle auf:
\begin{enumerate}
\item \textit{op(vektor, vektor)}
\item \textit{op(vektor, element)}
\item \textit{op(element, element)}
\end{enumerate}

Die Methode der Operation \textit{calculate} der Klasse \textit{BasicCalculation} ermittelt mit Hilfe von \emph{Visitoren} die auszuführende Logik und ruft nach dem Template-Entwurfsmuster die entsprechende Methode auf, die in der zweiten Stufe der Klassenhierarchie implementiert werden, nämlich in \textit{UnitImutabCalc} und \textit{UnitMutabCalc}:
\begin{itemize}
\item \textit{calc1Compound1Atomar}
\item \textit{calcAtomar}
\item \textit{calcComp}
\end{itemize}

Das Rechnen mit zusammengesetzten Quantitäten folgt den allgemeinen Gesetzten der Vektorenrechnung.

Die Addition und Subtraktion funktioniert nur mit \emph{einheitstypgleichen} Quantitäten. Dagegen funktioniert die Multiplikation und Division mit allen Quantitäten, vorausgesetzt, es wird eine \emph{Zieleinheit}, bzw. ein \emph{Zieleinheitstyp} gefunden. Ist dies nicht der Fall wird eine \emph{Ausnahme} ausgelöst. Die Signatur der Klasse 
\textit{UnitMutabCalc} ist bereits darauf vorbereitet, automatisch Einheiten und Typen zu generieren, sollte noch keine Zieleinheit als \emph{Unit-Stammsatz} angelegt sein. Zum Zeitpunkt der Fertigstellung dieser Dokumentation ist dieser Automatismus noch nicht implementiert.

Der einzige bisher implementierte logische Verglich \emph{isLessOrEqual} folgt genau dem gleichen Schema, ist jedoch nur als eine Klasse \textit{LessOrEqualComparison} realisiert.
 % Inhalt einbinden

% Kapitel
\newpage
\subsection{Measurement}

\blindtext % Inhalt einbinden % Inhalt einbinden

% Kapitel
\section{Ausblick}

\TODO[Ideen Formulieren]


\begin{description}
  \item[Multiplizitäten] Zurzeit besitzen Assoziationen keine Einschränkungen. 
Es werden sozusagen überall $0..n$-Multiplizitäten verwendet. 
Mit der Einführung von Multiplizitäten können Assoziationen eingeschränkt werden.
 
Es bietet sich beispielsweise an, eine Klasse Multiplicity zwischen \term{O++A} und \term{Type} zu hängen,
welche zwei boolsche Attribute $\leq$ und $\geq$ enthält. 
Damit lassen sich die vier Multiplizitäten $0..n$, $1..n$, $1$ und $0..1$ darstellen.
Dies zieht u.a. einige Konsistenzbedingungen nach sich, welche beim Anlegen von Links beachtet werden müssen. 
  \item[Mehrstellige Assoziationen] Aktuell gibt es für eine Assoziation nur genau ein Quell- und genau ein Zieltypen. Um Mehrstellige Assoziationen und Links zu implementieren, könnten die Formal- und Aktualparameter verwendet werden. Eine Assoziation bestünde dann aus den zwei explizit genannten Quell- und Zieltypen sowie einer Liste von Formalparametern wobei jeder Formalparameter eine weitere Quelle darstellen würde. Um dies adäquat auf der Exemplarebene abzubilden, würden die Links sich ihrer Aktualparameter bedienen. Das aktuelle \MM sieht diese Implementierung schon vor, da sowohl Formal- als auch Aktualparameter von den jeweiligen Oberklassen AbstractOperation und Message++Link ausgehen.
  \item[Messages] Wie schon im Kapitel \ref{Message:Message} erläutert wurden Messages in der ersten Implementierungsstufe nicht umgesetzt. Zum gegenwärtigen Zeitpunkt existiert für ein Message-Exemplar sowohl die Interpretation des Nachrichtenaufrufs, als auch des Nachrichteninhalts.
Bevor Messages umgesetzt werden können muss sich geeinigt werden, welche der genannten Interpretationen für ein Message-Exemplars zutrifft. 
  \item[ActualParameter] Da aus genannten Gründen weder Message-Exemplare erzeugt, noch mehrstellige Assoziationen/Links angelegt werden können, haben Aktualparameter derzeit keinerlei Daseinsberechtigung. Aus diesen Gründen wird die Verwaltung von Aktualparametern dem Anwender nicht angeboten.
  \item[Path]
  \item[Posting Rules]
  \item[Versionierung]
  \item[Zentrale Ablage der Constraints]
\end{description}
   % Inhalt einbinden

% Abbildungsverzeichnis
\newpage
\resetPageTitle % Wichtig: Seitentitel wieder zurücksetzen, da er auf der letzten Seite geändert wurde
\listoffigures % Abbildungsverzeichnis einfügen

% Tabellenverzeichnis
%\newpage
%\listoftables % Tabellenverzeichnis einfügen

% Abkürzungsverzeichnis
\newpage
\renewcommand*{\glspostdescription}{} % Text am Ende eines Verzeichniseintrags (Standard: ".")
\printglossary[type=\acronymtype,style=super,nonumberlist] % Abkürzungen einbinden

% Glossar
\newpage
\renewcommand*{\glspostdescription}{~--} % Text am Ende eines Glossareintrags (Standard: ".")
\printglossary[style=altlist,nonumberlist=false] % Glossar ausgeben

% Quellenverzeichnis
\newpage
\setPageTitle{Quellenverzeichnis} % Seitentitel ändern, da diese Section ausgeblendet ist
\section*{Quellenverzeichnis}
\addcontentsline{toc}{section}{Quellenverzeichnis} % Fügt diesen Punkt ins Inhaltsverzeichnis ein

% Formatieren des Quellenverzeichnisses (alphadin = Formatierung nach DIN 1505)
\bibliographystyleLiteratur{alphadin} % Literatur formatieren
\bibliographystyleRecht{alphadin} % Rechtsvorschriften formatieren
\bibliographystyleInternet{alphadin} % Internetquellen formatieren
\bibliographystyleUnternehmen{alphadin} % Unternehmensquellen formatieren

\bibliographyLiteratur{\content/lib/Literatur} % Einfügen der Literatur
\bibliographyRecht{\content/lib/Recht} % Einfügen der Rechtsvorschriften
\bibliographyInternet{\content/lib/Internet} % Einfügen der Internetquellen
\bibliographyUnternehmen{\content/lib/Unternehmen} % Einfügen der Unternehmensquellen % Inhalt einbinden

% Beginn des Anhangs
%\newpage
%\resetPageTitle % Wichtig: Seitentitel wieder zurücksetzen, da er auf der letzten Seite geändert wurde
%\appendix % Beginn des Anhangs (diese Zeile auskommentieren, falls kein Anhang vorhanden ist)
\appendixtoc % Schwarze Magie - niemals auskommentieren
%% Kapitel einbinden
\section{Beispielanhang}

\blindtext

% Kapitel einbinden
\section{Weiterer Anhang}

\blindtext % Inhalt einbinden

\end{document}