\subsection{ModelItem}

\term{ModelItem} ist als Abstraktion für alle Modellelemente konzeptioniert. 
Zentrale Aufgabe dieser Klasse ist es, einen Abhängigkeitsgraphen über Modellelemente erstellen zu können und 
das Löschen dieser zu vereinheitlichen. Der Abhängigkeitsgraph ist, durch die Modellierung bedingt, azyklisch.
Die in der Implementierung angewandte Pragmatik verbietet das Löschen von Modellelementen, von denen weitere Elemente 
abhängig sind. Der Entwurf ermöglicht es auch nachträglich dieses Vorgehen mit geringem Aufwand durch ein kaskadierendes Löschen zu 
ersetzen. 
 
\img[width=\relWidth{0.5}]{modelItem/modelItem.pdf}{Klasse ModelItem}{img_modelItem}

\subsubsection{Operationen}

\begin{description}
\item[fetchDependentItems()] 
liefert alle Modellelemente zurück, die vom Element auf dem sie aufgerufen wird, direkt abhängig sind.
Ein Element \term{A} ist von einem Element \term{B} abhängig, wenn \term{A} eine Referenz auf \term{B} hält, also z.b.   
\term{A} in \term{B} klassifiziert ist oder u.a. aus einem \term{B}-Objekt besteht.
 
\item[delete()] vereinheitlich das Löschen von ModelItems. For das Standardverhalten stellt ModelItem eine \term{Template-Method}  
bereit. Es wird zuerst \term{fetchDependentItems()} aufgerufen und geprüft ob abhängige Modellelemente 
existieren. Werden Abhänigkeiten gefunden, so wird eine \term{ConsistencyException} ausgelöst. Andernfalls wird anschließend 
die abstrakte Operation \term{prepareForDeletion()} aufgerufen. Lässt sich diese ohne Ausnahme ausführen wird das Modellelement 
abschließend durch den Aufruf von \term{getThis().delete\$ME} in der Datenbank als gelöscht markiert, sodass es durch \term {indexed}-Zugriffe 
und \term{invere-Getter} nicht mehr erreicht werden kann. Es muss sichergestellt werden, dass nach dem Aufruf von \term{delete()} keine 
keine Referenzen von anderen Objekten mehr existieren, damit das JAVA-Objekt vom Garbage-Collector erfasst werden kann. 



\item[prepareForDeletion()] ist als Analogon zu \term{initializeOnCreation()} entworfen worden und ermöglicht ein 
\qq{Aufräumen} vor dem tatsächlichen Löschen. Ein Singleton-Type erstellt z.B. im \term{initializeOnCreation()} sein zugehöriges 
Singleton-Objekt und löscht dies in \term{prepareForDeletion()}. 
\end{description}
