\subsection{Message und Link}\label{Message:Message}

\term{Messages} und \term{Links} haben eine gemeinsame Oberklasse \term{MessageOrLink} (siehe \refImg{img_operation}). 
In der ersten Ausbaustufe werden lediglich die \term{Links} implementiert. Zu den \term{Messages} existiert eine rudimentäre
Implementierung die dem Anwender aber über die Oberfläche nicht bereitgestellt wird, da die Semantik eines \term{Message}-Exemplars bislang nicht eindeutig geklärt ist.
Es existiert zum einen die Interpretation, dass \term{Messages} Aufrufe von Operationen sind. 
Eine weitere Interpretationsmöglichkeit ist, dass eine \term{Message} die Methode an sich inklusive ihres Inhalts beschreibt. 
Dazu müsste aber ein domänenspezifische Sprache (DSL) entwickelt und im Modell verwendet werden.

Da \term{Messages} in dieser Ausbaustufe nicht enthalten sind, verlieren auch Actualparameter ihre Daseinsberechtigung. 
Aus diesem Grund werden auch diese nicht implementiert.

Ein \term{Link} ist eine konkrete Ausprägung einer Assoziation. Folglich ist die Erstellung eines \term{Links} nur möglich, 
wenn der zu erstellende \term{Link} sich an die aus der Assoziation resultierenden Konsistenzbedingungen hält. 
Für \term{Links} stehen dem Anwender zwei Operationen zur Verfügung.

\begin{description}
\item[createLink]
Wie zuvor erwähnt, ist es für einen Link zwingend notwendig sich an die Konsistenzbedingungen der Assoziation zu halten. Um einen Link aufbauen zu können muss die zugrundeliegende  \textbf{Assoziation} in welcher der Link typisiert sein soll, sowie das \textbf{Quell-} und \textbf{Zielobjekt} angegeben werden.
Bei der Erstellung eines Links wird eine \term{ConsistencyException} erzeugt, wenn die Quell- und/oder Zielobjekte gemäß der Typebene nicht 
gestattet sind.
\begin{equation} \forall \, l \in Link: l.source.type.isLessOrEqual(l.type.source)
\end{equation} 
\begin{equation} \forall \, l \in Link: l.target.type.isLessOrEqual(l.type.target)
\end{equation}
Des Weiteren ist es Assoziationen möglich an mehreren Hierarchien teilzunehmen. Zyklische Assoziationen sind auf der Modellebene möglich, 
da sich sonst keinerlei Aggregationsbeziehungen abbilden lassen.
Auf der Ebene der Exemplare hingegen, müssen \term{alle Links, die in Assoziation typisiert sind, in jeder Hierarchie an der diese teilnehmen, zyklenfrei sein.}
Wenn ein Anwender versucht zyklische Links zu erzeugen, wird eine \term{CycleException} erzeugt und der jeweilige Link wird nicht erstellt.
\item[removeLink]
Beim Entfernen eines bestehenden Links gibt es keinerlei Restriktionen zu beachten. Nach Wahl des zu löschenden \textbf{Links} kann dieser ohne weitere Prüfung entfernt werden.
\end{description}