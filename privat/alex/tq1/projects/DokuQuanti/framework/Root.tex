% Benutzerdefinierte Konstanten einbinden
%%%%%%%%%%%%%%%%%%%%%%%%%%%%%%%%%%%%%%%%%%%%%%%%
%        BENUTZERDEFINIERTE KONSTANTEN         %
%%%%%%%%%%%%%%%%%%%%%%%%%%%%%%%%%%%%%%%%%%%%%%%%

% Konstante: Pfad zum Bilderverzeichnis
\newcommand{\imgPath}{\content/images/}

% Konstante: Farbe der Links im PDF-Dokument
% Standard für Datei: blue
% Standard für Druck: black
\newcommand{\linkColor}{blue}

% Konstante: Datum der Abgabe
% Wird auf der Titelseite und in der
% Ehrenwörtlichen Erklärung verwendet
\newcommand{\submissionDate}{\emph{Noch nicht eingereichte Vorabversion}}

% Konstante: Ort der Abgabe
% Wird in der Ehrenwörtlichen Erklärung verwendet
\newcommand{\submissionLocation}{Hannover}

% Konstante: Tiefe des Inhaltsverzeichnisses
\newcommand{\depthOfToc}{2}

% Konstante: Tiefe der Kapitelnummerierung
% "1" nummeriert bis \section ("Kapitel 1")
% "2" nummeriert bis \subsection ("Kapitel 1.1")
% "3" nummeriert bis \subsubsection ("Kapitel 1.1.1")
% "4" nummeriert bis \subsubsubsection ("Kapitel 1.1.1.1")
% "5" nummeriert bis \subsubsubsubsection ("Kapitel 1.1.1.1.1")
\newcommand{\depthOfSec}{5}
% Dokumentenklasse und Seitenformat
\documentclass[a4paper,12pt,parskip,numbers=noendperiod]{scrartcl}
% Seitenformat, hauptsächlich Seitenränder
\usepackage[paper=a4paper,left=40mm,right=20mm,top=20mm,bottom=30mm]{geometry}
% Zeichensatz
\usepackage[T1]{fontenc}
% Zeichensatz
\usepackage[utf8]{inputenc}
% Trennung nach neuer deutscher Rechtschreibung
\usepackage[ngerman]{babel}
% Ermöglicht das Einbinden von Bildern
\usepackage{graphicx}
% Ermöglicht das Anpassen des Zeilenabstandes
\usepackage{setspace}
\onehalfspacing % Zeilenabstand 150%
% Beschriftung von Floats, d. h. Bilder, Tabellen, etc.
\usepackage[format=hang,font={sf,small},labelfont={sf,small,bf}]{caption}[2008/04/01]
% Ermöglicht benutzerdefinierte Kopf- und Fußzeilen
\usepackage{fancyhdr}
% Bindet alle Verzeichnisse ins Inhaltsverzeichnis ein
% Option [nottoc] blendet das Inhaltsverzeichnis aus
\usepackage[nottoc]{tocbibind}
% Ermöglicht benutzerdefinierte Farben
\usepackage[table]{xcolor}
% Erzeugt klickbare Links im PDF
\usepackage[pdftex,breaklinks,linktocpage,bookmarksnumbered,colorlinks,linkcolor=\linkColor,filecolor=\linkColor,citecolor=\linkColor,urlcolor=\linkColor]{hyperref}
% Ermöglicht ein Glossar und Abkürzungsverzeichnis
\usepackage[acronym,toc]{glossaries}
% Ermöglicht Randnotizen
\usepackage{marginnote}
% Kann entfernt werden, sobald keine Blindtexte mehr gebraucht werden
\usepackage{blindtext}

% Nach Kategorien unterteiltes Quellenverzeichnis
\usepackage{multibib}
\newcites{Literatur,Recht,Internet,Unternehmen}%
	{Literatur,Rechtsvorschriften,Internetquellen,Unternehmensquellen}

%%%%%%%%%%%%%%%%%%%%%%%%%%%%%%%%%%
%             WARNUNG            %
%%%%%%%%%%%%%%%%%%%%%%%%%%%%%%%%%%

% In dieser Datei wirds richtig freakig.
% Also wenn du hier was ändern willst,
% dann solltest du wirklich Ahnung haben.
% Ansonsten lass lieber die Finger davon ;)


%%%%%%%%%%%%%%%%%%%%%%%%%%%%%%%%%%
%       Quellenverzeichnis       %
%%%%%%%%%%%%%%%%%%%%%%%%%%%%%%%%%%

\makeatletter
\renewenvironment{thebibliography}[1]
     {\subsection*{\refname}%
      \@mkboth{\MakeUppercase\refname}{\MakeUppercase\refname}%
      \list{\@biblabel{\@arabic\c@enumiv}}%
           {\settowidth\labelwidth{\@biblabel{#1}}%
            \leftmargin\labelwidth
            \advance\leftmargin\labelsep
            \@openbib@code
            \usecounter{enumiv}%
            \let\p@enumiv\@empty
            \renewcommand\theenumiv{\@arabic\c@enumiv}}%
      \sloppy
      \clubpenalty4000
      \@clubpenalty \clubpenalty
      \widowpenalty4000%
      \sfcode`\.\@m}
     {\def\@noitemerr
       {\@latex@warning{Empty `thebibliography' environment}}%
      \endlist}
\makeatother


%%%%%%%%%%%%%%%%%%%%%%%%%%%%%%%%%%
%       Anhangsverzeichnis       %
%%%%%%%%%%%%%%%%%%%%%%%%%%%%%%%%%%

\makeatletter
\newcommand*{\maintoc}{ % Hauptinhaltsverzeichnis
  \begingroup
    \@fileswfalse % Kein neues Verzeichnis öffnen
    \renewcommand*{\appendixattoc}{ % Trennanweisung im Inhaltsverzeichnis
      %\value{tocdepth}=-10000 % lokal tocdepth auf sehr kleinen Wert setzen
    }%
    \tableofcontents % Verzeichnis ausgeben
  \endgroup
}

\newcommand*{\appendixtoc}{ % Anhangsinhaltsverzeichnis
  \begingroup
    \edef\@alltocdepth{\the\value{tocdepth}} % tocdepth merken
    \setcounter{tocdepth}{-10000} % Keine Verzeichniseinträge
    \renewcommand*{\contentsname}{Anhangsverzeichnis} % Verzeichnisname ändern
    \renewcommand*{\appendixattoc}{ % Trennanweisung im Inhaltsverzeichnis
      \setcounter{tocdepth}{\@alltocdepth} % tocdepth wiederherstellen
    }%
    %\tableofcontents % Verzeichnis ausgeben
    \setcounter{tocdepth}{\@alltocdepth} % tocdepth wiederherstellen
  \endgroup
}
\newcommand*{\appendixattoc}{ % Trennanweisung im Inhaltsverzeichnis
}
\g@addto@macro\appendix{ % \appendix erweitern
  \newpage % Neue Seite
 \phantomsection % Sprungmarke korrigieren
  \addcontentsline{toc}{section}{\appendixname} % Eintrag ins Hauptverzeichnis
  \addtocontents{toc}{\protect\appendixattoc} % Trennanweisung in die toc-Datei
}
\makeatother % Workarounds einbinden

% @see: http://en.wikibooks.org/wiki/LaTeX/Colors#Defining_new_colors

% Für die nachfolgenden Zahlen ("A" bis "F" sind im Hexadezimalsystem auch Zahlen) gilt:
%%%%%%%%%%%%%%%%%%%%%%%%%%%%%%%%%%
%        JE HÖHER DIE ZAHL       %
%     DESTO HELLER DIE FARBE.    %
%%%%%%%%%%%%%%%%%%%%%%%%%%%%%%%%%%


%%%%%%%%%%%%%%%%%%%%%%%%%%%%%%%%%%
%   STANDARDFARBEN FÜR BEFEHLE   %
%%%%%%%%%%%%%%%%%%%%%%%%%%%%%%%%%%
\newcommand{\defaultTableHeadlineBackground}{gray85}
\newcommand{\defaultTableRowBackgroundOdd}{gray95}
\newcommand{\defaultTableRowBackgroundEven}{white}

%%%%%%%%%%%%%%%%%%%%%%%%%%%%%%%%%%
%              GRAU              %
%%%%%%%%%%%%%%%%%%%%%%%%%%%%%%%%%%
\definecolor{gray05}{gray}{0.05}
\definecolor{gray10}{gray}{0.1}
\definecolor{gray15}{gray}{0.15}
\definecolor{gray20}{gray}{0.2}
\definecolor{gray25}{gray}{0.25}
\definecolor{gray30}{gray}{0.3}
\definecolor{gray35}{gray}{0.35}
\definecolor{gray40}{gray}{0.4}
\definecolor{gray45}{gray}{0.45}
\definecolor{gray50}{gray}{0.5}
\definecolor{gray55}{gray}{0.55}
\definecolor{gray60}{gray}{0.6}
\definecolor{gray65}{gray}{0.65}
\definecolor{gray70}{gray}{0.7}
\definecolor{gray75}{gray}{0.75}
\definecolor{gray80}{gray}{0.8}
\definecolor{gray85}{gray}{0.85}
\definecolor{gray90}{gray}{0.9}
\definecolor{gray95}{gray}{0.95}

%%%%%%%%%%%%%%%%%%%%%%%%%%%%%%%%%%
%              ROT               %
%%%%%%%%%%%%%%%%%%%%%%%%%%%%%%%%%%
\definecolor{red1}{HTML}{110000}
\definecolor{red2}{HTML}{220000}
\definecolor{red3}{HTML}{330000}
\definecolor{red4}{HTML}{440000}
\definecolor{red5}{HTML}{550000}
\definecolor{red6}{HTML}{660000}
\definecolor{red7}{HTML}{770000}
\definecolor{red8}{HTML}{880000}
\definecolor{red9}{HTML}{990000}
\definecolor{redA}{HTML}{AA0000}
\definecolor{redB}{HTML}{BB0000}
\definecolor{redC}{HTML}{CC0000}
\definecolor{redD}{HTML}{DD0000}
\definecolor{redE}{HTML}{EE0000}
\definecolor{redF}{HTML}{FF0000}

%%%%%%%%%%%%%%%%%%%%%%%%%%%%%%%%%%
%              GRÜN              %
%%%%%%%%%%%%%%%%%%%%%%%%%%%%%%%%%%
\definecolor{green1}{HTML}{001100}
\definecolor{green2}{HTML}{002200}
\definecolor{green3}{HTML}{003300}
\definecolor{green4}{HTML}{004400}
\definecolor{green5}{HTML}{005500}
\definecolor{green6}{HTML}{006600}
\definecolor{green7}{HTML}{007700}
\definecolor{green8}{HTML}{008800}
\definecolor{green9}{HTML}{009900}
\definecolor{greenA}{HTML}{00AA00}
\definecolor{greenB}{HTML}{00BB00}
\definecolor{greenC}{HTML}{00CC00}
\definecolor{greenD}{HTML}{00DD00}
\definecolor{greenE}{HTML}{00EE00}
\definecolor{greenF}{HTML}{00FF00}

%%%%%%%%%%%%%%%%%%%%%%%%%%%%%%%%%%
%              BLAU              %
%%%%%%%%%%%%%%%%%%%%%%%%%%%%%%%%%%
\definecolor{blue1}{HTML}{000011}
\definecolor{blue2}{HTML}{000022}
\definecolor{blue3}{HTML}{000033}
\definecolor{blue4}{HTML}{000044}
\definecolor{blue5}{HTML}{000055}
\definecolor{blue6}{HTML}{000066}
\definecolor{blue7}{HTML}{000077}
\definecolor{blue8}{HTML}{000088}
\definecolor{blue9}{HTML}{000099}
\definecolor{blueA}{HTML}{0000AA}
\definecolor{blueB}{HTML}{0000BB}
\definecolor{blueC}{HTML}{0000CC}
\definecolor{blueD}{HTML}{0000DD}
\definecolor{blueE}{HTML}{0000EE}
\definecolor{blueF}{HTML}{0000FF}

%%%%%%%%%%%%%%%%%%%%%%%%%%%%%%%%%%
%              GELB              %
%%%%%%%%%%%%%%%%%%%%%%%%%%%%%%%%%%
\definecolor{yellow1}{HTML}{111100}
\definecolor{yellow2}{HTML}{222200}
\definecolor{yellow3}{HTML}{333300}
\definecolor{yellow4}{HTML}{444400}
\definecolor{yellow5}{HTML}{555500}
\definecolor{yellow6}{HTML}{666600}
\definecolor{yellow7}{HTML}{777700}
\definecolor{yellow8}{HTML}{888800}
\definecolor{yellow9}{HTML}{999900}
\definecolor{yellowA}{HTML}{AAAA00}
\definecolor{yellowB}{HTML}{BBBB00}
\definecolor{yellowC}{HTML}{CCCC00}
\definecolor{yellowD}{HTML}{DDDD00}
\definecolor{yellowE}{HTML}{EEEE00}
\definecolor{yellowF}{HTML}{FFFF00}

%%%%%%%%%%%%%%%%%%%%%%%%%%%%%%%%%%
%             TÜRKIS             %
%%%%%%%%%%%%%%%%%%%%%%%%%%%%%%%%%%
\definecolor{turquoise1}{HTML}{001111}
\definecolor{turquoise2}{HTML}{002222}
\definecolor{turquoise3}{HTML}{003333}
\definecolor{turquoise4}{HTML}{004444}
\definecolor{turquoise5}{HTML}{005555}
\definecolor{turquoise6}{HTML}{006666}
\definecolor{turquoise7}{HTML}{007777}
\definecolor{turquoise8}{HTML}{008888}
\definecolor{turquoise9}{HTML}{009999}
\definecolor{turquoiseA}{HTML}{00AAAA}
\definecolor{turquoiseB}{HTML}{00BBBB}
\definecolor{turquoiseC}{HTML}{00CCCC}
\definecolor{turquoiseD}{HTML}{00DDDD}
\definecolor{turquoiseE}{HTML}{00EEEE}
\definecolor{turquoiseF}{HTML}{00FFFF}

%%%%%%%%%%%%%%%%%%%%%%%%%%%%%%%%%%
%              LILA              %
%%%%%%%%%%%%%%%%%%%%%%%%%%%%%%%%%%
\definecolor{purple1}{HTML}{110011}
\definecolor{purple2}{HTML}{220022}
\definecolor{purple3}{HTML}{330033}
\definecolor{purple4}{HTML}{440044}
\definecolor{purple5}{HTML}{550055}
\definecolor{purple6}{HTML}{660066}
\definecolor{purple7}{HTML}{770077}
\definecolor{purple8}{HTML}{880088}
\definecolor{purple9}{HTML}{990099}
\definecolor{purpleA}{HTML}{AA00AA}
\definecolor{purpleB}{HTML}{BB00BB}
\definecolor{purpleC}{HTML}{CC00CC}
\definecolor{purpleD}{HTML}{DD00DD}
\definecolor{purpleE}{HTML}{EE00EE}
\definecolor{purpleF}{HTML}{FF00FF} % Framework-Farben einbinden
%%%%%%%%%%%%%%%%%%%%%%%%%%%%%%%%%%%%%%%%%%%%%%%%
%          BENUTZERDEFINIERTE FARBEN           %
%%%%%%%%%%%%%%%%%%%%%%%%%%%%%%%%%%%%%%%%%%%%%%%%
% @see: http://en.wikibooks.org/wiki/LaTeX/Colors#Defining_new_colors

%%%%%%%%%%%%%%%%%%%%%%%%%%%%%%%%%%
%       FARBEN FÜR BEFEHLE       %
%%%%%%%%%%%%%%%%%%%%%%%%%%%%%%%%%%
%\newcommand{\defaultTableHeadlineBackground}{gray85}

%%%%%%%%%%%%%%%%%%%%%%%%%%%%%%%%%%
%          NEUE FARBEN           %
%%%%%%%%%%%%%%%%%%%%%%%%%%%%%%%%%%
%\definecolor{gray05}{gray}{0.05}
%\definecolor{red1}{HTML}{cc0000} % Benutzerdefinierte Farben einbinden
%%%%%%%%%%%%%%%%%%%%%%%%%%%%%%%%%%%%%%%%%%%%%%%%
%              ALLGEMEINE BEFEHLE              %
%%%%%%%%%%%%%%%%%%%%%%%%%%%%%%%%%%%%%%%%%%%%%%%%

% Neuer Befehl: Unterkapitel unter \subsubsection
\newcommand{\subsubsubsection}[1]{\paragraph{#1}\hfill}

% Neuer Befehl: Unterkapitel unter \subsubsubsection
\newcommand{\subsubsubsubsection}[1]{\subparagraph{#1}\hfill}

% Neuer Befehl: Wechsel auf Sonderschriftart
\newcommand{\fontHeadline}[1]{{\sffamily#1}}

% Neuer Befehl: Setzt den Seitentitel auf den übergebenen Wert
%				OBACHT! ändert den Titel ebenfalls für alle folgenden Seiten -> \resetPageTitle benutzen
\newcommand{\setPageTitle}[1]{\fancyfoot[L]{\fontHeadline{\footnotesize\nouppercase{#1}}}}
% Neuer Befehl: Setzt den Seitentitel auf den Standardwert
\newcommand{\resetPageTitle}{\setPageTitle{\leftmark}}

% Neuer Befehl: Gibt die übergebene Breite relativ zur Textbreite zurück
\newcommand{\relWidth}[1]{#1\textwidth}

% Neuer Befehl: Fachwörter im Text hervorheben
\newcommand{\term}[1]{\emph{#1}}

% Neuer Befehl: Gibt den übergebenen Text in einfachen deutschen Anführungszeichen zurück
\newcommand{\q}[1]{\glq#1\grq\xspace}

% Neuer Befehl: Gibt den übergebenen Text in doppelten deutschen Anführungszeichen zurück
\newcommand{\qq}[1]{\glqq#1\grqq\xspace}

% Neuer Befehl: Ein sichtbares TODO in den Text einfügen
\newcommand{\todo}[1]{%
	\textcolor{redD}{\fontHeadline{{\bfseries ~TODO~}#1~}}%
	\marginnote{\textcolor{redD}{\fontHeadline{{\bfseries !!!}}}}%
}

% Neuer Befehl: Ein sichtbares TODO in den Text einfügen
\newcommand{\TODO}[1][Hier gibt es etwas zu tun]{\todo{#1}}

% Neuer Befehl: Eine sichtbare Frage in den Text einfügen
\newcommand{\QUESTION}[1][]{%
	\textcolor{green8}{\fontHeadline{{\bfseries ~FRAGE~}#1~}}%
	\marginnote{\textcolor{green8}{\fontHeadline{{\bfseries ???}}}}%
}


%%%%%%%%%%%%%%%%%%%%%%%%%%%%%%%%%%%%%%%%%%%%%%%%
%                REFERENZIEREN                 %
%%%%%%%%%%%%%%%%%%%%%%%%%%%%%%%%%%%%%%%%%%%%%%%%

% Neuer Befehl: Erzeugt ein neues Label für ein Kapitel, damit es referenziert werden kann
% @in {}: Labelname des Kapitels
\newcommand{\lblSec}[1]{\label{sec:#1}}

% Neuer Befehl: Gibt eine Referenz zu dem übergebenen Kapitel zurück
% @in {}: Labelname des Kapitels
% @in []: Beschriftung vor der Referenz
\newcommand{\refSec}[2][Abschnitt~]{#1\ref{sec:#2}}

% Neuer Befehl: Gibt die Seite zurück, auf der sich das übergebene Kapitel befindet
% @in {}: Labelname des Kapitels
% @in []: Beschriftung vor der Referenz
\newcommand{\refPageSec}[2][Seite~]{#1\pageref{sec:#2}}

% Neuer Befehl: Gibt eine Referenz zu dem übergebenen Bild zurück
% @in {}: Labelname des Bildes
% @in []: Beschriftung vor der Referenz
\newcommand{\refImg}[2][Abbildung~]{#1\ref{img:#2}}

% Neuer Befehl: Gibt die Seite zurück, auf der sich das übergebene Bild befindet
% @in {}: Labelname des Bildes
% @in []: Beschriftung vor der Referenz
\newcommand{\refPageImg}[2][Seite~]{#1\pageref{img:#2}}

% Neuer Befehl: Gibt eine Referenz zu der übergebenen Tabelle zurück
% @in {}: Labelname der Tabelle
% @in []: Beschriftung vor der Referenz
\newcommand{\refTbl}[2][Tabelle~]{#1\ref{tbl:#2}}

% Neuer Befehl: Gibt die Seite zurück, auf der sich die übergebene Tabelle befindet
% @in {}: Labelname der Tabelle
% @in []: Beschriftung vor der Referenz
\newcommand{\refPageTbl}[2][Seite~]{#1\pageref{tbl:#2}}

% Neuer Befehl: Gibt eine Referenz zu dem übergebenen Bibliothekseintrag zurück
% @in {}: Labelname des Bibliothekseintrags
% @in []: Seiten- bzw. Stellenangabe innerhalb des Eintrags
\newcommand{\refLibLiteratur}{\citeLiteratur}
\newcommand{\refLibInternet}{\citeInternet}
\newcommand{\refLibRecht}{\citeRecht}
\newcommand{\refLibUnternehmen}{\citeUnternehmen}


%%%%%%%%%%%%%%%%%%%%%%%%%%%%%%%%%%%%%%%%%%%%%%%%
%            ABKÜRZUNGEN & GLOSSAR             %
%%%%%%%%%%%%%%%%%%%%%%%%%%%%%%%%%%%%%%%%%%%%%%%%

% Neuer Befehl: Definiert eine neue Abkürzung.
% @in {}: Label
% @in {}: Abkürzung
% @in {}: Beschreibung
\newcommand{\addAc}[3]{%
	\newacronym{#1}{#2}{#3}%
}

% Neuer Befehl: Definiert eine neue Abkürzung mit Pluralformen.
% @in {}: Label
% @in {}: Abkürzung
% @in {}: Abkürzung im Plural
% @in {}: Beschreibung
% @in {}: Beschreibung im Plural
\newcommand{\addAcp}[5]{%
	\newacronym[plural={#3},descriptionplural={#5}]{#1}{#2}{#4}%
}

% Neuer Befehl: Gibt beim ersten Vorkommen Beschreibung + Abkürzung aus,
% danach nur noch die Abkürzung
% Hinweis: Beeinflusst den \ac-Befehl
% @in {}: Label der Abkürzung
\newcommand{\ac}{\gls}

% Neuer Befehl: Gibt die Abkürzung mit angehängtem Suffix aus
% (z. B. für den Genitiv).
% Hinweis: Beeinflusst den \ac-Befehl
% @in {}: Label der Abkürzung
% @in {}: Suffix
\newcommand{\ace}[2]{\glsdisp{#1}{\glsname{#1}#2}}

% Neuer Befehl: Gibt nur die Abkürzung aus.
% Hinweis: Beeinflusst den \ac-Befehl _nicht_
% @in {}: Label der Abkürzung
\newcommand{\acs}{\glsname}

% Neuer Befehl: Gibt nur die Beschreibung aus.
% Obacht: Funktioniert nicht mit Sonderzeichen oder Umlauten!
% Hinweis: Beeinflusst den \ac-Befehl _nicht_
% @in {}: Label der Abkürzung
\newcommand{\acl}{\glsdesc}

% Neuer Befehl: Gibt immer Beschreibung + Abkürzung aus.
% Hinweis: Beeinflusst den \ac-Befehl _nicht_
% @in {}: Label der Abkürzung
\newcommand{\acf}{\glsfirst}

% Neuer Befehl: Gibt beim ersten Vorkommen Beschreibung + Abkürzung im Plural aus,
% danach nur noch die Abkürzung
% Hinweis: Beeinflusst den \ac-Befehl
% @in {}: Label der Abkürzung
\newcommand{\acp}{\glspl}

% Neuer Befehl: Gibt die Abkürzung im Plural aus.
% Hinweis: Beeinflusst den \ac-Befehl _nicht_
% @in {}: Label der Abkürzung
\newcommand{\acsp}{\glsplural}

% Neuer Befehl: Gibt Beschreibung + Abkürzung im Plural aus.
% Hinweis: Beeinflusst den \ac-Befehl _nicht_
% @in {}: Label der Abkürzung
\newcommand{\acfp}{\glsfirstplural}

% Neuer Befehl: Gibt einen Link zur angegebenen Abkürzung
% mit dem angegebenen Text (statt der vordefinierten Beschreibung) aus.
% Hinweis: Beeinflusst den \ac-Befehl
% @in {}: Label der Abkürzung
% @in {}: Anzeigetext
\newcommand{\aca}{\glsdisp}

% Neuer Befehl: Glossareintrag hinzufügen.
% @in {}: Label des Eintrags
% @in {}: Name des Eintrags
% @in {}: Pluralform des Namens
% @in {}: Beschreibung
\newcommand{\addGl}[4]{%
	\newglossaryentry{#1}{name=#2,description={#4},plural=#3,sort=#1}%
}

% Neuer Befehl: Gibt den Namen des Glossareintrags aus
% und erzeugt einen Link zum entsprechenden Eintrag.
% @in {}: Label des Eintrags
\newcommand{\gl}{\gls}

% Neuer Befehl: Gibt den Namen des Glossareintrags in der Pluralform aus
% und erzeugt einen Link zum entsprechenden Eintrag.
% @in {}: Label des Eintrags
\newcommand{\glp}{\glspl}

% Neuer Befehl: Gibt einen Link zum angegebenen Glossareintrag
% mit dem angegebenen Text (statt der vordefinierten Beschreibung) aus.
% @in {}: Label der Eintrags
% @in {}: Anzeigetext
\newcommand{\gla}{\glsdisp}


%%%%%%%%%%%%%%%%%%%%%%%%%%%%%%%%%%%%%%%%%%%%%%%%
%                    BILDER                    %
%%%%%%%%%%%%%%%%%%%%%%%%%%%%%%%%%%%%%%%%%%%%%%%%

% Neuer Befehl: Bild ohne Beschriftung einfügen
% @in {} 1:	Pfad des Bildes
% @in []:	Optionen für das Bild wie Breite, Drehung, etc. - mehrere Optionen kommasepariert, Reihenfolge einhalten!
%			Key			Value				Beschreibung
%			---------------------------------------------------------------------------
%			angle		<zahl>				Dreht das Bild um <zahl> Grad
% 			width		<zahl>				Breitenangabe in Pixeln
%						<zahl>cm/mm/in		Breitenangabe in Zentimeter/Millimeter/Inch
%						\relWidth{<zahl>}	Breitenangabe relativ zur Textbreite
%			height		s. width
%			scale		<zahl>				Skaliert das Bild auf die angegebene Größe
%			---------------------------------------------------------------------------
%			Beispiel: \imgUnlabeled{bild.jpg} oder \imgUnlabeled[angle=-90,width=42cm]{bild.jpg}
\newcommand{\imgUnlabeled}[2][scale=1.0]{\includegraphics[#1]{\imgPath#2}}

% Neuer Befehl: Bild mit Beschriftung einfügen
% @in {}1:	Pfad des Bildes
% @in {}2:	Sichtbare Bildbeschriftung
% @in {}3:	Internes Label (eineindeutig im Gesamtdokument!)
% @in []:	Optionen für das Bild, s. \imgUnlabeled
% Beispiel: \img{bild.jpg}{Ein schönes Bild}{bild1} oder \img[angle=-90,width=42cm]{bild.jpg}{Ein schönes Bild}{bild1}
\newcommand{\img}[4][scale=1.0]{
	\begin{figure}[h!tbp]
	\centering
	\imgUnlabeled[#1]{#2}
	\caption{#3}
	\label{img:#4}
	\end{figure}
}


%%%%%%%%%%%%%%%%%%%%%%%%%%%%%%%%%%%%%%%%%%%%%%%%
%                   TABELLEN                   %
%%%%%%%%%%%%%%%%%%%%%%%%%%%%%%%%%%%%%%%%%%%%%%%%

% Neuer Befehl: Fügt eine Tabelle ohne Linien ein
% @in {}: Sichtbare Beschriftung
% @in {}: Labelname zum Referenzieren der Tabelle
% @in {}: Tabellenstil, [ll] erzeugt beispielsweise zwei linksbündige Spalten
% @in {}: Tabelleninhalt (Tipp: Neue Zeilen mit \tr[] erzeugen)
% @in {}: Zusatzoptionen wie Farbdefinitionen für Zeilen
%		  \rowcolors{1}{red}{green} erzeugt bspw. ungerade Zeilen mit rotem
%		  und gerade Zeilen mit grünen Hintergrund ab der ersten Zeile
% @see: http://en.wikibooks.org/wiki/LaTeX/Tables
\newcommand{\tblNoBorder}[5]{
	\begin{table}[h!tbp]
	#5
	\caption{#1}
	\centering
	\begin{tabular}[t]{#3}
	#4
	\end{tabular}
	\label{tbl:#2}
	\end{table}
}

% Neuer Befehl: Fügt eine Tabelle mit einer horizontalen Trennlinie oben ein
% @in {}: Sichtbare Beschriftung
% @in {}: Labelname zum Referenzieren der Tabelle
% @in {}: Tabellenstil, [|l|l|] erzeugt beispielsweise zwei linksbündige Spalten mit vertikalen Trennlinien
%		  Mögliche Optionen: l = linksbündig, r = rechtsbündig, c = zentriert
%							 p{\relWidth{0.5}} = linksbündig mit fester Breite, bricht Text automatisch um
% @in {}: Tabelleninhalt (Tipp: Neue Zeilen mit \tr erzeugen)
% @in {}: Zusatzoptionen wie Farbdefinitionen für Zeilen
%		  \rowcolors{1}{red}{green} erzeugt bspw. ungerade Zeilen mit rotem
%		  und gerade Zeilen mit grünen Hintergrund ab der ersten Zeile
% @see: http://en.wikibooks.org/wiki/LaTeX/Tables
\newcommand{\tbl}[5]{\tblNoBorder{#1}{#2}{#3}{\hline#4}{#5}}

% Neuer Befehl: Erzeugt eine neue Zeile in einer Tabelle
% @in []: Freilassen (nicht weglassen!), um keine Linie zu erzeugen
\newcommand{\tr}[1][\hline]{\\#1}

% Neuer Befehl: Formatiert Text in einer Zelle als Überschriftszelle
% @in {}: Der Text in der Zelle (ohne "&")
% @in []: Hintergrundfarbe der Zelle
\newcommand{\thl}[2][\defaultTableHeadlineBackground]{\textbf{#2}\cellcolor{#1}}

% Neuer Befehl: Zusatzoption für Farbdefinitionen in Tabellen: Färbt die Zeilen abwechseln mit Standardfarben
% @in []: Erste Zeile ab der gefärbt wird (standardmäßig ab der ersten, sprich alle Zeilen werden gefärbt)
\newcommand{\defaultTblBg}[1][1]{\rowcolors{#1}{\defaultTableRowBackgroundOdd}{\defaultTableRowBackgroundEven}}


%%%%%%%%%%%%%%%%%%%%%%%%%%%%%%%%%%%%%%%%%%%%%%%%
%       SEITENSPEZIFISCHE FORMATIERUNGEN       %
%%%%%%%%%%%%%%%%%%%%%%%%%%%%%%%%%%%%%%%%%%%%%%%%

% Neuer Befehl: Formatbox Titelseite
\newcommand{\titlepageBox}[1]{\fontHeadline{\normalsize#1}\\\vfill{}}

% Neuer Befehl: Formatbox Titelseite mit Überschrift
\newcommand{\titlepageBoxHeadline}[2]{\fontHeadline{\normalsize#1}\\[0.2cm]\titlepageBox{#2}} % Framework-Kommandos einbinden
%%%%%%%%%%%%%%%%%%%%%%%%%%%%%%%%%%%%%%%%%%%%%%%%
%          BENUTZERDEFINIERTE BEFEHLE          %
%%%%%%%%%%%%%%%%%%%%%%%%%%%%%%%%%%%%%%%%%%%%%%%%

% Neuer Befehl: Beispiel
%\newcommand{\example}{\TODO[Gib was aus, was mir gefällt!]} % Benutzerdefinierte Kommandos einbinden

\setcounter{tocdepth}{\depthOfToc} % Tiefe des Inhaltsverzeichnisses setzen
\setcounter{secnumdepth}{\depthOfSec} % Tiefe der Kapitelnummerierung setzen

% Kurzanleitung (weiteres siehe Commands-Datei)

% Ein Glossareintrag wird folgendermaßen hinzugefügt:
%	\addGl{Label}{Name}{Name im Plural}{Beschreibung}
% Das Label sollte dabei nur [A-Za-z0-9] enthalten
% und wird nur zum Referenzieren benutzt (nie im Dokument).
% Es muss natürlich global eindeutig sein.

% Ein Glossareintrag wird im Text folgendermaßen referenziert:
%	\gl{Label}	Gibt den Namen des Glossareintrags aus
%				und erzeugt einen Link zum entsprechenden Eintrag.
%	\glp{Label}	Gibt den Namen des Glossareintrags in der Pluralform aus
%				und erzeugt einen Link zum entsprechenden Eintrag.

% Möchte man nicht den normalen Namen ausgeben,
% gibt es zudem die folgende Möglichkeit:
%	\gla{Label}{Alternative Beschriftung}

\addGl{Acronym}{Acronym}{Acronyme}{%
	Acronyms behave a bit differently than normal glossary terms. On first use the ac command will
	display \qq{<full> (<abbrv>)}. On subsequent uses only the abbreviation will be displayed.
}
\addGl{Webservice}{Webservice}{Webservices}{%
	Das World Wide Web Consortium definiert die Bereitstellung eines Webservices als Unterstützung zur
	Zusammenarbeit zwischen verschiedenen Anwendungsprogrammen, die auf unterschiedlichen Plattformen
	und/oder Frameworks betrieben werden.
	Ein Webservice oder Webdienst ist eine Software-Anwendung, die mit einem Uniform Resource Identifier
	(URI) eindeutig identifizierbar ist und deren Schnittstelle als XML-Artefakt definiert, beschrieben
	und gefunden werden kann. Ein Webservice unterstützt die direkte Interaktion mit anderen
	Software-Agenten unter Verwendung XML-basierter Nachrichten durch den Austausch über
	internetbasierte Protokolle.
}
\addGl{Xena}{Xena}{Xenaesi}{%
	Dies ist ein Artikel über Xena, die fast so aussieht wie Brigitte. Brigitte hat lange blonde Haare,
	trägt ne Brille und nur Schuhe.
}











 % Glossar einbinden
\deftranslation[to=ngerman]{Glossary}{Glossar} % Titel übersetzen
% Kurzanleitung (weiteres siehe Commands-Datei)

% Eine Abkürzung wird folgendermaßen hinzugefügt:
%	\addAc{Label}{Kurzform}{Langform}
% Das Label sollte dabei nur [A-Za-z0-9] enthalten
% und wird nur zum Referenzieren benutzt (nie im Dokument).
% Es muss natürlich global eindeutig sein.

% Alternativ lässt sich eine Abkürzung mit Pluralform hinzufügen:
%	\addAcp{Label}{Kurzform}{Kurzform im Plural}{Langform}{Langform im Plural}

% Eine Abkürzung wird im Text folgendermaßen referenziert:
%	\ac{Label}	Gibt beim ersten Vorkommen Beschreibung + Abkürzung aus, danach nur noch die Abkürzung
%	\acf{Label}	Beschreibung + Abkürzung wird ausgegeben
%	\acs{Label}	Nur die Abkürzung wird ausgegeben
%	\acl{Label}	Nur die Beschreibung wird ausgegeben

% Für die Pluralformen gibt es analog:
%	\acp{Label}
%	\acfp{Label}
%	\acsp{Label}
% Leider gibt es \aclp{Label} aufgrund eines internen Fehlers nicht.

% Möchte man nicht die normale Kurz- bzw. Langform ausgeben,
% gibt es zudem die folgende Möglichkeit:
%	\aca{Label}{Alternative Beschriftung}

% Und falls man z. B. den Genitiv bilden möchte:
%	\ace{Label}{Suffix}	Gibt die Kurzform mit angehängtem Suffix aus

\addAc	{EDV}{EDV}{Elektronische Datenverarbeitung}
\addAc	{GmbH}{GmbH}{Gesellschaft mit beschränkter Haftung}
\addAcp	{ibA}{ibA}{ibAs}{iVAS-basierende Anwendung}{iVAS-basierende Anwendungen}
\addAc	{IT}{IT}{Informationstechnologie}
\addAc	{iVAS}{iVAS}{ivv Versicherungsanwendungssystem}
\addAc	{ivv}{ivv}{Informationsverarbeitung für Versicherungen GmbH}
\addAc	{Java EE}{Java EE}{Java Platform, Enterprise Edition}
\addAc	{OEVO}{ÖVO}{Öffentliche Versicherung Oldenburg}
\addAc	{VGH}{VGH}{Versicherungsgruppe Hannover} % Abkürzungsverzeichnis einbinden
\deftranslation[to=ngerman]{Acronyms}{Abkürzungsverzeichnis} % Titel übersetzen
\makeglossaries % Glossar indizieren

\pagestyle{fancy} % Benutzerdefinierte Konfiguration der Kopf- und Fußzeile aktivieren
\fancyhf{} % Setzt die Kopf- und Fußzeile zurück (scheint wichtig zu sein)
\renewcommand{\headrulewidth}{0pt} % Dicke der Trennlinie in der Kopfzeile
\setlength{\footskip}{36pt} % Abstand zwischen Fußzeile und Textblock
\renewcommand{\footrulewidth}{0.4pt} % Dicke der Trennlinie in der Fußzeile
\renewcommand{\footruleskip}{6pt} % Abstand der Trennlinie in der Fußzeile zum Text
\fancyfoot[R]{\fontHeadline{\footnotesize\thepage}} % Beschriftung rechte Seite Fußzeile

% Die nachfolgenden Wörter werden im gesamten Dokument
% ausschließlich wie angegeben getrennt.
% Obacht: Umlaute und Sonderzeichen werden nicht unterstützt!

\hyphenation{%
In-for-ma-ti-ons-ver-ar-bei-tung
Ver-si-che-run-gen
} % Zusätzliche Trennregeln einbinden
\begin{document} % Ab hier beginnt der wirkliche Text, alles nachfolgende erscheint im Dokument
\resetPageTitle % Seitentitel initial setzen

% Titelseite
\pagenumbering{Roman} % Nummerierung der Seiten in römischen Zahlen
\thispagestyle{empty} % Kopf- und Fußzeile ausblenden
\begin{center}

~\vfill

\titlepageBox{
	\imgUnlabeled[width=\relWidth{0.35}]{logo-fhdw.pdf}
}

\titlepageBoxHeadline{Fachhochschule für die Wirtschaft Hannover}{-- FHDW --}

\titlepageBoxHeadline{Mitschrift der Vorlesung}{\textbf{Quantitative Forschungsmethoden}}
\titlepageBoxHeadline{Thema}{\textbf{Protokoll vom 18.05.2013}}


\titlepageBoxHeadline{Verfasser:}{
	Niels Alexander Bellhäuser
}

\titlepageBoxHeadline{2. Theoriequartal}{%
	Studiengang Master of Science (M.Sc.)\\%
	Business Process Engineering%
}

\titlepageBoxHeadline{Eingereicht am:}{\today}

\end{center} % Inhalt einbinden

% Abstract
%\newpage
%\setPageTitle{Abstract} % Seitentitel ändern, da diese Section ausgeblendet ist
%\include{\content/pages/Abstract} % Inhalt einbinden

% Inhaltsverzeichnis
\newpage
\resetPageTitle % Wichtig: Seitentitel wieder zurücksetzen, da er auf der letzten Seite geändert wurde
\maintoc % Inhaltsverzeichnis einfügen

% Beginn des Inhalts
\newpage
\pagenumbering{arabic} % Nummerierung der Seiten in arabischen Zahlen
\setcounter{page}{1} % Seitenzahlen zurück auf Anfang setzen
\glsresetall % Alle verwendeten Abkürzungen zurücksetzen

% Kapitel
\section{Einleitung}

Im Rahmen des Master Studiums (M.Sc. Studiengang: Business Process Engineering) an der FHDW Hannover
war es für die Studiengruppe \qq{HFP412} die Aufgabe, Inhalte und Hintergründe der Veranstaltung \qq{Quantitative Forschungsmethoden}\footnote{Dozentin: Frau Dr. Sylvie Gasnier} zu dokumentieren.

\img[width=\relWidth{0.9}]{skalenniveaus.pdf}{Verschiedene Zahlen- oder Skalenniveaus}{img_skalenniveaus}

Die vorliegende Ausarbeitung konzentriert sich auf den Zweig der qualitativen Skalen. Genauer das Thema der \qq{Nominal- und Ordinalskalen} (siehe \refImg{img_skalenniveaus}). 
Die theoretischen Hintergründen werden anhand von praktischen Beispielen mit Hilfe von Microsoft Excel untermauert. 
Als Datenquelle dient hierfür der Münchner Mietspiegel aus dem Jahre 2003\footnote{\url{http://data.ub.uni-muenchen.de/2/1/miete03.asc}}.

Um die statistische Auswertung zu erleichtern, müssen einige Vorarbeiten geleistet werden. Zunächst müssen pro Merkmalsart die verschiedenen Ausprägungen 
analysiert werden. Hierbei entstehen ein Codeplan und eine Datenmatrix, welche als Grundlage für die folgenden Auswertungen dienen. 
Auf die konkreten Ausprägungsarten wird zu einem späteren Zeitpunkt eingegangen. Um an die
Ausprägungsarten herzuführen, werden in Abschnitt \ref{sec:defNot} zunächst Codeplan und Datenmatrix
skizziert.
 % Inhalt einbinden

% Kapitel
%\section{Grundlagen}

\blindtext

\subsection{Idee des Metamodells}

\blindtext

\subsection{Theoretischer Ansatz von Martin Fowler}

\TODO[Bezug zum Buch herstellen und als Auslöser darstellen] \refLibLiteratur{fowler1997analysis}

\TODO[Beschreiben, welche Teile umgesetzt wurden, und dann auf die Folgekapitel verweisen]
 % Inhalt einbinden

% Kapitel
\section{Metamodell} 

% Kapitel
\subsection{Types}

\TODO[Kapitel schreiben] % Inhalt einbinden

% Kapitel
\subsection{Operations und Associations}

\subsubsection{Operationen}

Die Klasse Operation repräsentiert die Abbildung von Operationen in dem erarbeiteten Modell.
Wie in dem Modell aus Abbildung XXX ersichtlich, hat die Oberklasse (AbstractOperation) zwei direkte Assoziationen zu Type. 
Im Kontext von Operationenen ist die Quelle (source) der Typ, welcher die Operation enthält. 
Das Ziel (target) einer Operation entspricht dem Rückgabetypen.
Neben Quell und Zieltypen enthält eine Operation eine Menge von Formalparametern welche 
wiederum über die Klasse Type typisiert sind.

Die Klasse Operation enthält nur genau zwei Operation.
Die Operation \textbf{isStatic()} gibt genau dann true zurück, wenn der Quelltyp dieser Operation der leeren Summe entspricht.
Zusätzlich zu der Bedingung aus isStatic gilt für die zweite Operation \textbf{isConstant()}, dass die Operation keine Parameter enthalten darf. 

\subsubsubsection{OperationManager} \newline
Der OperatioManager bietet 10 transaktionale Operationen und beinhaltet zur Verwaltung vier Listen wobei zwei davon nur abgeleitete Informationen
enthalten.

\begin{description}
\item[Operation** operations] Diese Liste enthält alle Operationen die der Manager verwaltet.
\item[derived Operation** staticOperations] Diese Liste enthält ausschließlich statische Operationen: $\forall \, o  \in staticOperations : o.isStatic()==true$ 
\item[derived Operation** constants] Diese Liste enthält ausschließlich Konstanten. Demnach gilt: $\forall \, o  \in constants : o.isConstant()==true$
\item[FormalParameter** formalParameters] Hier werden alle Formalparameter abgelegt die der Manager verwaltet.
\end{description}


TODO Operationen beschreiben!

\subsubsection{Assoziationen}

Die Klasse Association repräsentiert die Möglichkeit Assoziationen zwischen zwei beliebigen Typen abzubilden.
Neben dem Quell- und Zieltypen, kann eine Assoziation an beliebig vielen Hierarchien teilnehmen.
Eine Assoziation entspricht der Interpretation einer Observation, wenn alle Zieltypen entweder singleton sind oder abstrakt und als Unterklassen nur singletons enthalten.
Wie bei vielen anderen Modellbestandteilen, sind auch die Namen der Assoziation indiziert. Daraus folgt auch, dass es keine gleichnamigen 
Assoziationen geben darf.

Wie in der Abbildung XXX zu sehen ist, können sowohl Operationen als auch Assoziationen Formalparameter beinhalten. Die Möglichkeit der Zuweisung von
Formalparametern an Assoziationen, wird dem Anwender über die Oberfläche nicht bereitgestellt. 
Eine Mögliche zur Interpretation von Formalparametern für Assoziationen sind die n-stelligen Assoziationen
welche in der ersten Implementierungsstufe nicht umgesetzt wurden.

\subsubsubsection{AssociationManager} \newline
Der AssoziationManager bietet sechs transaktionale Operationen an und beinhaltet zwei Listen, welche er verwaltet.
Die eine Liste beinhaltet alle erstellten Assoziationen und die andere alle Hierarchien. Diese beiden Listen werden dem Anwender an der 
Oberfläche präsentiert und dienen der Verwaltung der darin enthaltenen Objekte.

Im Folgenden werden die Operationen des Managers kurz beschrieben, wobei sowohl \textbf{Parameter} als auch \emph{Exceptions} durch entsprechende Formatierungen hervorgehoben wurden.

\begin{description}
\item[createAssociation] Bei Aufruf dieser Operation müssen \textbf{Quell- und Zieltyp} sowie der \textbf{Name} der zu erstellenden Assoziation 
angegeben werden.
Sollte der Name schon von einer anderen Assoziation verwendet werden, wirft die Operation eine \emph{DoubleDefinitionException} und die Assoziation wird nicht
erstellt. Ein weiterer Grund für das nicht Erstellen einer Assoziation ist, wenn Quell- oder Zieltyp der leeren Summe entsprechen. In einem solchen Fall
wird eine \emph{ConsistencyException} erstellt.
Wenn keine der genannten Exceptions aufgetreten ist, wird die Assoziation erstellt und zur Liste der bekannten Assoziationen hingefügt. 
\item[removeAssociation] Diese Operation erwartet beim Aufruf nur die zu entfernende \textbf{Assoziation}. 
Bei erfolgreicher Durchführung dieser Operation, wird die
angegebene Assoziation aus dem System entfernt.
Sollte es ein oder mehrere Exemplare (Links) zu dieser Assoziation geben oder sich die zu löschende Assoziation in mindestens einer Hierarchie befinden, wird eine \emph{ConsistencyException} erzeugt und die Assoziation wird nicht entfernt. 
\item[createHierarchy] Da es in dem erarbeiteten Modell keine leeren Hierarchien geben darf, wird neben dem Namen der zu erstellenden \textbf{Hierarchie} auch eine 
\textbf{Assoziation} erwartet. Nach erfolgreicher Durchführung, wird die Hierarchie erstellt und die angegebene Assoziation der neuen Hierarchie zugeordnet.
Insbesondere folgende Gründe führen zum Misserfolg:
\begin{itemize}
\item Es exisitiert bereits eine Hierarchie mit diesem Namen. Das hat zur Folge, dass eine \emph{DoubleDefinitionException} erzeugt wird.
\item Es existiert auf der Exemplarebene ein Zyklus. Wie in der Beschreibung zu addAssociation erläutert, ist dieser Umstand ist für Hierarchien untersagt.
 Es resultiert eine \emph{CycleException.}
\end{itemize}
\item[removeHierarchy] Nach erfolgreicher Ausführung dieser Operation, wird die übergebene \textbf{Hierarchie} aus dem System entfernt. 
Es wird lediglich die Hierarchie entfernt, die beinhaltenden Assoziationen an sich bleiben weiterhin bestehen.
\item[addAssociation] Diese Operation erwartet zum einen die \textbf{Hierarchie} zu der die Assoziation hinzugefügt werden soll und zum Anderen die \textbf{Assoziation} selbst.
Wenn es auf der Exemplarbene keine zyklischen Links zu dieser Assoziation gibt, wird die Assoziation der Hierarchie zugeordnet. Sollte es zyklische Links
zu dieser Assoziation geben oder die Assoziation ist bereits in dieser Hierarchie, werden entsprechende Exceptions (\emph{CycleException,} bzw. \emph{DoubleDefinitionException}) 
erstellt und die Assoziation wird der Hierarchie nicht zugeordnet. UMSCHREIBEN(wirklich)
\item[removeAssoFrmHier] Zum Entfernen von Assoziationen aus Hierarchien, ist es notwendig dieser Operation sowohl die \textbf{Hierarchie}, als auch die zu entfernende 
\textbf{Assoziation} anzugeben. Dabei wird lediglich die Assoziation aus der Hierarchie entfernt, die Assoziation an sich bleibt bestehen. 
Sofern die angegebene Assoziation an dieser Hierarchie teilnimmt, wird diese Verbindung durch die Ausführung dieser Operation entfernt. 
Sollte dies nicht der Fall sein, wird eine \emph{NotAvailableException} geworfen. Ein Sonderfall bildet sie letzte Assoziation einer
Hierarchie. Diese darf gemäß des Modells nicht entfernt werden, da leere Hierarchien nicht zuläßig sind. Beim Versucht die letzte
Assoziation einer Hierarchie aus der Hierarchie zu entfernen, wird eine \emph{ConsistencyException} erzeugt und der Versucht wird abgebrochen. 
\end{description}



\subsection{Messages und Links}

Messages und Links haben eine gemeinsame Oberklasse MessageOrLink. In der
ersten Ausbaustufe werden lediglich die Links implementiert. Zu den Messages existiert eine rudimentäre
Implementierung die dem Anwender aber über die Oberfläche nicht bereitgestellt wird, da die 
Semantik eines Messageexemplars bislang nicht eindeutig geklärt ist. \vspace{15pt}

Ein Link ist eine konkrete Ausprägungen einer Assoziation. 
Folglich ist die Erstellung eines Links nur möglich, wenn eine Assoziation als Typ gewählt wurde und der zu erstellende Link
sich an die  daraus ergebenden Konsistenzbedingungen hält.
Für Links stehen dem Anwender nur zwei Operationen zur Verfügung.

\subsubsection{Einen Link erstellen - createLink}
Wie zuvor erwähnt, ist es für einen Link zwingend notwendig sich an die Konsistenzbedingungen der Assoziation zu halten.
Bei der Erstellung eines Links wird eine \emph{ConsistencyException} erzeugt, wenn die Quell- und/oder Zielobjekte gemäß der Typebene nicht 
gestattet sind.

\begin{enumerate}
\item $\forall \, l \in Link: l.source.type.isLessOrEqual(l.type.source)$
\item $\forall \, l \in Link: l.target.type.isLessOrEqual(l.type.target)$
\end{enumerate}

Desweiteren ist es Assoziationen möglich an mehreren Hierarchien teilzunehmen. Zyklische Assoziationen sind auf der Metaebene möglich, 
da sich sonst keinerlei Aggregationsbeziehung abbilden ließe.
Auf der Ebene der Exemplare hingegen, müssen \emph{alle Links, die in Assoziation typisiert sind, welche an einer oder mehreren Hierarchien teilnehmen, zyklenfrei sein.}
Wenn ein Anwender versucht zyklische Links zu erzeugen, wird eine \emph{CycleException} erzeugt und der jeweilige Link wird nicht erstellt.

\subsubsection{Löschen eines Links - removeLink}
Beim Entfernen eines bestehenden Links gibt es keinerlei Restriktionen zu beachten. Der Anwender kann bestehende Links beliebig löschen. % Inhalt einbinden

% Kapitel
\subsection{Quantity}
TODO Dient zum Abbilden von Werten mit Einheit

\subsubsection{Die Manager} \newline
\subsubsubsection{UnitTypeManager} \newline
Der UnitTypeManager beinhaltet zwei auf der Oberfläche sichtbare und zwei nicht sichtbare Listen zur Verwaltung von Einheiten (Units) und Einheitentypen (UnitTypes).

\begin{description}
\item[AbsUnitType** unitTypes] Diese Liste enthält alle Einheitentypen, die der Manager verwaltet.
\item[AbsUnit ** units] Hier werden alle Einheiten abgelegt, die der Manager verwaltet.
\item[ReferenceType** refTypes] Diese Liste enthält alle ReferenzTypen, die der Manager verwaltet. Sie ist nicht sichtbar und dient nur zur internen Verarbeitung von zusammengesetzten Einheitentypen (CompoundUnitTypes)
\item[Reference** refs] Hier werden alle Referenzen abgelegt, die der Manager verwaltet. Sie ist nicht sichtbar und dient nur zur internen Verarbeitung von zusammengesetzten Einheiten (CompoundUnits)
\end{description}

\subsubsubsection{QuantityManager} \newline

\subsubsection{Einheitentypen} \newline

\subsubsection{Einheiten} \newline
Fachlich gesehen können Einheiten (Units) von Quantitäten angenommen werden und sind in Einheitentypen typisiert.
Dabei wird zwischen atomaren Einheiten (z.B. Meter [m]) und zusammengesetzten Einheiten (z.B. Kilometer pro Stunde [km/h]) unterschieden.
\subsubsubsection{Verwalten von Units} \newline
Für die Verwaltung von Units und CompoundUnits stellt der UnitTypeManager acht transaktionale Operationen bereit.
Folgende dieser Operationen sind direkt über die Oberfläche erreichbar:

\begin{description}
\item[createUnit]
Diese Operationen dient zum Erstellen einer neuen Unit. Da jede Unit fachlich in einem UnitType typisiert werden und einen Namen haben muss, können dieser Methode diese Werte entsprechend übergeben werden. Eine DoubleDefinitionException wird geworfen, wenn eine Unit mit dem gewählen Nmane bereits existiert.
\item[changeUName]
Diese Operationen dient zum umbenennen einer Unit. Auch hier wird die DoubleDefinitionException im doppelten Namensfall geworfen.
\item[fetchScalar]
Liefert die eine CompountUnit, die keine Referenzen zu anderen Units aufweist.
\item[addReference]
Diese Operation kann sowohl auf Zusammengesetzten Einheiten, als auch auf atomaren Einheiten angewendet werden. Sie dient zum erstellen von CompoundUnits. Entsprechend der createUnit-Methode wird hier ein Name benötigt. Die entsprechende neue Referenz zur ausgewählten Unit wird durch einen Exponenten definiert. Die DoubleDefinitionException wird auch hier geworfen, dalls es zu Namenskonflikten kommt.
\end{description}

Folgende Operationen sind für die interne Verwarbeitung relevant und können nicht über die Oberfläche aufgerufen werden:
\begin{description}

\item[getExistingCU]
Hier wird anhand einer Liste von vorhandenen Referenzen eine CompoundUnit ermittelt, welche durch genau diese Referenzen definiert ist. Sollte diese Unit noch nicht existieren, wird null zurückgeliefert. Diese Operation dient zum vermeiden von Doppelt anlegten CompoundUnits.
\item[fetchCU]
Diese Operation ist ähnlich der getExistingCU()-operation. jedoch wird hierbei die CompoundUnit angelegt, falls sie noch nicht existiert. Die Angabe eines Namens ist erforderlich, falls eine neue CompoundUnit zustande kommen sollte. Auch hier wird entsprechend eine DoubleDefinitionException geworfen, falls eine Unit mit dem gewählen Nmane bereits existiert.
\item[fetchReference]
Mihilfe dieser Operation kann eine Referenz-Instanz mit einem gewissen Exponenten auf eine bestimmte Unit ermittelt werden. Falls diese Instanz noch nicht existierte, wird sie erzeugt. Das dient zum vermeiden von Doppelt anlegten Referenzen.
\end{description}

\subsubsubsection{Conversions} \newline
Eine Conversion, also die Umrechnung von einer Unit zur entspreched zum UnitTyp gehörigen DefaultUnit, kann über zwei Wege zustande kommen, bzw. verändet werden: Zum einen über die setConversion-Operation und zum anderen über die setDefaultUnit-Operation.
Conversions sind immer lineare Funktionen, damit eine Umkehrbarkeit gewährleistet ist. Jede Conversion enthält also einen Faktor und eine Konstante um der linearen Funktion mx+b gerecht zu weden.
Mittels setConversion() wird eine bereits gesetzte Conversion für eine Unit überschrieben.
Bei setDefaultUnit() wird eine Umrechnung von bereits vorhanden Conversions notwendig, da diese ja in Abhängigkeit zu einer nun veralteten DefaultUnit angegeben wurden. Das betrifft alle Umrechnungen für Units zum selben UnitType wie die DefaultUnit. Die folgende Grafik beschreibt die Umrechnungen, falls sich eine DefaultUnit ändert.
\begin{figure}[setDefaultUnit]
	\includegraphics{images/setDefaultUnit}
	\caption{Umrechnung beim Ändern einer DefaultUnit}
\end{figure}

\subsubsection{Quantitäten} \newline % Inhalt einbinden

% Kapitel
\subsection{Measurement}
Martin Fowler unterscheidet in seinem Werk \refLibLiteratur{fowler1997analysis} zwischen qualitativen und quantitativen Messungen. 
Wird beispielsweise die Blutgruppen (A, B, AB und 0) eines Menschen typisiert, so handelt es sich um eine qualitative Messung. 
Im Kontext dieses Projekts wird diese Art von Messung als \term{Observation} (siehe Abschnitt \ref{Operation:Associationen}) bezeichnet. 
Unter quantitativen Messungen, bezeichnet als \term{Measurement}, wird beispielsweise das Feststellen der Körpergröße eines Menschen verstanden.
Measurements (beispielsweise Körpergröße von Hugo Egon Balder: 190cm) werden in Form von Quantitäten bzw. \term{Quantities} auf Konten (Accounts) erfasst.
Des Weiteren gibt es \term{MeasurementTypes} und \term{AccountTypes}, welche in diesem Abschnitt noch genauer erläutert werden.

\TODO[Alex: noch Ergänzungen? Bzw. 'ne bessere Überleitung zum folgenden Absatz?]

Auf der Modellebene befinden sich im Bereich \term{Measurement} zunächst drei Klassen: \term{MeasurementType}, \term{AccountType}
und deren gemeinsame Oberklasse \term{QuantifiedObjectType}. Auf der Exemplarebene finden sich deren Pendants als \term{Measurement},
\term{Account} und \term{QuantifiedObject}.

Die Operation \term{aggregate} des \term{QuantifiedObjects} fordert zudem eine sogenannte \term{AggregationStrategy},
die in vier Unterklassen konkret ausgeprägt werden kann.
Zusätzlich existieren drei Manager: der \term{MeasurementTypeManager}, der \term{AccountTypeManager} und der \term{AccountManager}.
Diese Klassen werden im Folgenden nach Themengebieten beschrieben.


\subsubsection{QuantifiedObject}

\TODO[Kapitel schreiben]

-- Attribute: MObject object

-- Nur Operation aggregate


\subsubsection{QuantifiedObjectType}

\TODO[Kapitel schreiben]

-- Attribute: MType type, AbsUnitType unitType

-- Keine Operationen


\subsubsection{Account}
Ein Konto (Account) gehört zu einem bestimmten Objekt (Object) und beinhaltet Einträge (entries) in Form von quantitativen Messungen (Measurements). 
Jedes Konto ist in einem Typ (AccountType) klassifiziert und kann außerdem mehrere Unterkonten (Komposition von subAccounts) besitzen.
Auf einem Konto kann auf Basis einer definierten Strategie (AggregationStrategy) aggregiert werden. 
Beispielsweise kann so Summe, Durchschnitt, Maximum oder Minimum der Einträge errechnet werden. 

Im Metamodell besitzt ein Account die folgenden Assoziationen.
\begin{description}
	\item[MAccountType type] Typ des Accounts.
	\item[Account** subAccounts hierarchy AccountHierarchy] Liste von Unterkonten.
	\item[Measurement** entries] Liste von Einträgen.
	\item[MObject object] Von QuantifiedObject geerbt.
\end{description}

Hinzu kommen die folgenden Operationen, welche zum Befüllen der o.g. Listen dienen.
\begin{description}
	\item[addEntry] Fügt einen neues Measurement in die entries-Liste hinzu. Das Element muss vom selben Typ sein wie die anderen Einträge der Liste (Konsistenzbedingungen folgen weiter unten).  
	\item[addSubAccount] Fügt einem Konto ein neues Unterkonto hinzu. Hierbei muss der Typ des neuen Unterkontos $\leq$ dem Typ des Hauptkontos sein (Konsistenzbedingungen folgen weiter unten).
\end{description}


\subsubsection{AccountType}
Der Typ eines Accounts (AccountType) dient zur Klassifizieren eines Kontos. Beispielsweise kann ein Konto als Bankkonto klassifiziert werden. Untertyp (SubAccountType) kann an dieser Stelle z.B. ein Giro- oder Kreditkartenkonto sein.
Sobald ein AccountType Exemplare besitzt, kann dieser nicht mehr gelöscht werden (siehe folgender Abschnitt). Um Hierarchien dieser Art abbilden zu können sind die folgenden Assoziationen und Operationen notwendig. 

Im Metamodell besitzt ein AccountType (neben der vom \term{QuantifiedObjectType} geerbten) die folgende Assoziation.
\begin{description}
	\item[MAccountType** subAccountTypes hierarchy MAccountTypeHierarchy] Diese Liste verbindet AccountTypes in Form einer Hierachie. Besonders bei dem Hinzufügen von SubAccounts ist diese Hierarchie relevant (dazu mehr im folgenden Abschnitt).
\end{description}

Hinzu kommt die folgende Operation, welche zum Befüllen der o.g. Liste dient.
\begin{description}
	\item[addSubAccountType] Fügt einen neuen AccountType der subAccountTypes Liste eines AccountTypes hinzu.
\end{description}


\subsubsection{Account \& AccountType}
Wie bereits beschrieben, ist ein Account in einem AccountType klassifiziert. Dieser Typ ist unter anderem relevant, wenn dem Konto ein neues Unterkonto hinzugefügt werden soll.  
In Abbildung \refImg{Measurement:AccTypeHier} wird eine AccountType Hierarchie dargestellt. \term{A} ist in diesem Beispiel der oberste AccountType, welcher die darunterliegenden SubAccountTypes referenziert. 
Auf \term{F} wird von keinem AccountType referenziert und refernziert auch selbst keine anderen AccountTypes.  

\img[width=\relWidth{0.4}]{measurement/AccountTypeHierarchy.png}{AccountType Hierarchie}{Measurement:AccTypeHier}

Nun sollen Exemplare (also Accounts) erstellt werden, welche SubAccounts enthalten. In Abbildung \refImg{Measurement:SubAccounts} können die im Folgenden beschriebenen Fälle nachvollzogen werden.
Der Einfachheit halber wird mit \term{K1:A} ein Account \term{K1} mit AccountType \term{A} bezeichnet. Da $C \leq A$ ist, darf \term{K2} als SubAccount zu \term{K1} hinzugefügt werden.
In den anderen drei Fällen ist der AccountType vom SubAccount entweder nicht $\leq$ dem Typen des Hauptkontos (Beispiel 2 und 4, von links) oder aber der AccountType ist in einer anderen Hierarchie (Beispiel 3, von links).

\img[width=\relWidth{0.5}]{measurement/SubAccounts.png}{SubAccount Exemplare}{Measurement:SubAccounts}

Daraus ergibt sich die folgende Konsistenzbedingung, welche bei Nichteinhaltung zu einer \term{ConsistencyException} führt.
\begin{equation} \forall \, a \in Account: \forall s \in a.subaccounts: s.type \leq a.type
\end{equation}




\subsubsection{AccountManager}
Der AccountManager dient zur Erstellung von Accounts und beinhaltet alle Accounts des Modells.

Im Metamodell besitzt der AccountManager die folgende Assoziation.
\begin{description}
	\item[Account** accounts] Diese Liste beinhaltet alle Accounts des Modells.
\end{description}

Hinzu kommt die folgende Operation, welche zum Befüllen der o.g. Liste dient.
\begin{description}
	\item[createAccount] Dient zum Erstellen von Accounts. Es wird ein Name, ein \term{MAccountType} sowie ein \term{MObject} übergeben, auf welches sich das Konto bezieht.
\end{description}

Neben den Accounts werden auch die \term{AccountTypes} zentral verwaltet. Dazu dient der \term{AccountTypeManager}.


\subsubsection{AccountTypeManager}
Der AccountTypeManager dient zur Erstellung von \term{AccountTypes} und beinhaltet alle \term{AccountTypes} des Modells.

Im Metamodell besitzt der \term{AccountTypeManager} die folgende Assoziation.
\begin{description}
	\item[MAccountType** accountTypes] Beschreiben
\end{description}

Hinzu kommt die folgende Operation, welche zum Befüllen der o.g. Liste dient.
\begin{description}
	\item[createAccountType] Dient zum Erstellen von \term{AccountTypes}. Es wird ein Name, ein \term{MType} und ein \term{UnitType} übergeben.
\end{description}
\TODO[Alex: Brauchen wir hier noch eine Konsistenzbedingung bzgl. Account-AccountType-UnitType = Account.entries-Measurement.Quantity-Type???]

Wie bereits beschrieben werden in Accounts Messungen in Form von Measurements erfasst. 
Im folgenden Abschnitt wird auf die Assoziationen und Operationen von Measurements genauer eingegangen.
\TODO[Überleitung via account.entries - passt so?]


\subsubsection{Measurement}

\TODO[Kapitel schreiben]

-- Attribute: MMeasurementType type, AbsQuantity quantity

-- Keine Operationen


\subsubsection{MeasurementType}

\TODO[Kapitel schreiben]

-- Keine Attribute

-- Keine Operationen


\subsubsection{Measurement \& MeasurementType}

\TODO[Alex: passt das? Oder denk ich da falsch herum?]
\begin{equation} \forall \, m_1, m_2 \in Account.entries: m_1.type = m_2.type
\end{equation}


\subsubsection{MeasurementTypeManager}

\TODO[Kapitel schreiben]

-- Attribute: MMeasurementType** measurementTypes

-- Operationen:

\begin{description}
	\item[createMeasurementType] Beschreiben
\end{description}


\subsubsection{AggregationStrategy}

\TODO[Kapitel schreiben]

-- Keine Attribute

-- Nur Operation aggregateMeasurements


\subsubsection{Löschen im Bereich Measurement}

\TODO[Kapitel schreiben]

-- Was ist löschbar?
-- Was nicht? Warum? % Inhalt einbinden % Inhalt einbinden

% Kapitel
\section{Ausblick}
Im Laufe des Projekts sind viele Ideen aufgekommen, die aufgrund von Entwurfsentscheidungen oder der kurzen Laufzeit des Projektes 
nicht umgesetzt werden konnten. Dieser kurze Ausblick skizziert eine Auswahl von Erweiterungsmöglichkeiten und 
soll nachträgliche Implementierungen anregen.

\begin{description}
  \item[Multiplizitäten] Zurzeit besitzen Assoziationen keine Einschränkungen. 
	Es werden entsprechend überall $0..n$-Multiplizitäten verwendet. 
	Mit der Einführung von Multiplizitäten können Assoziationen eingeschränkt werden.
	Es bietet sich beispielsweise an, eine Klasse \term{Multiplicity} zwischen \term{Association} und \term{Type} zu hängen,
	welche zwei boolsche Attribute $\leq$ und $\geq$ enthält. Damit lassen sich die vier Multiplizitäten $0..n$, $1..n$, $1$ und $0..1$ darstellen.
	Dies zieht u.~a. einige Konsistenzbedingungen nach sich, welche beim Anlegen von Links beachtet werden müssen. 
  \item[Mehrstellige Assoziationen] Aktuell gibt es für eine Assoziation nur genau einen Quell- und genau einen Zieltypen. 
  	Um mehrstellige Assoziationen und Links zu implementieren, bietet sich die Verwendung von Formal- und Aktualparameter an. 
  	Eine Assoziation besteht dann aus den zwei explizit genannten Quell- und Zieltypen sowie einer Liste von Formalparametern, 
  	wobei jeder Formalparameter eine weitere Quelle darstellt. Um dies adäquat auf der Exemplarebene abzubilden, 
  	bedienen sich die Links ihrer Aktualparameter. Das aktuelle \MM sieht diese Implementierung schon vor, da sowohl 
  	Formal- als auch Aktualparameter von den jeweiligen Oberklassen \term{AbstractOperation} und \term{Message++Link} ausgehen.
  \item[Gleichheit von Operationen] Zum derzeitigen Entwicklungsstand sind zwei Operationen gleich, sobald sie den gleichen Namen haben. 
  	Dies ist in vielen Fällen nicht	sinnvoll, da zwei Operationen keinerlei Bezug zueinander haben, wenn sie in unterschiedlichen Typen liegen. 
  	Selbst dann nicht, wenn sie gleich heißen. Ein praktikableres Konzept ist, dass zwei Operationen gleich sind, sobald sie dieselbe Quelle, 
  	den gleichen Namen und die gleichen Parameterlisten (also Anzahl der Parameter und Typisierung dieser) haben. 
  \item[Messages] Wie schon im Kapitel \ref{Message:Message} erläutert, wurden Messages in der ersten Implementierungsstufe nicht umgesetzt. 
  	Zum gegenwärtigen Zeitpunkt existiert für ein Message-Exemplar sowohl die Interpretation des Nachrichtenaufrufs, als auch des Nachrichteninhalts.
	Bevor Messages umgesetzt werden können, muss entschieden werden, welche der genannten Interpretationen für ein Message-Exemplar zutrifft. 
  \item[ActualParameter] Da aus den genannten Gründen weder Message-Exemplare erzeugt, noch mehrstellige Assoziationen/Links angelegt werden können, 
  	haben Aktualparameter derzeit keinerlei Daseinsberechtigung. Von daher wird dem Anwender die Verwaltung von Aktualparametern nicht angeboten.
  \item[Zentrale Ablage der Constraints] Durch die Einführung von \term{ModelItem} wird die Möglichkeit geschaffen, einen Abhängigkeitsgraphen 
	auszuwerten. Dadurch können alle Modellelemente ermittelt werden, deren Konsistenz nach einer Änderung zu prüfen ist. Wenn alle Konsistenzbedingungen 
	in einer zentralen Komponente abgelegt sind, ist es möglich, nach einer Änderung einfach die Konsistenz aller (transitiv) abhängigen Modellelemente zu überprüfen. 
	Da alle Änderungen in Transaktionen geschehen, kann bei einem inkonsistenten Folgezustand ein Rollback eingeleitet werden.
	Großer Vorteil ist, dass veränderliche Elemente keinerlei Kenntnis über die Konsistenzbedingungen ihrer abhängigen Modellobjekte mehr benötigen.
\end{description}

Weitere Themen neben den oben genannten sind beispielsweise Versionierung, Posting Rules, Path. Diese werden in diesem Dokument jedoch nicht weiter ausgeführt. % Inhalt einbinden

% Abbildungsverzeichnis
\newpage
\resetPageTitle % Wichtig: Seitentitel wieder zurücksetzen, da er auf der letzten Seite geändert wurde
\listoffigures % Abbildungsverzeichnis einfügen

% Tabellenverzeichnis
%\newpage
%\listoftables % Tabellenverzeichnis einfügen

% Abkürzungsverzeichnis
\newpage
\renewcommand*{\glspostdescription}{} % Text am Ende eines Verzeichniseintrags (Standard: ".")
\printglossary[type=\acronymtype,style=super,nonumberlist] % Abkürzungen einbinden

% Glossar
\newpage
\renewcommand*{\glspostdescription}{~--} % Text am Ende eines Glossareintrags (Standard: ".")
\printglossary[style=altlist,nonumberlist=false] % Glossar ausgeben

% Quellenverzeichnis
\newpage
\setPageTitle{Quellenverzeichnis} % Seitentitel ändern, da diese Section ausgeblendet ist
\section*{Quellenverzeichnis}
\addcontentsline{toc}{section}{Quellenverzeichnis} % Fügt diesen Punkt ins Inhaltsverzeichnis ein

% Formatieren des Quellenverzeichnisses (alphadin = Formatierung nach DIN 1505)
\bibliographystyleLiteratur{alphadin} % Literatur formatieren
\bibliographystyleRecht{alphadin} % Rechtsvorschriften formatieren
\bibliographystyleInternet{alphadin} % Internetquellen formatieren
\bibliographystyleUnternehmen{alphadin} % Unternehmensquellen formatieren

\bibliographyLiteratur{\content/lib/Literatur} % Einfügen der Literatur
%\bibliographyRecht{\content/lib/Recht} % Einfügen der Rechtsvorschriften
%\bibliographyInternet{\content/lib/Internet} % Einfügen der Internetquellen
%\bibliographyUnternehmen{\content/lib/Unternehmen} % Einfügen der Unternehmensquellen % Inhalt einbinden

% Beginn des Anhangs
\newpage
\resetPageTitle % Wichtig: Seitentitel wieder zurücksetzen, da er auf der letzten Seite geändert wurde
\appendix % Beginn des Anhangs (diese Zeile auskommentieren, falls kein Anhang vorhanden ist)
\appendixtoc % Schwarze Magie - niemals auskommentieren
% Kapitel einbinden
%\include{\content/sections/appendix/Beispielanhang}

% Kapitel einbinden
%\include{\content/sections/appendix/Weiterer_Anhang} % Inhalt einbinden

\end{document} % Seiten und Kapitel einbinden