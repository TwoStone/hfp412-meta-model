\section{Protokoll vom 18.05.2013}

\subsection{Wiederholung vom letzten Mal}

\TODO[Bild Grundgesamtheit und Stichprobe vgl. Skript 4.3]

\begin{description}
	\item[Kapitel 5] Konfidenzintervalle (KI) -- Schätzverfahren
	\begin{description}
		\item[Kapitel 5.1] Punktschätzer
		\item[Kapitel 5.2] Intervallschätzung für Parameter $\Theta$
	\end{description}
\end{description}

KI für $\mu$: SI für $\hat{\Theta}$ $(\Theta \pm \epsilon)$ $\to$ KI für $\Theta$ $(\hat{\Theta}  \pm \epsilon)$

---

\begin{description}
	\item[1. Fall] $X$ normalverteilt, $\sigma^2$ bekannt $\Rightarrow$ KI für $\mu$:
	\[\bar{x} \pm z_{1-\frac{\alpha}{2}} * \frac{\sigma}{\sqrt{n}}\]
	Das Konfidenzniveau beträgt hierbei $= 1 - \alpha$
	
	\item[2. Fall] $n \geq 30$, $\sigma^2$ bekannt $\Rightarrow$ KI für $\mu$:
	\[\bar{x} \pm z_{1-\frac{\alpha}{2}} * \frac{\sigma}{\sqrt{n}}\]
	Das Konfidenzniveau beträgt hierbei $\approx 1 - \alpha$
	
	\item[3. Fall] $X$ normalverteilt, $\sigma^2$ unbekannt $\Rightarrow$ KI für $\mu$:
	\[\bar{x} \pm t_{n-1; 1-\frac{\alpha}{2}} * \frac{s}{\sqrt{n}}\]
	
	\item[4. Fall] $n \geq 100$, $\sigma^2$ unbekannt $\Rightarrow$ KI für $\mu$:
	\[\bar{x} \pm z_{1-\frac{\alpha}{2}} * \frac{s}{\sqrt{n}}\]
\end{description}


\subsection{\qq{Los gehts}}

\begin{itemize}
  \item Spalten $nm$ bis $rooms$ kopieren
  \item Beispiel: Durchschnittsmiete für eine Zwei-Zimmer-Wohnung
  \item Filter auf $rooms$: Nur Zeilen mit Wert $2$ anzeigen
  \item Jetzt Werte der Spalten $nm$ bis $rooms$ kopieren und darunter einfügen (sonst rechnet er trotzdem mit den
  ausgeblendeten Zeilen)
  \item Um $n$ zu erhalten, benutzen wir folgende Formel: \[=Z\ddot{A}HLEN(A2058:A2772)\]
  Alternativ hätten wir die Formel \[=Z\ddot{A}HLENWENN(A2:A2054;2)\] benutzen können, ohne die Werte kopieren zu müssen
  \item Um die Standardabweichung $s$ zu errechnen, benutzen wir die folgende Formel: \[=STABW.S(A2058:A2772)\]
  \item Um $z_{1-\frac{\alpha}{2}}$ zu berechnen, geben wir ein: \[=NORM.S.INV(0.975)\]
  \item Die Durchschnittsmiete $dm$ berechnen wir mit der Formel \[=MITTELWERT(A2058:A2772)\]
  \item Die Mietober- und Mietuntergrenze können wir nun folgendermaßen berechnen: \[dm \pm z_{1 -
  \frac{\alpha}{2}} * \frac{s}{\sqrt{n}}\]
  und bekommen das Intervall $[474,59; 499]$ heraus
  \item \TODO[Was berechnen wir hier? \qq{Abweichung Konfidenintervall}] \[=KONFIDENZ.NORM(\alpha; s; n)\]
  \[=KONFIDENZ.NORM(0,05; F2061; F2059)\]
  \[= 12,21~\widehat{=}~\epsilon
  ~\widehat{=}~z_{1-\frac{\alpha}{2}} * \frac{s}{\sqrt{n}}
  ~\widehat{=}~z_{1-\frac{\alpha}{2}} * \frac{\sigma}{\sqrt{n}}\]
  Somit können wir die vorherige Formel auch wie folgt aufschreiben:
  \[dm \pm \epsilon = \{Obergrenze; Untergrenze\}\]
  \item Konfidenz für T-Verteilung \[=KONFIDENZ.T(\alpha; s; n)\]
  \[=KONFIDENZ.T(0,05; F2061; F2059)\]
  \item Björn stellt die Frage: \qq{Wie kommt man auf die Grenzen für das Konfidenzintervall?}
  \[=T.INV(0,975; n-1)\]
  \emph{Hinweis: Bis zur Version 2007 kann Excel nur die Konfidenz mit der Normalverteilung errechnen. Der Befehl
  KONFIDENZ $\widehat{=}$ KONFIDENZ.NORM in Excel 2007.}
\end{itemize}

---

 