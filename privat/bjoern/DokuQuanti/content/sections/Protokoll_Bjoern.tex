\section{Definition und Notationen}\label{sec:defNot}
Die Menge aller für die Unterschung relevanten Merkmalsträger wird in den statistischen
Wissenschaften als \qq{Grundgesamtheit} bezeichnet. Die Menge aller in der Untersuchung auftretenden
Merkmalsträger wird \qq{Stichprobe} genannt. Die Gesamtheit aller Daten (Merkmale, Merkmalsträger
und Merkmalsausprägungen) aus der Stichprobe wird als Beobachtungsdaten oder \qq{Urliste} bezeichnet
(siehe dazu \refImg{img_grundgesamtheitStichprobe}) (vgl. \refLibLiteratur[S.~11~ff.]{GasnierStat}).

\img[width=\relWidth{0.9}]{grundgesamtheitStichprobe.pdf}{Zusammenhang zwischen Grundgesamtheit und
Stichprobe}{img_grundgesamtheitStichprobe}
 
Damit die erhobenen Daten von einem EDV System (z.B. Microsoft Excel oder SPSS) verarbeitet werden
können, müssen diese zunächst aufbereitet werden. Das bedeutet, dass die Informationen, die
beispielsweise in ausgefüllten Fragebögen enthalten sind, in Form einer Datenmatrix aufbereitet
werden müssen. Dafür ist es außerdem wichtig, die einzelnen Fragen (Variablen) und Antworten
(Ausprägungen) zu codieren. Hierbei sollte darauf geachtet werden, dass die Codierung vollständig
ist. Es muss beispielsweise bedachte werden, dass eine Person die Antwort unter Umständen
verweigern kann. Aus diesem Grund wird meist eine Kategorie \qq{missing value} hinzugefügt, welche
dann in der Ausprägung z.B. den Wert \qq{0} erhält. 

Die Datenmatrix setzt sich (mithilfe der Codierung) somit aus aus Merkmalsträgern (X-Achse) und den 
zugehörigen Ausprägungen des Merkmals (Y-Achse) zusammen (siehe \refImg{img_merkmaleAuspraegungen}).

\img[width=\relWidth{0.5}]{merkmaleAuspraegungen.pdf}{Datenmatrix - Merkmale und
Ausprägungen}{img_merkmaleAuspraegungen}

Genauer besteht die Datenmatrix aus $m$ Merkmalsträgern: $j=1,...,m$ als Laufindex für den
$j$-ten Merkmalsträger. Als $X$ wird das allgemeine Merkmal bezeichnet.

$x_j$ bezeichnet die Ausprägung des Merkmals $X$ für den $j$-ten Merkmalsträger. 

Die Datenreihe oder \qq{Urliste} (eine Spalte der Datenmatrix) für ein Merkmal $X$
lautet dann: \[x_1, x_2 ,..., x_j ,..., x_{m-1},x_m\]

Je nach Skalenniveau wird man mit dieser Urliste mehr oder weniger anfangen können. Die
Daten des Münchner Mietspiegels von 2003 lassen sich für die in dieser Ausarbeitung
behandelten Skalenniveaus (nominale und ordinale) problemlos als Datenmatrix verwenden.


\section{Nominalskala}

Die Nominalskala dient zur Klassifikation und Identifikation von Untersuchungsobjekten (z.B.
die Merkmale Geschlecht oder Fakultätszugehörigkeit bei Studierenden). Die Analyse nominalskalierter
Daten beschränkt sich auf Häufigkeitsanalysen (vgl. \refLibInternet{gabler:nominalskala} und
\refLibInternet{gabler:skalenniveau}).

Ein Merkmal skaliert nominal (v. lat. nomen \qq{Name} aus griech. onoma; Pl.: Nomina, auch Nomen), 
wenn seine möglichen Ausprägungen zwar unterschieden werden können, aber keine natürliche Rangfolge aufweisen.
Ein solches Merkmal wird messbar gemacht durch eine Beschreibung von Kategorien,
nach der jede Untersuchungseinheit (genau) einer Kategorie zugeordnet werden kann. 
Das Ergebnis einer solchen Operationalisierung heißt dann \qq{Nominalskala}. 
Wegen des Fehlens der Ordnung ist dabei -skala (von lat. scalae ‚Leiter, Treppe‘) eigentlich nicht
angemessen und ist im Zusammenhang mit den anderen Skalenniveaus zu sehen (vgl.
\refLibInternet{wiki:nominalskala}).

Eine Nominalskala muss die folgenden Eigenschaften erfüllen:
\begin{description}
\item[Reflexivität:] $a \sim a$. Jedes Objekt ist zu sich selbst äquivalent.
\item[Symmetrie:] $a \sim b \implies b \sim a$. Wenn $a$ zu $b$ äquivalent ist, dann ist auch $b$ äquivalent zu $a$ (und umgekehrt).
\item[Transitivität:] $a \sim b \land b \sim c \implies a \sim c$. Wenn $a$ zu $b$ äquivalent und $b$ zu $c$ äquivalent ist, dann ist $a$ äquivalent zu $c$.
\item[Homomorphie:] Die Beschreibung der Kategorien muss so erfolgen, dass die dadurch definierte Abbildung strukturerhaltend (homomorph) ist.
\end{description}

Beispiele für nominalskalierte Merkmale sind:
\begin{description}
  \item[Geschlecht:] männlich, weiblich
  \item[Geburtsort:] Hannover, München, Berlin
  \item[Religionszugehörigkeit:] evangelisch, katholisch, muslimisch
  \item[Stadtbezirk:] Nordstadt, Südstadt, Oststadt, Weststadt, Mitte
\end{description}



\subsection{Häufigkeitstabelle}
Bei nominalen Merkmalen, können die vorhandenen Merkmalsausprägungen ($x_i,i=1,...,k$) aufgelistet
und deren jeweilige Anzahl ($n_i$ = absolute Häufigkeiten) ermittelt werden.
So entsteht eine Häufigkeitstabelle.

Im Folgenden wird mit den Daten des Mietspiegels eine Häufigkeitstabelle bzgl. der Bezirke erstellt.

\begin{enumerate}
  \item Die Spalte \qq{bez} (Bezirke) kopieren, in ein neues Arbeitsblatt einfügen, anschließend
  sortieren und benennen. Damit wird eine geordnete Urliste erzeugt.
  \item Ein Blick auf den Filter verrät, dass 25 verschiedene Bezirke existieren.
  \item Drei Spalten mit den Überschriften $x_i$ (vorhandene Ausprägungen), $n_i$ (Häufigkeiten) und
  $\frac{n_i}{n}$ (Relative Häufigkeit) versehen.
  \item Die Spalte $x_i$ mit Zahlen von 1 bis 25 versehen.
  \item Die Spalte $n_i$ wird mit der Excel Funktion $\text{ZÄHLENWENN}$ versehen.
  \[=\text{ZÄHLENWENN}(bez;x_i)\] Beispielsweise sieht der
  Eintrag für die Zelle $E4$ wie folgt aus (wobei die Urliste von $A2$ bis $A2054$ geht und
  die Spalte $D$ die Überschrift $x_i$ trägt): \[=\text{ZÄHLENWENN}(A2:A2054;D4)=43\]
  \item Die genannte Excel Formel wird für jede der 25 Ausprägungen verwendet, wobei Beginn und Ende
  der Urliste gleich bleiben und sich nur der Eintrag $x_i$ verändert.
  \item Die Spalte der Ausprägungen wird anschließend summiert und ergibt den Wert $2053$, welcher
  die Anzahl der Datensätze aus der Urliste ist.
  \item Durch das Teilen des Wertes der Häufigkeiten ($n_i$) durch die soeben
  ausgerechnete Summe, errechnet sich die relative Häufigkeit. \[=n_i/SummeHaeufigkeiten\]
  Beispielsweise sieht der Eintrag für die Zelle $F4$ wie folgt aus (wobei in $E29$ die Summe der
  Häufigkeiten steht): \[=E4/E29=2,1\%\]
\end{enumerate} 
 
\newpage
Es ergibt sich die folgende Häufigkeitstabelle:

\begin{table}[h]
\caption{Häufigkeitstabelle für Bezirke}\label{tab:haeufigkeitenBez}
\begin{center}
\begin{tabular}{p{1.25cm}p{1.25cm}p{1.25cm}}
\hline \hline
\textbf{$x_i$} & \textbf{$n_i$} & \textbf{$\frac{n_i}{n}$}\\ 
\hline
1	&	43	&	2,1\% \\
2	&	161	&	7,8\% \\
3	&	132	&	6,4\% \\
4	&	137	&	6,7\% \\
5	&	139	&	6,8\% \\
6	&	66	&	3,2\% \\
7	&	69	&	3,4\% \\
8	&	62	&	3,0\% \\
9	&	177	&	8,6\% \\
10	&	58	&	2,8\% \\
11	&	70	&	3,4\% \\
12	&	78	&	3,8\% \\
13	&	98	&	4,8\% \\
14	&	60	&	2,9\% \\
15	&	43	&	2,1\% \\
16	&	115	&	5,6\% \\
17	&	67	&	3,3\% \\
18	&	82	&	4,0\% \\
19	&	106	&	5,2\% \\
20	&	50	&	2,4\% \\
21	&	56	&	2,7\% \\
22	&	24	&	1,2\% \\
23	&	14	&	0,7\% \\
24	&	29	&	1,4\% \\
25	&	117	&	5,7\% \\
\hline
$\Sigma$ & \textbf{2053}
\end{tabular}
\end{center}
\label{default}
\end{table}%  


\subsection{Darstellung als Diagramm}
Nominale Merkmale, bzw. die Daten der Häufigkeitstabelle lassen sich ebenso gut als Kreisdiagramm
darstellen. Säulendiagramme für die einzelnen Häufigkeiten und Treppenfunktionen für die kumulierten
Häufigkeiten sind erst bei ordinalen, intervallskalierten oder diskreten Merkmalen sinnvoll (im
nächsten Abschnitt wird im Zuge der ordinalen Merkmale genauer darauf eingegangen).

Um ein Kreisdiagramm in Microsoft Excel darzustellen, müssen die folgenden Schritte durchlaufen
werden:

\begin{enumerate}
  \item Spalten $x_i$ und $n_i$ der Häufigkeitstabelle markieren.
  \item Auf \qq{Einfügen -> Kreisdiagramm} klicken.
\end{enumerate}

\img[width=\relWidth{0.8}]{nominalTorte.pdf}{Kreisdiagramm}{img_nominalTorte}


\subsection{Maßzahlen / Kennzahlen}
In der deskriptiven Statistik existieren verschiedene Kennzahlen, welche es ermöglichen, sich mit wenigen quantitativen Daten 
bereits eine gute Übersicht über Verteilungen, Mittelwerte, etc. zu verschaffen. 
Es gibt unter anderem die folgenden Kennzahlen, welche zu den Lagemaßen gehören und Auskunft über das Zentrum einer Verteilung geben (vgl. \refLibInternet{wiki:kennzahl}).
\begin{itemize}
  	\item Arithmetisches Mittel
	\item Median
	\item Modalwert
\end{itemize}

Der Modus oder Modalwert ist bei einer empirischen Häufigkeitsverteilung der häufigste Wert. Diese
Maßzahl ist für alle Daten (mindestens nominal) anwendbar. Der Modus ist für ordinale oder diskrete
Merkmale die Ausprägung mit der höchsten Häufigkeit. Bei stetigen Merkmalen wird die Mitte der
Klasse mit der größten Häufigkeitsdichte als Modus bezeichnet, wobei die Bezeichnung \qq{dichtester
Wert} verwendet wird. Es wird außerdem unterschieden zwischen unimodaler, bi- oder multimodaler
Verteilung, welche die Anzahl der Maxima bestimmt. Die Aussagekraft dieser Messzahl ist jedoch eher schwach 
(vgl. \refLibLiteratur[S.~18~ff.]{GasnierStat}).
\begin{description}
  \item[Unimodal:] Nur ein Maximum. Im folgenden Beispiel ist die häufigste Ausprägung 4 (8
  Beobachtungen) und somit der Modus \qq{4}. \[\{1,1,1,1,2,2,2,3,4,4,4,4,4,4,4,4,5,5\}\]
  \item[Bimodal:] Genau zwei Maxima. Das folgende Beispiel besitzt zwei Modi, \qq{2}
  und \qq{5}, je mit der Häufigkeit 2. \[\{1,2,2,3,4,5,5,6,7\}\]
  \item[Multimodal:] Mehr als zwei Maxima. Beispielsweise \qq{2}, \qq{5} und
  \qq{6}. \[\{1,2,2,3,4,5,5,6,6,7\}\]
\end{description} 

Neben der Methode des \qq{scharfen Hinsehens} gibt es in Microsoft Excel die Funktion $\text{MODALWERT}$ um den Modus auszurechnen.
Angewendet auf das Beispiel (siehe Tabelle \ref{tab:haeufigkeitenBez}) ergibt sich somit: \[=\text{MODALWERT}(bez)=9=8,6\%\]
In Worten: Der Bezirk \qq{9} kommt am häufigsten vor (nämlich 177 mal, was 8,6\% entspricht). Da dieser Wert das einzige Maximum ist, trifft außerdem die Eigenschaft der \qq{Unimodalität} zu.

\section{Ordinalskala}
Die Ordinalskala ordnet die Untersuchungsobjekte nach ihrem Rang (z.B. Rating A ist besser als
Rating B), sagt jedoch nichts über das Ausmaß der Unterschiede aus. Zulässige mathematische
Operationen bei ordinalskalierten Daten sind beispielsweise die Berechnung des Modus und des
Medians (vgl. \refLibInternet{gabler:skalenniveau}).

Genauer dient die Ordinalskala der Charakterisierung von (Zufalls-)Variablen mit Ausprägungen,
zwischen denen eine natürliche Rangordnung besteht.
Ordinal-Variablen enthalten also Nominal-Informationen und auch Informationen über die Reihung (Ordnung) der Variablenwerte. 
Beobachtungen auf einem Merkmal mit ordinalem Messniveau können hinsichtlich dieses Merkmals gruppiert und ihrer Größe nach geordnet werden.
Werden die Merkmalsausprägungen (Kategorien) mit (Rang-)Zahlen (Ordnungsziffern) bezeichnet, werden diese so gewählt, 
dass die Rangfolge der Zahlen der Rangfolge der Ausprägungen entspricht. D.h. eine Beobachtung bzw. ein Objekt mit einem höheren Rang besitzt 
auch eine höhere Ausprägung auf dem betrachteten Merkmal als eine Beobachtung mit einem niedrigeren Rang. 
Über die Größe des Merkmalsunterschieds zwischen den Objekten, d.h. die Abstände zwischen den Rangplätzen, 
lässt sich aber keine Aussage machen (vgl. \refLibInternet{gabler:ordinalskala}).


Zusätzlich zu den Bedingungen zur Konstruktion einer Nominalskala erfordert die Konstruktion einer Ordinalskala die Eigenschaft der Trichotomie (Dreiteilung)\footnote{Diese Eigenschaft gilt in jeder total geordneten Menge.}:
\begin{description}
	\item[Trichotomie:] Es gilt eine der folgenden Eigenschaften:
		\begin{itemize}
		  \item $a < b$
		  \item $a = b$
		  \item $a > b$ 
		\end{itemize} 
\end{description}

Nachfolgende Auflistung enthält Beispiele für ordinalskalierte Merkmale (vgl. \refLibInternet{wiki:ordinalskala}).
\begin{description}
  \item[Zufriedenheit mit einem Produkt:] sehr zufrieden > eher zufrieden > eher unzufrieden > sehr unzufrieden
  \item[Schulische Leistung:] sehr gut > gut > befriedigend > ausreichend > mangelhaft > ungenügend
  \item[Selbsteinstufung des Einkommens:] hoch > mittel > niedrig\footnote{Wird das Einkommen in
  Klassen eingeteilt (z.B. 0 bis 999 Euro, 1000 bis 2000 Euro, über 2000 Euro), 
  handelt es sich um ein ordinal skaliertes Merkmal. Wird dagegen der genaue Betrag erhoben und statistisch verarbeitet, liegt ein metrisches Merkmal vor. 
  Da die Auskunftsbereitschaft bei der Angabe des genauen Einkommens geringer ist, wird in vielen
  Umfragen auf eine Abfrage der Einkommensklassen zurückgegriffen.}
  \item[Dienstrang beim Militär:] General > Major > Leutnant > Feldwebel > Unteroffizier > Gefreiter
\end{description} 




\subsection{Häufigkeitstabelle}
Im Folgenden wird mit den Daten des Mietspiegels eine Häufigkeitstabelle bzgl. der Wohnqualität
erstellt. Zuvor stellen wir jedoch fest, dass die Wohnqualität über zwei Merkmale (Wohngut und
Wohnbest) festgestellt wurde. Diese müssen zunächst zusammengeführt werden, damit sich die Codierungen nicht
überdecken (Wohngut(2) = Wohnbest(1)). 

Neben den Spalten für die vorhandenen Ausprägungen, Häufigkeiten und relative Häufigkeit werden außerdem die kumulierte Häufigkeit $S_i$ und die kumulierte relative Häufigkeit $F_i$ mit hinzugefügt.
\begin{description}
\item[Kumulierte Häufigkeit oder Summenhäufigkeit ($S_i$):] Summe der Häufigkeiten der Merkmalsausprägungen von der kleinsten Ausprägung bis hin zu der jeweils betrachteten Schranke (vgl. \refLibInternet{wiki:kumhaeufigkeit}).
\item[Kumulierte relative Häufigkeit ($F_i$):] Gibt den Anteil der Elemente einer Menge wieder, bei denen eine bestimmte Merkmalsausprägung vorliegt. Sie wird berechnet, indem die absolute Häufigkeit eines Merkmals in einer zugrundeliegenden Menge durch die Anzahl der Objekte in dieser Menge geteilt wird (vgl. \refLibInternet{wiki:relhaeufigkeit}).
\end{description}

Es sind die folgenden Schritte in Microsoft Excel notwendig:

\begin{enumerate}
  \item Die neue zusammengeführte Spalte \qq{wohnquali} in ein neues Arbeitsblatt kopieren,
  sortieren und benennen.
  \item Ein Blick auf den Filter verrät, dass 3 verschiedene Wohnqualitäten existieren.
  \item Fünf Spalten mit den Überschriften $x_i$ (vorhandene Ausprägungen), $n_i$
  (Häufigkeiten), $\frac{n_i}{n}$ (relative Häufigkeit), $S_i$ (kumulierte Häufigkeit) und $F_i$
  (Kumulierte relative Häufigkeit) versehen.
  \item Die Spalte $x_i$ mit Zahlen von 0 bis 2 versehen.
  \item Die Spalte $n_i$ wird mit der Excel Funktion $\text{ZÄHLENWENN}$ versehen.
  \[=\text{ZÄHLENWENN}(wohnquali;x_i)\] Beispielsweise sieht der
  Eintrag für die Zelle $E4$ wie folgt aus (wobei die Urliste von $A2$ bis $A2054$ geht und
  die Spalte $D$ die Überschrift $x_i$ trägt): \[=\text{ZÄHLENWENN}(A2:A2054;D4)=1205\]
  \item Die genannte Excel Formel wird für jede der 3 Ausprägungen verwendet, wobei sich nur der Eintrag $x_i$ verändert.
  \item Die Spalte der Ausprägungen wird anschließend summiert und ergibt den Wert $2053$, welcher
  die Anzahl der Datensätze aus der Urliste ist.
  \item Durch das Teilen des Wertes der Häufigkeiten ($n_i$) durch die soeben
  ausgerechnete Summe, errechnet sich die relative Häufigkeit. \[=n_i/SummeHaeufigkeiten\]
  Beispielsweise sieht der Eintrag für die Zelle $F4$ wie folgt aus (wobei in $E29$ die Summe der
  Häufigkeiten steht): \[=E4/E29=58,7\%\]
  \item Die Excel Funktion $\text{HÄUFIGKEIT}$ errechnet die kumulierte Häufigkeit. 
  Für die Zelle $G4$ sieht der Eintrag beispielsweise wie folgt aus: \[=\text{HÄUFIGKEIT}(wohngut;E4)=1205\] 
  \item Die kumulierte relative Häufigkeit errechnet sich dementsprechend durch die Division von kumulierter Häufigkeit durch die Summe. 
  Für die Zelle $H4$ sieht der Eintrag beispielsweise wie folgt aus: \[=G4/E29=58,7\%\]
\end{enumerate} 
 
Es ergibt sich die folgende Häufigkeitstabelle:

\begin{table}[h]
\caption{Häufigkeitstabelle für Wohnqualität} \label{tab:excel_bsp}
\begin{center}
\begin{tabular}{p{1.25cm}p{1.25cm}p{1.25cm}p{1.25cm}p{1.25cm}}
\hline \hline
\textbf{$x_i$} & \textbf{$n_i$} & \textbf{$\frac{n_i}{n}$} & \textbf{$S_i$} & \textbf{$F_i$}\\ 
\hline
0 & 1205 & 58,7\% & 1205 & 58,7\% \\
1 & 803 & 39,1\% & 2008 & 97,8\% \\
2 & 45 & 2,2\% & 2053 & 100\% \\
\hline
$\Sigma$ & \textbf{2053}
\end{tabular}
\end{center}
\label{default}
\end{table}% 

Auf die relative Summenhäugigkeit bezogen, kann man die folgenden Aussagen treffen:
\begin{enumerate}
\item Anteil der Werte, die mehr als $X_i$ haben: $1-F_i$
\item Anteil der Werte, die mindestens $X_i$ haben: $1-F_{i-1}$
\end{enumerate}

Betrachten wir bezüglich dieser beiden Aussagen den Wert $F_1=97,8\%$. 
Dieser sagt zunächst aus, dass höchstens 97,8\% der Wohnungen eine gute Qualität aufweisen.
Außerdem sagt der Wert in Verbindung mit Aussage 1 aus, dass $1-97,8\%=2,2\%$ eine bessere Qualität als \qq{gut} aufweisen 
und in Verbindung mit Aussage 2, dass mindestens $1-F_0=1-58,7\%=41,3\%$ der Wohnungen eine gute Qualität besitzen.

 
\subsection{Darstellung als Diagramm}
Die Häufigkeiten sowie die kumulierten Häufigkeiten lassen sich als Balkendiagramm darstellen (siehe \refImg{img_darthaeufigkeit} und \refImg{img_treppenfunktion}).
Bei den kumulierten Häufigkeiten wird die Verteilfunktion in Form einer sogenannten Treppenfunktion dargestellt (Balkendiagramm ohne Zwischenräume).
\img[width=\relWidth{0.8}]{darstderhaeufigkeit.pdf}{Darstellung der
Häufigkeiten ($x_i,n_i$)}{img_darthaeufigkeit}


\img[width=\relWidth{0.8}]{treppenfunktion.pdf}{Darstellung der kum. Häufigkeiten, Vert.-Funktion:
Treppenfunktion ($x_i,F_i$)}{img_treppenfunktion}



\subsection{Maßzahlen / Kennzahlen}
Wie bereits im letzten Abschnitt angesprochen, ist der Modus bereits ab einem nominalen Skalenniveau
bestimmbar. Aus diesem Grund wird auf diesen Wert an dieser Stelle nicht weiter eingegangen. Ab einem
ordinalen Skalenniveau können Prozentpunkte (Quantile) allgemein bestimmt werden (vgl. \refLibLiteratur[S.~18~ff.]{GasnierStat}). 

Ein $p$-Quantil ist ein Lagemaß in der Statistik, wobei $p$ eine reelle Zahl zwischen 0 und 1 ist. 
Das $p$-Quantil ist ein Wert einer Variablen oder Zufallsvariablen, der die Menge aller Merkmalswerte (die Verteilung) in zwei Abschnitte unterteilt: 
Links vom p-Quantil liegt der Anteil $p \equiv p \cdot 100\,\%$ aller Beobachtungswerte oder der Gesamtzahl der Zufallswerte oder der Fläche unter der Verteilungskurve; 
rechts davon liegt der jeweilige restliche Anteil $1-p \equiv (1-p) \cdot 100\,\%$. Die Zahl p heißt auch der Unterschreitungsanteil (vgl. \refLibInternet{wiki:quantil}).

Das meist verwendete Quantil ist der Median oder auch Zentralwert, bei welchem $p=0,5$ ist (bezeichnet als $X_{0,5}$) und 
welcher dadurch die Beobachtungswerte halbiert. Bei geradem $n$ ist der Median = Durchschnitt aus den beiden mittleren benachbarten Werten. 
Bei ungeraden Werten der mittlere Punkt. Man muß garantieren, dass $F(x0,5) \leq 0,5$ (vgl. \refLibLiteratur[S.~19~ff.]{GasnierStat}).

Der Median für das Beispiel bzgl. der Wohnqualität errechnet sich wie folgt:
\[
	n=2053
	X_{0,5}=X_{1027}
\]

Der Median (n ist ungerade) ist der 1027. Wert der geordneten Reihe, also $0$. 

Neben dem Median ($X_{0,5}$) gibt es noch weitere Quantile bzw. Quartile (lat. \qq{Viertelwerte}). Das untere (Q1, $X_{0,25}$) und das obere Quartil (Q3, $X_{0,75}$).
Der Median entspricht hierbei dem Q2. Bezogen auf das Beispiel ist das untere Quartil ist gleich \qq{0}, d.h. 25\% der Wohnungen besitzen eine Standard-Qualität. Das obere Quartil ist gleich \qq{1}, d.h. 
75\% der Wohnungen besitzen höchstens eine gute Qualität.

Microsoft Excel für die angesprochenen Maßzahlen auch entsprechende Funktionen an. 
Um den Median zu berechnen, gibt es die Funktion $\text{MEDIAN}$, welche für unser Beispiel mit \qq{wohnquali} befüllt wird. 
Außerdem bietet Excel für die Berechnung der jeweiligen Quartile die Funktion $\text{QUARTILE}$, welche neben der Matrix (wohnquali) einen Parameter
für das gewünschte Quartil bereithält. Hier kann schnell festgestellt werden, dass $\text{MEDIAN(wohnquali)}=\text{QUARTILE(wohnquali;2)}$ ist. 