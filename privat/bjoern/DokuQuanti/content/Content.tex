\begin{document} % Ab hier beginnt der wirkliche Text, alles nachfolgende erscheint im Dokument
\resetPageTitle % Seitentitel initial setzen

% Titelseite
\pagenumbering{Roman} % Nummerierung der Seiten in römischen Zahlen
\thispagestyle{empty} % Kopf- und Fußzeile ausblenden
\begin{center}

~\vfill

\titlepageBox{
	\imgUnlabeled[width=\relWidth{0.35}]{logo-fhdw.pdf}
}

\titlepageBoxHeadline{Fachhochschule für die Wirtschaft Hannover}{-- FHDW --}

\titlepageBoxHeadline{Dokumentation des Projekts}{\textbf{Metamodell}}

\titlepageBoxHeadline{Verfasser:}{
	Alexander Bellhäuser,
	Björn Bodensieck,\\
	Malte Kastner,
	Thimo Koenig,\\
	Stefan Pietsch,
	Stefanie Rademacker,\\
	Marius Tempelmeier,
	Niklas Walter,\\
	Christin Weckbrod,
	Patrick Wolf
}

\titlepageBoxHeadline{1. Theoriequartal}{%
	Studiengang Master of Science (M.Sc.)\\%
	Business Process Engineering%
}

\titlepageBoxHeadline{Eingereicht am:}{\submissionDate}

\end{center} % Inhalt einbinden

% Abstract
%\newpage
%\setPageTitle{Abstract} % Seitentitel ändern, da diese Section ausgeblendet ist
%\section*{Abstract}

 % Inhalt einbinden

% Inhaltsverzeichnis
\newpage
\resetPageTitle % Wichtig: Seitentitel wieder zurücksetzen, da er auf der letzten Seite geändert wurde
\maintoc % Inhaltsverzeichnis einfügen

% Beginn des Inhalts
\newpage
\pagenumbering{arabic} % Nummerierung der Seiten in arabischen Zahlen
\setcounter{page}{1} % Seitenzahlen zurück auf Anfang setzen
\glsresetall % Alle verwendeten Abkürzungen zurücksetzen

% Kapitel
\section{Einleitung}

Im Rahmen des Master Studiums (M.Sc. Studiengang: Business Process Engineering) an der FHDW Hannover
war es für die Studiengruppe \qq{HFP412} die Aufgabe, Inhalte und Hintergründe der Veranstaltung \qq{Quantitative Forschungsmethoden}\footnote{Dozentin: Frau Dr. Sylvie Gasnier} zu dokumentieren.

\img[width=\relWidth{0.9}]{skalenniveaus.pdf}{Verschiedene Zahlen- oder Skalenniveaus}{img_skalenniveaus}

Die vorliegende Ausarbeitung konzentriert sich auf den Zweig der qualitativen Skalen. Genauer das Thema der \qq{Nominal- und Ordinalskalen} (siehe \refImg{img_skalenniveaus}). 
Die theoretischen Hintergründen werden anhand von praktischen Beispielen mit Hilfe von Microsoft Excel untermauert. 
Als Datenquelle dient hierfür der Münchner Mietspiegel aus dem Jahre 2003\footnote{\url{http://data.ub.uni-muenchen.de/2/1/miete03.asc}}.

Um die statistische Auswertung zu erleichtern, müssen einige Vorarbeiten geleistet werden. Zunächst müssen pro Merkmalsart die verschiedenen Ausprägungen 
analysiert werden. Hierbei entstehen ein Codeplan und eine Datenmatrix, welche als Grundlage für die folgenden Auswertungen dient. 
Auf die konkreten Ausprägungsarten wird zu einem späteren Zeitpunkt eingegangen. Um an die
Ausprägungsarten herzuführen, werden in Abschnitt \ref{sec:defNot} zunächst Codeplan und Datenmatrix
skizziert.
 % Inhalt einbinden

% Protokoll Björn (Vorlesung vom 13.04.2013)

\section{Nominale Merkmale}
Allgemeines Geplänkel zum Thema nominale Merkmale\ldots

\subsection{Häufigkeitstabelle}

\subsection{Darstellung als Diagramm}

\subsection{Messzahlen / Kennzahlen}


\section{Ordinale Merkmale}
Allgemeines Geplänkel zum Thema ordinale Merkmale\ldots
\subsection{Häufigkeitstabelle}

\subsection{Darstellung als Diagramm}
Darstellung der Häufigkeit (Balken- / Säulendiagramm)

Darstellung der kumulierten Häufigkeit (Verteilfunktion: Treppenfunktion)

\subsection{Messzahlen / Kennzahlen}

Perzentile - Prozentpunkte (Quantile) als Lagemaße

Median
 
Quartile % Inhalt einbinden

% Abbildungsverzeichnis
\newpage
\resetPageTitle % Wichtig: Seitentitel wieder zurücksetzen, da er auf der letzten Seite geändert wurde
\listoffigures % Abbildungsverzeichnis einfügen

% Tabellenverzeichnis
%\newpage
%\listoftables % Tabellenverzeichnis einfügen

% Abkürzungsverzeichnis
\newpage
\renewcommand*{\glspostdescription}{} % Text am Ende eines Verzeichniseintrags (Standard: ".")
\printglossary[type=\acronymtype,style=super,nonumberlist] % Abkürzungen einbinden

% Glossar
\newpage
\renewcommand*{\glspostdescription}{~--} % Text am Ende eines Glossareintrags (Standard: ".")
\printglossary[style=altlist,nonumberlist=false] % Glossar ausgeben

% Quellenverzeichnis
\newpage
\setPageTitle{Quellenverzeichnis} % Seitentitel ändern, da diese Section ausgeblendet ist
\section*{Quellenverzeichnis}
\addcontentsline{toc}{section}{Quellenverzeichnis} % Fügt diesen Punkt ins Inhaltsverzeichnis ein

% Formatieren des Quellenverzeichnisses (alphadin = Formatierung nach DIN 1505)
\bibliographystyleLiteratur{alphadin} % Literatur formatieren
\bibliographystyleRecht{alphadin} % Rechtsvorschriften formatieren
\bibliographystyleInternet{alphadin} % Internetquellen formatieren
\bibliographystyleUnternehmen{alphadin} % Unternehmensquellen formatieren

\bibliographyLiteratur{\content/lib/Literatur} % Einfügen der Literatur
\bibliographyRecht{\content/lib/Recht} % Einfügen der Rechtsvorschriften
\bibliographyInternet{\content/lib/Internet} % Einfügen der Internetquellen
\bibliographyUnternehmen{\content/lib/Unternehmen} % Einfügen der Unternehmensquellen % Inhalt einbinden

% Beginn des Anhangs
%\newpage
%\resetPageTitle % Wichtig: Seitentitel wieder zurücksetzen, da er auf der letzten Seite geändert wurde
%\appendix % Beginn des Anhangs (diese Zeile auskommentieren, falls kein Anhang vorhanden ist)
\appendixtoc % Schwarze Magie - niemals auskommentieren
%% Kapitel einbinden
\section{Beispielanhang}

\blindtext

% Kapitel einbinden
\section{Weiterer Anhang}

\blindtext % Inhalt einbinden

\end{document}